\setchapterpreamble[u]{\margintoc}
\chapter{Cuantización canónica del campo escalar libre}
\labch{Part}

\begin{center}
  \large Todo esto lo he sacado del \cite{Dobdado}
\end{center}
\section{Cuantización canónica del campo bosónico libre}

Una vez visto lo que es un campo clásico vamos a cuantizarlo, para ello vamos primero a ver como se cuantizaban las cosas. 

En mecánica clásica teniamos el Hamiltoniano el cual estaba definido como función de $(q_{i},p_{i})$, sin embargo, en mecánica cuantica usamos operadores

\[\left.\begin{array}{c}
  q_{i} \xrightarrow{Cuantizamos}\hat{q}_{i} \\
  p_{i} \xrightarrow{Cuantizamos}\hat{p}_{i}
\end{array}\right\} \comm{\hat{q}_{i}}{\hat{p}_{i}}=i\delta_{ij}\]

Como podemos sospechar que en vez de tener variables como $(q_{i},p_{i})$ tengamos los operadores $(hat{q}_{i},\hat{p}_{i})$ complica un poco las cosas, ya que habrá que asegurarse siempre que se cumplan ciertas reglas de conmutación.

Si $(q_{i},p_{i})$ han pasado a convertirse en operadores $H$, tambien lo será, es decir, $\hat{H}(hat{q}_{i},\hat{p}_{i})$. 

En mecánica cuántica la ecuación que describía la dinámica de un sistema era la ecuación de Schrödinger, y su variable dinámica era la función de onda 
\[i\dv{}{t}\ket{\psi(t_{0})}=\hat{H}\ket{\psi(t_{0})}\]

En este caso nos encontramos con un problema, ya que la variable dinámica es la función de onda, y es la que tiene la dependencia temporal, y queremos irnos a un campo en donde no hay funciones de onda. La solución a este problema es la \textbf{Imagen de Heisenberg}, la cual nos permite poner la dependencia temporal en los operadores en vez de en las funciones de onda, es decir, 

\[q_{i}\longrightarrow \hat{q}_{i}(t)=e^{i \hat{H}t}\hat{q}_{i}e^{-i \hat{H}t}=\mathcal{U}^{-1} \hat{q}_{i}\mathcal{U}\]

como el operador evolución temporal $\mathcal{U}$ cumple la relación $\mathcal{U}^{-1}\mathcal{U}=\mathbb{I}$, entonces conservamos la relación de conmutación de la imagen de Schrödinger 

\[\comm{\hat{q}_{i}(t)}{\hat{p}_{i}(t)}=i\delta_{ij}\]

Basicamente hemos movido la dinámica a los operadores, y esto es importante porque cuando pasamos (en el tema anterior) al formalismo de campos, pasamos las coordenadas generalizadas, en ningún momento hablamos de funciones de onda. 

\subsection{Cuantización en teoría de campos}

Ahora una vez visto como cuantizamos cosas y sabiendo como se pasa de la coordenadas generalizadas a un campo, vamos a juntar las dos cosas y ver que pasa, para ello expresemos los operadores $\hat{q}_{i}(t)$ como campos

\[\hat{q}_{i}(t)\longrightarrow \hat{\phi}(\vec{x},t)\]
\todo{El subindice $i$ hace referencia a las coordenadas $\vec{x}$, es decir, $q\rightarrow\phi$ y $i\rightarrow \vec{x}$}

En teoría clasica de campos podiamos obtener un campo real $\phi \in \mathbb{R}$ con el que describir una partícula con el lagarangiano 
\[\mathcal{L}_{0}=\frac{1}{2}\partial_{\mu}\phi\phi \partial^{\mu}\phi-\frac{1}{2}m^{2}\phi^{2}\]

Del cual podiamos obtener el momento canónico conjugado como 

\[\Pi(x)=\pdv{\mathcal{L}}{\partial_{0}\phi}= \dot{\phi}\]

Con estas 2 cantidades podiamos escribir el Hamiltoniano del sistema como 

\[H_{0}=\int d \vec{x}\mathcal{H}=\int d \vec{x}(\Pi \phi-\mathcal{L})=\frac{1}{2}\int d \vec{x}(\Pi^{2}+(\vec{\nabla}\phi)^{2}+m^{2}\phi^{2})\]

SI ahora cuantizamos el campo pasamos a obtener que 
\[\left.\begin{array}{c}
  \phi_{i} \xrightarrow{Cuantizamos}\hat{\phi}(\vec{x},t) \\
  \Pi_{i} \xrightarrow{Cuantizamos}\hat{\Pi}(\vec{x},t)
\end{array}\right\} \comm{\hat{\phi}(\vec{x},t)}{\hat{\Pi}(\vec{x},t)}=i\delta(\vec{x}-\vec{x'})\]

En este caso a diferencia de los operadores, los campos son un continuo (por defiinición, ya que ocupan todo el espacio) no obtenemos la delta de Kronecker, si no la de Dirac. 

Otras reglas de conmutación que cumplen estos campos cuantizados son 
\[\comm{\hat{\phi}(\vec{x},t)}{\hat{\phi}(\vec{x'},t)}=\comm{\hat{\Pi}(\vec{x},t)}{\hat{\Pi}(\vec{x'},t)}\]

De esta forma obtenemos que el operador Hamiltoniano en este formalismo es 

\[\hat{H}_{0}=\int d \vec{x}\frac{1}{2}\left[\hat{\Pi}^{2}+(\vec{\nabla}\hat{\phi}(\vec{x},t))^{2}+m^{2}\hat{\phi}(\vec{x},t)^{2}\right]\]

Como los operadores de campo son continuos, estos cumplen la ecuación de heisenberg de forma que 

\[i\partial_{0}\hat{\phi}(\vec{x},t)=\comm{\hat{\phi}(\vec{x},t)}{\hat{H}_{0}}\]


Se puede demostrar que $\hat{\phi}(\vec{x},t)$ cumple la ecuación de Klein-Gordon.

\begin{proof}
  \[(\Box + m^{2})\hat{\phi}(\vec{x},t)=0\]
\end{proof}

\subsection{Solución general de la ecuación de Klein-Gordon}

Volviendo al formalismo de campos clásicos y a la ecuación de Klein-Gordon, obtuvimmos soluciones con la forma 
\[e^{-ikx}\]
para soluciones con energía positiva y con la forma
\[e^{ikx}\]

para soluciones con energía negativa, en donde $k^{\mu}=(k^{0},\vec{k})$ y $x^{\mu}=(t,\vec{x})$. Estas soluciones complian las siguientes condiciones:

\[h^{2}-m^{2}=0 \qquad k^{0}=E_{k}=\sqrt{\vec{k}^{2}+m^{2}}\text{ con }E_{k}>0\]

Si nos centramos en las soluciones con energía positiva, podemos buscar otras soluciones, que mientras cumplan las condiciones anteriores también serán soluciones válidas. 

Esta función 

\[\int d^{4}\frac{k}{(2\pi)^{3}}\alpha_{\vec{k}}\theta(k^{0})\delta(k^{2}-m^{2})e^{-ikx}\]

es solución poque $\delta(k^{2}-m^{2})$ garantiza que $k^{2}-m^{2}=0$ y $\theta(k^{0})$ (es la función escalón\todo{$\theta(k^{0})=\left\{\begin{array}{c}0 \text{ si } k^{0}<0 \\ 1 \text{ si }k^{0}>0\end{array}\right.$}) granatiza que $E_{k}>0$ y es proporcional a las soluciones que obtuvimos en el tema 1. 

Podemos definir la delta de una función como 
\begin{definition}[Delta de Dirac de una función]
  Podemos definir la delta de Dirac de una función como 
  \[\delta(f(x_{0}))=\sum_{i=1}^{\infty} \frac{\delta(x-x_{i})}{\abs{f'(x_{i})}} \]
\end{definition}

En el caso de nuestra solución esto se aplica como 
\[\delta(k^{2}-m^{2})=\delta((k^{0})^{2}-\vec{k}^{2}-m^{2})=\frac{1}{2k^{0}}\delta(k^{0}-E_{k}) + \text{Un termino de }k^{0}<0\]

Y como la última delta nos fuerza a que $k^{0}=E_{k}$. 

La última condición que debemos cumplir es que $\phi\in \mathbb{R}$, por lo que juntando este desarrollo con el mismo para las soluciones con energía negativa, tenemos que la solución más general a la ecuación de Klein-Gordon es:
\[\phi(\vec{x},t)=\int d\tilde{k}(\alpha_{\vec{k}}e^{-ikx}+\alpha^{*}_{\vec{k}}e^{ikx})\]

en donde los coeficientes $\alpha_{\vec{k}}$ y $\alpha^{*}_{\vec{k}}$ aseguran que se cumpla la última condición


\begin{corollary}
  El termino 
  \[d^{4}\frac{k}{(2\pi)^{3}}\delta(k^{2}-m^{2})\]
  se convierte en 
  \[\]
  pasando a una integral sobre el trivector $\vec{k}$ por el efecto de la delta, este término de ahora en adelante llamaremos $d \tilde{k}$. 

  Además que, tanto $d^{4}k$ como $k^{2}$ son invariantes Lorentz (esto es porque el jacobiano de la transformación es $0$), entonces 
  \[d \tilde{k}=d \tilde{k'}\]
\end{corollary}
\subsection{Soluciones en el formalismo de campo cuántico}

Para conseguir las soluciones en el formalismo de campo cuántico podemos reutilizar la misma estructura que para un campo clásico. 

Para ello seguimos teniendo que tener soluciones $\hat{\phi}(\vec{x},t)\in \mathbb{R}$, el análogo clásico era que $\phi(\vec{x},t)=\phi^{*}(\vec{x},t)$. EN este formalismo cuántico esta condición se puede obtener imponiendo que las soluciones sean autoadjuntas, es decir, $\hat{\phi}(\vec{x},t)=\hat{\phi}^{\dagger}(\vec{x},t)$.

De esta forma vemos que ya no podemos seguir usando los números $\alpha_{\vec{k}}$, si no que tenemos que cambiarlos por unos operadores llamados  aniquilación $\hat{a}_{\vec{k}}$ y destrucción $\hat{a}^{\dagger}_{\vec{k}}$.

Así obtenemos que las soluciones de la ecuación de Klein-Gordon de un campo cuántico son 

\[\hat{\phi}(\vec{x},t)=\int d \tilde{k} (\hat{a}_{\vec{k}}e^{-ikx}+\hat{a}^{\dagger}_{\vec{k}}e^{ikx})\]

\subsubsection{Resumen de los operadores de campo $\hat{\phi}(\vec{x},t)$}

\begin{itemize}
  \item Estos operadores satisfacen la ecuación de Heisenberg 
  \[i\partial_{0}\hat{\phi}(\vec{x},t)=\comm{\hat{\phi}(\vec{x},t)}{\hat{H}_{0}}\]
  \item Tambien Satisfacen la ecuación de Klein-Gordon
  \[(\Box +m^{2})\hat{\phi}(\vec{x},t)=0\]
  \item Son autoadjuntos 
  \[\hat{\phi}(\vec{x},t)=\hat{\phi}^{\dagger}(\vec{x},t)\]
  \item Cumplen la regla de conmutación
  \[\comm{\hat{\phi}(\vec{x},t)}{\dot{\hat{\phi}}(\vec{x},t)}=i\delta(\vec{x}-\vec{x'})\]
  
  Esto no lo hemos obtenido realmente, pero si forzamos a que cumplan estas reglas podemos obtener las reglas de conmutación de los operadores creación y destrucción. Estas reglas son
  \[\comm{\hat{a}_{\vec{k}}}{\hat{a}_{\vec{k'}}}=\comm{\hat{a}^{\dagger}_{\vec{k}}}{\hat{a}^{\dagger}_{\vec{k'}}}=0\]
  y \[\comm{\hat{a}_{\vec{k}}}{\hat{a}^{\dagger}_{\vec{k}}}= 2E_{k}(2\pi)^{3}\delta(\vec{k}-\vec{k'})\]
\end{itemize}

Una vez encontrados los operadores, nos falta encontrar a que estados los aplicamos, ya dijimos que no ibamos a tener funciones de onda, ¿como vamos a describir a las cosas si no usamos funciones de onda?

\subsection{Espacio de Fock}

El espacio de Fock es el equivalente al espacio de Hilbert en campos cuánticos. 

\begin{definition}[Espacio de Fock] 
\\
  El espacio de Fock ${\mathcal {F}}(H)$, en mecánica cuántica es un espacio de Hilbert especial, que se construye como suma directa de productos tensoriales de otro espacio de Hilbert dado 
$H$. Dicho espacio se usa para describir el estado cuántico de un sistema formado por un número variable o indeterminado de partículas.
  
\end{definition}

Dentro de este espacio vamos a encontrar los estados de Fock, que son los que usaremos en este formalismo de campo cuantizado. 

\begin{definition}[Estados de Fock]
  Si nos limitamos, por simplicidad, a un sistema con un solo tipo de partícula y un solo modo (con lo que formalmente estamos describiendo un oscilador armónico), un estado de Fock se representa por $\ket{n}$, donde $n$ es un valor entero. Esto significa que hay $n$ cuantos de excitación en ese modo. Así, $\ket{0}$ corresponde al estado fundamental (sin excitación), o estado que representa el vacío cuántico (esto es diferente de 0 ($\braket{0}=1$), que es el vector nulo que no es un estado posible del sistema al no ser un vector unitario)
\end{definition}

Los estados de Fock han de cumplir las siguientes condiciones:
\begin{itemize}
  \item \[\hat{a}_{\vec{k}}\ket{0}=0 \qquad \forall \vec{k}\]
  Si recordamos al operador $\hat{a}_{\vec{k}}$ lo denominamos operador destrucción, por lo que al vacío no podemos quitarle partículas
  \item \[\hat{a}^{\dagger}_{\vec{k}}\ket{0}=\ket{\vec{k}}\]
  Si recordamos al operador $\hat{a}^{\dagger}_{\vec{k}}$ lo denominamos operador creación, si al vacío le aplicamos este operador aparecerá una partícula.

  De esta forma podemos ver que cualquier estado puede ser construido aplicando el operador creación al estado de vacío $\ket{0}$. 
\end{itemize}

Ahora vamos a contruir la normalización de este conjunto de estados, esto es:

\[\braket{\vec{k}}{\vec{k'}}= \ev{\hat{a}_{\vec{k}}\hat{a}^{\dagger}_{\vec{k'}}}{0}=\ev{\comm{\hat{a}_{\vec{k}}}{\hat{a}_{\vec{k'}}}}{0}+\cancelto{0}{\ev{\hat{a}^{\dagger}_{\vec{k'}}\hat{a}_{\vec{k'}}}{0}}=2E_{k}(2\pi)^{3} \var(\vec{k}-\vec{k'})\]

\begin{remark}[Normalización de estados con momento bien definidio]
Como podemos observar la norma de los estados con $\vec{k}=\vec{k'}$ nos da 0, sin embargo podemos redefinir la delta como: 

\[\var(\vec{k})=\frac{1}{(2\pi)^{3}}\int_{\mathbb{R}^{3}}d \vec{x}e^{ikx}\chi(x)\]

si $\chi(x)=0\qquad\forall x\in\Omega$, es decir, que la función es 0 para cualquier valor de $x$ que no pertenezca a una región del espacio, entonces 

\[\var(0)=\frac{1}{(2\pi)^{3}}V(\Omega)\] 

si el espacio $\Omega \equiv \mathbb{R}^{3}$ entonces $V(\Omega)=V$

Cuando los momentos están bien definidos $\var(\vec{k}-\vec{k'})=\var(0)$ y podemos hacer la normalización así:

\[\braket{\vec{k}}=2E_{k}V\]

\end{remark}

Una vez obtenida la normalización vamos a construir estados de multiples partículas, empezando por 2:
\[\hat{a}^{\dagger}_{\vec{k}_{1}}\hat{a}^{\dagger}_{\vec{k}_{2}}\ket{0}=\ket{\vec{k_{1}}, \vec{k_{2}}}\]

Esto quiere decir que para crear un estado vamos a multiplicar tantos operadores creación al estado de vacío como partículas haya en el. De esta forma un estado con $N$ partículas es 

\[\ket{\vec{k_{1}}, \vec{k_{2}}, \cdots, \vec{k_{N}}}=\frac{1}{\sqrt{n!}} \hat{a}^{\dagger}_{\vec{k}_{1}}\hat{a}^{\dagger}_{\vec{k}_{2}}\cdots \hat{a}^{\dagger}_{\vec{k}_{N}}\ket{0}\]

Tambien podemos tener un estado cualquiera $\ket{\psi}\in \mathcal{F}$ formado por muchos estados diferentes con cada uno diferentes partículas como 

\[\sum_{n=0}^{\infty}\frac{1}{\sqrt{n!}} f(\vec{k}_{1}, \vec{k}_{2},\cdots \vec{k}_{n}) \prod_{n=0}^{\infty}\hat{a}^{\dagger}_{\vec{k}_{n}} \ket{0}\]

Obteniendo un estado con la forma 

\[\ket{\phi}= \ket{0}+ \alpha\ket{\vec{k}_{1}}+\beta\ket{\vec{k}_{1} \vec{k}_{2}} + \cdots\]

además, como 

\[\comm{\hat{a}_{\vec{k}}}{\hat{a}^{\dagger}_{\vec{k'}}}=0 \longrightarrow \ket{\vec{k_{1}}, \vec{k_{2}}}=\ket{\vec{k_{2}}, \vec{k_{2}}}\]

Es decir, son funciones simétricas y por lo tanto serán lo que llamaremos bosones.

\subsubsection{Operadores en el espacio de Fock}
Una vez definidos los estados, lo lógico es calcular algunos valores esperados necesarios y ver como se hace. Para ellos calculamos el valor esperado de la energía de vacio como 
\paragraph{Operador Energía}
\begin{DispWithArrows}[format=c, displaystyle]
\ev{\hat{H}_{0}}{0}=\frac{1}{2} \int d \tilde{k} \braket{\vec{k}}= \frac{V}{2}\int d \tilde{k}= \infty
\end{DispWithArrows}

Como vemos la energía de vacio diverge, lo cual no tiene sentido por lo que podemos definir un nuevo Hamiltoniano como 

\begin{DispWithArrows}[format=c, displaystyle]
\hat{H'}_{0}=\hat{H}_{0} - \ev{\hat{H}_{0}}{0} \\
\ev{\hat{H}_{0}}{0}=0
\end{DispWithArrows}

De esta forma la energía de vacio se anula, lo cual tiene más sentido. 

Como ya mencionamos al pasar de las coordenadas clasicas $q,p$ a los operadores $\hat{q}, \hat{p}$ teniamos el problema de que estos operadores no conmutan, por lo que hay muchas formas de obtener $\hat{H}_{0}$, para no tener esa ambigüedad a la hora de obtener el Hamiltoniano definimos el producto normal 

\begin{definition}[Producto normal]
  El producto normal se define como 

  \begin{DispWithArrows}[format=c, displaystyle]
  :\hat{a}_{\vec{k}}\hat{a}^{\dagger}_{\vec{k}}:=\hat{a}^{\dagger}_{\vec{k}}\hat{a}_{\vec{k}}
  \end{DispWithArrows}

  Basicamente, lo que hace es ordenar los operadores de forma que todos los conjugados se encuentren a la izquierda.
\end{definition}

De esta forma podemos definir el Hamiltoniano $\hat{H'}$ como 

\begin{DispWithArrows}[format=c, displaystyle]
  \hat{H'} \equiv :\hat{H}_{0}:=\int d \tilde{k}E_{k}\hat{a}^{\dagger}_{\vec{k}}\hat{a}_{\vec{k}}
\end{DispWithArrows}

En donde encontramos un nuevo operador como multiplicación de 2 operadores, este es el operador número y se define como 

\begin{definition}[Operador número]
  El operador número de una partícula se define como:
  \begin{DispWithArrows}[format=c, displaystyle]
  \hat{n}_{k}=\hat{a}^{\dagger}_{\vec{k}}\hat{a}_{\vec{k}}
  \end{DispWithArrows}

  Este operador nos devuelve el número de partículas en un estado.
\end{definition}

De esta forma el Hamiltoniano $H'$ es 

\begin{DispWithArrows}[format=c, displaystyle]
\int d \tilde{k}E_{k} \hat{n}_{k}
\end{DispWithArrows}

Una vez definido el operadore número, es conveniente definir el operador número total de partículas 

\begin{definition}[Operador número total de partículas]
  El operador número total de partículas se define como: 

  \begin{DispWithArrows}[format=c, displaystyle]
  \hat{N}=\int d\tilde{k}\hat{n}_{k}
  \end{DispWithArrows}

  Este operador nos devuelve el número total de partículas en un estado. 
  \begin{DispWithArrows}[format=ccc, displaystyle]
  \hat{N}\ket{0}=0 & \hat{N}\ket{\vec{k}}=1\ket{\vec{k}} & \hat{N}\ket{\vec{k}_{1}\vec{k}_{2}} =2\ket{\vec{k}_{1}\vec{k}_{2}}
  \end{DispWithArrows}

  Hay que tener en cuenta que en el caso de $\ket{\psi}=\alpha \ket{\vec{k}}+\beta \ket{\vec{k'}\vec{k''}}$ no tenemos un autoestado, por lo que el número total de partículas no estará bien definido y podrá dar números que no son enteros.
\end{definition}

Dado que la construcción del producto normal nos pone el operador aniquilación en primer lugar obtenemos que el valor esperado de la energía de vacío es
\begin{DispWithArrows}[format=c, displaystyle]
\ev{\hat{H'}}{0}=0
\end{DispWithArrows}

Otros resultados de este Hamiltoniano son:
\begin{itemize}
  \item $\hat{H'}\ket{0}$
  \item $\hat{H'}\ket{\vec{k}}=E_{k}\ket{\vec{k}}$
  \item $\hat{H}\ket{\vec{k}_{1}\vec{k}_{2}}=(E_{k_{1}}+E_{k_{2}})\ket{\vec{k}_{1}\vec{k}_{2}}$
\end{itemize}
\paragraph{Operador momento lineal }
Tambien se puede definir el operador momento lineal mediante el producto normal como 

\begin{DispWithArrows}[format=c, displaystyle]
:\hat{P}:=\int d\tilde{k}\hat{n}_{k}\vec{k}
\end{DispWithArrows}
\paragraph{Operador campo}

El operador campo conjugado crea una partícula y nos devuelve la posición de esa partícula en el campo, es decir,

\begin{DispWithArrows}[format=c, displaystyle]
\hat{\Phi}^\dagger(\vec{x},0)\ket{0}=\ket{\psi}=\ket{\vec{x}}
\end{DispWithArrows}
\begin{proof}
  Esto se puede ver como 
  \begin{DispWithArrows}[format=c, displaystyle]
  \braket{\vec{k}}{\psi}=e^{-ikx} \\
  \braket{\vec{k}}{\vec{x}}=e^{-ikx} \\
  \end{DispWithArrows}

  Y como ambas cosas son iguales, podemos sustituir.
\end{proof}

\subsection{Propagador del campo escalar libre}
El propagador de un campo se definer como:

\begin{definition}

  Definimos el propagador como: 
  \begin{DispWithArrows}[format=c, displaystyle]
  G(x,y)=\ev{T(\hat{\phi}(x),\hat{\phi}(y))}{0}
  \end{DispWithArrows}
  Donde $\hat{\phi}(x)$ es el campo escalar libre en el punto $x$ y $T(\hat{\phi}(x),\hat{\phi}(y))$ es el operador de ordenación temporal, el cual ordena los tiempos de mayor a menor, es decir, 
  \[\begin{array}{ccc}
    \hat{\phi}(x),\hat{\phi}(y) &\text{si} & x^{0}>y^{0}\\
    \hat{\phi}(y),\hat{\phi}(x) &\text{si} & x^{0}<y^{0}\\
  \end{array}\]
\end{definition}
Ahora podemos ver que tenemos 2 casos:
\begin{enumerate}
  \item $x^{0} > y^{0}$
  
  En este caso tenemos que el propagador es 

  \begin{DispWithArrows}[format=ll, displaystyle]
  G(x,y)= & \ev{\hat{\phi}(x),\hat{\phi}(y)}{0} \\
  =&\int d\tilde{k}e^{-ikx}\int d \tilde{k'}e^{ik'y}\braket{\vec{k}}{\vec{k'}} \\
  =&\int d\tilde{k}e^{-ik(x-y)}
  \end{DispWithArrows}

  \item $x^{0} < y^{0}$
  
  De forma análoga tenemos que el propagador es 

  \begin{DispWithArrows}[format=ll, displaystyle]
  G(x,y)= & \ev{\hat{\phi}(y),\hat{\phi}(x)}{0}\\
  =& \int d\tilde{k}e^{-iky}\int d \tilde{k'}e^{ik'x}\braket{\vec{k}}{\vec{k'}} \\
  &=\int d\tilde{k}e^{-ik(y-x)}
  \end{DispWithArrows}
\end{enumerate}

El progpagador del campo escalar libre es una función que realmente se comporta como una distribución de probabilidad ya que permite el uso de deltas de Dirac. 

Podemos definir el propagador de forma general como 

\begin{definition}[Propagador General I]
  Podemos definir la versión más general del propagador como:

  \begin{DispWithArrows}[format=c, displaystyle]
  G(x,y)=\theta(x^{0}-y^{0})\int d\tilde{k}e^{-ik(x-y)}+ \theta(y^{0}-x^{0})\int d\tilde{k}e^{-ik(y-x)}
  \end{DispWithArrows}

  
\end{definition}

Aunque esta difinición no es muy comoda, podemos definirla de forma más sencilla haciendo uso del siguiente teorema: 

  \begin{theorem}
    ~~

    \begin{DispWithArrows}[format=c, displaystyle]
    G(x,y)=\lim_{\epsilon\rightarrow 0}i\int \frac{d^{4}k}{(2\pi)^{4}}\frac{e^{-ik(x-y)}}{k^{2}-m^{2}+i\epsilon}
    \end{DispWithArrows}
    Primero se hace la integral y luego se hace el límite
  \end{theorem}

\begin{proof}
  A continuación demostraremos este teorema para el caso $x^{0} > y^{0}$ y para el caso $x^{0} < y^{0}$ se procederá de manera análoga.


\end{proof}


\section{Cuantización del campo escalar complejo}

En esta sección vamos a dar la forma del operador campo complejo, de la misma forma que el campo escalar libre.


A continuación vamos a definir el campo escalar complejo como $\phi \in \mathbb{C}$ dado que el campo es complejo, no tenemos la misma condición de mantenerlo real, por lo que $\phi \neq \phi^{*}$, es decir, el conjugado del no tiene porque ser igual que el campo. 

También definimos el módulo al cuadrado del campo complejo como $\abs{\phi}^{2}=\phi\phi^{*}$. 

De esta forma el  lagrangiano del campo complejo es \marginnote[-2.2cm]{
  \begin{kaobox}[frametitle=Recuerda]
    Cuando se habla de Lagrangiano O hamiltoniano pero se usa $\mathcal{L}$ o $\mathcal{H}$, estamos hablando de densidades Lagrangianas y Hamiltonianas respectivamente. El lagrangiano o hamiltoniano será $L=\int d\vec{x}\mathcal{L}$ y $H=\int d\vec{x}\mathcal{H}$
\end{kaobox}
}
\begin{DispWithArrows}[format=c, displaystyle]
\mathcal{L}_0=\partial_{\mu}\phi\partial^{\mu}\phi - m^{2}\abs{\phi}^{2}
\end{DispWithArrows}

Este campo complejo también obedece las ecuaciones de Klein-Gordon, es decir, 

\begin{DispWithArrows}[format=c, displaystyle]
(\Box +m^{2})\phi=0 \\
(\Box +m^{2})\phi^{*}=0
\end{DispWithArrows}

Una vez obtenido el lagrangiano podemos obtener el Hamiltoniano, pero antes es necesario definir el momento canónico conjugado, el cual es 

\begin{DispWithArrows}[format=c, displaystyle]
\Pi=\pdv{\mathcal{L}}{\partial_{0}\phi}\equiv \dot{\phi}^{*} \Arrow{Y su complejo} \\
\Pi^{*}=\pdv{\mathcal{L}}{\partial_{0}\phi^{*}}\equiv \dot{\phi}
\end{DispWithArrows}

Y ahora si, obtenemos el Hamiltoniano como 

\begin{DispWithArrows}[format=c, displaystyle]
\mathcal{H}_{0}=\Pi\Pi^{*}+ \vec{\nabla}\cdot \phi\vec{\nabla}\cdot\phi^{*} + m^{2}\abs{\phi}^{2}
\end{DispWithArrows}



Ahora pasamos a sistituir el campo complejo por los operadores de campo complejo, es decir, 

\begin{DispWithArrows}[format=c, displaystyle]
\phi \longrightarrow \hat{\phi} \\
\phi^{*} \longrightarrow \hat{\phi}^{\dagger}
\end{DispWithArrows}

Los conmutadores no triviales de estos operadores, es decir, que son diferentes de 0 son:

\begin{DispWithArrows}[format=c, displaystyle]
\comm{\hat{\phi}(\vec{x},t)}{\dot{\hat{\phi}}^{\dagger}(\vec{x},t)}=i\delta(\vec{x}-\vec{x}') \\
\comm{\hat{\phi}(\vec{x},t)}{\dot{\Pi}(\vec{x},t)}=i\delta(\vec{x}-\vec{x}')

\end{DispWithArrows}

Según la ecuación de evolución temoral de Heisenberg la evolución temporal del operador campo complejo es 
\begin{DispWithArrows}[format=c, displaystyle]
i\partial_{0}\hat{\phi}(\vec{x},t)=\comm{\hat{\phi}(\vec{x},t)}{H_{0}}
\end{DispWithArrows}

Tambien sabemos que cumplen la ecuación de Klein-Gordon al igual que hicimos con el campo escalar libre, es decir,

\begin{DispWithArrows}[format=c, displaystyle]
  (\Box +m^{2})\hat{\phi}=0 \\
  (\Box +m^{2})\hat{\phi}^{\dagger}=0
  \end{DispWithArrows}

  A continuación vamos a dar la forma del operador campo complejo, de la misma forma que el campo escalar libre, pero ahora, como no tenemos la restricción de que el campo sea real ($\phi \neq \phi^{*}$) entonces tenemos 

  \begin{DispWithArrows}[format=c, displaystyle]
  \hat{\phi}(\vec{x},t)=\int d\tilde{k}(\hat{a}_{\vec{k}}e^{-ikx}+\hat{b}^{\dagger}_{\vec{k}}e^{ikx}) \\
  \hat{\phi}^{\dagger}(\vec{x},t)=\int d\tilde{k}(\hat{a}^{\dagger}_{\vec{k}}e^{ikx}+\hat{b}_{\vec{k}}e^{-ikx})
  \end{DispWithArrows}

\subsubsection{Estados de Fock y sus operadores en el campo complejo}
  Al igual que en el caso del campo escalar libre, tenemos los estados del espacio de Fock y sus operadores. Ahora partimos del mismo estado de Fock que en el caso del campo escalar libre, cuyas condiciones eran:

  \begin{DispWithArrows}[format=c, displaystyle]
  \braket{0}=1 \\
  \hat{a}_{\vec{k}}\ket{0}=0 \Arrow{Añadimos una más}\\
  \hat{b}_{\vec{k}}\ket{0}=0 
  \end{DispWithArrows}

  Esta última condición es nueva, ya que el campo complejo no tiene la condición de mantenerse real. 

  Los nuevos operadores $\hat{b}^{\dagger}_{\vec{k}}$ y $\hat{b}_{\vec{k}}$ funcionan de manera similar a los operadores $\hat{a}^{\dagger}_{\vec{k}}$ y $\hat{a}_{\vec{k}}$ del campo escalar libre. Solo que estos operadores serán los \textbf{Operadores creación y aniquilación de antipartículas}, respectivamente.

  Entonces los estados de Fock para el campo complejo son los mismos que para el campo escalar libre, pero ahora con el operador de creación y aniquilación de antipartículas, es decir:

  \begin{DispWithArrows}[format=c, displaystyle]
    \hat{a}^{\dagger}_{\vec{k}}\ket{0}=\ket{\vec{k}}\Arrow{Pero además} \\
    \hat{b}^{\dagger}_{\vec{k}}\ket{0}=\ket{\underline{\vec{k}}} \\
    \hat{a}^{\dagger}_{\vec{k}_{1}}\hat{a}^{\dagger}_{\vec{k}_{2}}\ket{0}=\ket{\vec{k_{1}}, \vec{k_{2}}}\Arrow{Pero además} \\
    \hat{b}^{\dagger}_{\vec{k}_{1}}\hat{b}^{\dagger}_{\vec{k}_{2}}\ket{0}=\ket{\underline{\vec{k_{1}}}, \underline{\vec{k_{1}}}}
  \end{DispWithArrows}

  Y esto con todo lo que hemos visto hasta ahora, basicamente añadir las mismas operaciones del campo escalar libre con los nuevos operadores de creación y aniquilación de antipartículas y poniendo una barra debajo para diferenciar los estados. 

  Otra cosa que podemos obtener en este campo son estados mixtos que tienen la forma 

  \begin{DispWithArrows}[format=c, displaystyle]
    \hat{a}^{\dagger}_{\vec{k}_{1}}\hat{b}^{\dagger}_{\vec{k}_{2}}\ket{0}=\ket{\vec{k_{1}}, \underline{\vec{k_{1}}}}
  \end{DispWithArrows}

  Esto es debido a que las reglas de conmutación no triviales son 

  \begin{DispWithArrows}[format=c, displaystyle]
  \comm{\hat{a}_{\vec{k}}}{\hat{a}^{\dagger}_{\vec{k}}}=(2\pi)^{3}E_k \var(\vec{k}-\vec{k'})\\
  \comm{\hat{b}_{\vec{k}}}{\hat{b}^{\dagger}_{\vec{k}}}=(2\pi)^{3}E_k \var(\vec{k}-\vec{k'})\\
  \comm{\hat{a}_{\vec{k}}}{\hat{b}^{\dagger}_{\vec{k}}}=\comm{\hat{b}_{\vec{k}}}{\hat{a}^{\dagger}_{\vec{k}}}=\comm{\hat{a}_{\vec{k}}}{\hat{b}_{\vec{k}}}=\comm{\hat{a}^{\dagger}_{\vec{k}}}{\hat{b}^{\dagger}_{\vec{k}}}=0

  \end{DispWithArrows}

  \paragraph{Operador Energía}

  Una vez definidos los operadores y los estado de Fock, podemos definir el Hamiltoniano de forma aáloga al campo escalar libre, pero ahora con el operador de creación y aniquilación de antipartículas.

  \begin{DispWithArrows}[format=c, displaystyle]
    \hat{H'} \equiv :\hat{H}_{0}:=\int d \tilde{k}E_{k}(\hat{a}^{\dagger}_{\vec{k}}\hat{a}_{\vec{k}}+\hat{b}^{\dagger}_{\vec{k}}\hat{b}_{\vec{k}})=\int d \tilde{k}E_{k}(\hat{n}_k+\hat{\overline{n}}_k)
  \end{DispWithArrows}

  Al aplicarlo al estado vacio obtenemos 0, ya que gracias a usar el producto normal no hay ni energías negativas ni divergencias. Aplicando este operador sobre un estado cualquiera obtenemos 

  \begin{DispWithArrows}[format=c, displaystyle]
  \left.\begin{array}{c}
  \hat{H'}\ket{\vec{k}}=E_{k}\ket{\vec{k}} \\
  \hat{H'}\ket{\underline{\vec{k}}}=E_{k}\ket{\underline{\vec{k}}} 
  \end{array}\right\} \forall \ket{\psi} \in \mathscr{F} \ev{\hat{H'}}{\psi}=0
  \end{DispWithArrows}

  En donde $\hat{\overline{n}}_k$ es el operador número de antipartícula, el operador número total de antipartícuales es

  \begin{DispWithArrows}[format=c, displaystyle]
  \hat{\overline{N}}=\int d \tilde{k}\hat{\overline{n}}_k
  \end{DispWithArrows}
\paragraph{Operador Carga}
  Un operador que no nos intereso antes, ya que en el campo escalar libre todas las partículas tenian la misma carga, es el operador carga, el cual devuelve la carga de un estado 

  \begin{DispWithArrows}[format=c, displaystyle]
  \hat{Q}=q\int d\tilde{k}(\hat{n}_k-\hat{\overline{n}}_k)
  \end{DispWithArrows}

  La carga de los sistemas se conserva, es decir,

  \begin{DispWithArrows}[format=c, displaystyle]
  \comm{\hat{H'}}{\hat{Q}}=0
  \end{DispWithArrows}

  Y podemos obtener la carga de una partícula en un estado cualquiera de la forma

  \begin{DispWithArrows}[format=c, displaystyle]
    \hat{Q}\hat{a}^\dagger _{k\vec{k}}\ket{0}=q\ket{\vec{k}} \\
    \hat{Q}\hat{b}^\dagger _{k\vec{k}}\ket{0}=-q\ket{\vec{k}} 
  \end{DispWithArrows}

  Las funciones de onda en este caso siguen siendo simétricas por lo que seguimos en un campo bosónico.
  \section{¿Porque estos campos?}

  Como vimos en el primer tema la ecuación de Klein-Gordon no describe bien la mecánica cuántica de las partículas haciendo uso de las funciones de onda, pero si que es una buena ecuación para describir las partículas si se usan campos.

  Además, La ventaja de usar campos, es que con 1 solo campo definimos todas las partícuals que existen en el universon, en el caso de los campos escalares podemos definir:

  \begin{itemize}
    \item Bosones sin antipartículas con espín 0
    
    Como fotónes, Bosones de Higgs, etc.
    \item Bosones con antipartículas y con espín 0
    
    Como piones y muchos otros. (En estructura seguro que los visteis, eran tantos que no recuerdo ninguno)
  \end{itemize}