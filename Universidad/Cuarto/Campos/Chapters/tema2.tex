\setchapterpreamble[u]{\margintoc}
\chapter{Teoría clásica de campos}
\labch{Part}

\begin{center}
  \large Todo esto lo he sacado del \cite{Dobdado}
\end{center}
\section{Campos clásicos relativistas}

En la teoría de campos clásicos relativistas, el campo es un concepto abstracto que se puede entender como una función de varias variables que depende de las coordenadas espaciales y de un parámetro temporal.

Al igual que las funciones de onda, tenemos un campo $\Phi$ que de pende de las $\phi_{i}$ coordenadas, y cada una de estas coordenadas esta definida por unas coordenadas espaciales $\vec{x}$ y un parámetro temporal $t$.

A la hora de aplicar transformaciones en este formalismo podemos emplear 2 tipos de puntos de vista:
\begin{enumerate}
  \item \textbf{Actitud activa}: Esta será la que usaremos y será aquella en la que las transformaciones se hacen al sistema 
  \item \textbf{Actitud pasiva}: En este caso aplicamos las transformaciones al sistema de referencia.
\end{enumerate}
\begin{corollary}
  Cuando hay simetria estos puntos de vista pueden ser iguales.
\end{corollary}

Dado que adpotamos la actitud activa, una transformación sobre el campo lo alterará de forma que 

\[\Phi(\vec{x},t)\xrightarrow{\text{T}} \Phi(\vec{x'})\]

De esta forma definimos las variaciones de campo como $\tilde{\delta}$:

\begin{definition}[Variación tilda]
  La variación tilda $\tilde{\delta}$ de un campo $\Phi$ es la diferencia entre el campo en un punto $\vec{x}$ y el campo en un punto $\vec{x'}$ que se obtiene al aplicar una transformación $\text{T}$ al campo original. Es decir,
  
  \[\tilde{\delta}\Phi=\underbrace{\Phi'(\vec{x'})-\Phi(\vec{x})}_{\text{Trans. General}}=\Phi'(\vec{x'}+\delta\vec{x})-\Phi(\vec{x})=\]
  \[\Phi'(x^{\mu})+\partial_{\mu}\Phi'(x^{\mu})(\delta x^{\mu})-\Phi(x^{\mu})\]

  En donde el termino del medio se va ya que no quedaría lineal ($\Phi'(x^{\mu})=\Phi(x^{\mu})+\delta \Phi$). 

  De esta forma, la variación tilda de un campo $\Phi$ es:
  \[\tilde{\delta}\Phi=\underbrace{\Phi'(\vec{x})-\Phi(\vec{x})}_{\text{Variación \\ intrínseca }}+\underbrace{\delta x^{\mu}\partial_{\mu}\Phi(x^{\mu})}_{\text{Termino de \\ transporte}}=\delta\Phi+\delta x^{\mu}\partial_{\mu}\Phi(x^{\mu})\]
  
\end{definition}

\subsection{Casos simples}
Los casos simples son aquellos en los que las variaciones del campo tienen o no tiene término de transporte, es decir, 
\[\delta x^{\mu}\neq 0 \longrightarrow\text{Transformación tipo espacio-tiempo}\]
\[\delta x^{\mu}= 0 \longrightarrow\text{Transformación tipo interna} \rightarrow \tilde{\delta}\Phi=\delta\Phi\]
A continuación definiremos que son estas variaciones:
\begin{definition}[Variación intrínseca]
  La variación 
  $\delta x^\mu=\epsilon^a A_a^\mu(x) \quad \epsilon^a \ll 1 \quad \partial_\mu \epsilon^a=0 \rightarrow$ transformación global

\end{definition}
\begin{definition}[Variación tilda]
  $\hat{\delta} \phi=\epsilon^a F_{i, a}(\Phi, \partial \Phi) \quad \quad \partial_\mu \epsilon^a \neq 0 \rightarrow$ transformaciór local (Gauge)
\end{definition}
\subsubsection{Transformaciones de Lorentz}

\begin{example}[Campo escalar]
  
\end{example}
\begin{example}[Espinores de Weyl]
  
\end{example}
\begin{example}[Campo vectorial]
  
\end{example}
\begin{example}[Campo tensorial]
  
\end{example}
\section{Dinámica de los campos clásicos}

Por la mecánica lagrangiana sabemos que para estudiar la dinámica usamos las llamadas coordenadas generalizadas $q_{i}(t)$. En el formalismo de campos utilizamos el propio campo para definir la dinámica, es decir, supongamos que tenemos un campo $\Phi$ el cual esta compuesto por el valor del campo en un punto que denotamos $\phi_{i}(\vec{x},t)$ estas $\phi_{i}$ son las coordenadas generalizadas en el formalismo de campos. Esto se puede ver como:

\[q_{i}\to\phi_{i}(\vec{x},t)\]

Esto quiere decir, que vamos a tener tantas coordenadas como puntos del espacio esten definidos en nuestro campo y cada punto del espacio $\phi_{i}$ esta definido por unas coordeandas. 

De esta forma, la teoría de campos que buscamos para entender la dinámica de un sistema es una \textit{Teoría de campos local}, es decir, que lo que le ocurre a un punto del campo solo afecte a otros puntos cercanos y no a todo el campo y que además sea relativista.
\subsection{Formalismo Lagrangiano}
Para ello definimos la densidad lagrangiana $\mathcal{L}$ la cual depende de las coordenadas generalizadas $\phi_{i}$ y de sus velocidades generalizadas $\partial_{\mu}\phi_{i}$
\subsection{Formalismo Hamiltoniano}

Una vez descrito el formalismo Lagrangiano, podemos ir al formalismo Hamiltoniano de forma análoga a como lo haciamos en mecánica clásica, para ello definimos el \textit{Momento canónico generalizado} como 

\[\Pi(x)=\pdv{\mathcal{L}}{\partial(\partial_{0}\phi_{i})}\text{ con }\partial_{0}\phi_{i}= \dot{\phi}\]

De forma que el Hamiltoniano queda definido como 

\[\mathcal{H}=\Pi\phi_{i}-\mathcal{L}\]

\subsection{Ejemplos}

\begin{example}[Campo escalar real]
  
\end{example}
\begin{example}[Campo escalar real]
  
\end{example}
\begin{example}[Campo escalar real]
  
\end{example}
\begin{example}[Campo escalar real]
  
\end{example}
\begin{example}[Campo escalar real]
  
\end{example}
\section{Teorema de Noether}
\section{Transformaciones de simetria espacio-temporales}
\subsection{Traslaciones}
\subsection{Transformaciones de Lorentz}