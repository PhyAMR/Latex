\setchapterpreamble[u]{\margintoc}
\chapter{Teoría clásica de campos}
\labch{Part}

\begin{center}
  \large Todo esto lo he sacado del \cite{Dobdado}
\end{center}
\section{Campos clásicos relativistas}

En la teoría de campos clásicos relativistas, el campo es un concepto abstracto que se puede entender como una función de varias variables que depende de las coordenadas espaciales y de un parámetro temporal.

Al igual que las funciones de onda, tenemos un campo $\Phi$ que de pende de las $\phi_{i}$ coordenadas, y cada una de estas coordenadas esta definida por unas coordenadas espaciales $\vec{x}$ y un parámetro temporal $t$.

A la hora de aplicar transformaciones en este formalismo podemos emplear 2 tipos de puntos de vista:
\begin{enumerate}
  \item \textbf{Actitud activa}: Esta será la que usaremos y será aquella en la que las transformaciones se hacen al sistema 
  \item \textbf{Actitud pasiva}: En este caso aplicamos las transformaciones al sistema de referencia.
\end{enumerate}
\begin{corollary}
  Cuando hay simetria estos puntos de vista pueden ser iguales.
\end{corollary}

Dado que adpotamos la actitud activa, una transformación sobre el campo lo alterará de forma que 

\[\Phi(\vec{x},t)\xrightarrow{\text{T}} \Phi(\vec{x'})\]

De esta forma definimos las variaciones de campo como $\tilde{\delta}$:

\begin{definition}[Variación tilda]
  La variación tilda $\tilde{\delta}$ de un campo $\Phi$ es la diferencia entre el campo en un punto $\vec{x}$ y el campo en un punto $\vec{x'}$ que se obtiene al aplicar una transformación $\text{T}$ al campo original. Es decir,
  
  \[\tilde{\delta}\Phi=\underbrace{\Phi'(\vec{x'})-\Phi(\vec{x})}_{\text{Trans. General}}=\Phi'(\vec{x'}+\delta\vec{x})-\Phi(\vec{x})=\]
  \[\Phi'(x^{\mu})+\partial_{\mu}\Phi'(x^{\mu})(\delta x^{\mu})-\Phi(x^{\mu})\]

  En donde el termino del medio se va ya que no quedaría lineal ($\Phi'(x^{\mu})=\Phi(x^{\mu})+\delta \Phi$). 

  De esta forma, la variación tilda de un campo $\Phi$ es:
  \[\tilde{\delta}\Phi=\underbrace{\Phi'(\vec{x})-\Phi(\vec{x})}_{\text{Variación \\ intrínseca }}+\underbrace{\delta x^{\mu}\partial_{\mu}\Phi(x^{\mu})}_{\text{Termino de \\ transporte}}=\delta\Phi+\delta x^{\mu}\partial_{\mu}\Phi(x^{\mu})\]
  
\end{definition}

\subsection{Casos simples}
Los casos simples son aquellos en los que las variaciones del campo tienen o no tiene término de transporte, es decir, 
\[\delta x^{\mu}\neq 0 \longrightarrow\text{Transformación tipo espacio-tiempo}\]
\[\delta x^{\mu}= 0 \longrightarrow\text{Transformación tipo interna} \rightarrow \tilde{\delta}\Phi=\delta\Phi\]
A continuación definiremos que son estas variaciones:
\begin{definition}[Variación intrínseca]
  La variación 
  $\delta x^\mu=\epsilon^a A_a^\mu(x) \quad \epsilon^a \ll 1 \quad \partial_\mu \epsilon^a=0 \rightarrow$ transformación global

\end{definition}
\begin{definition}[Variación tilda]
  $\hat{\delta} \phi=\epsilon^a F_{i, a}(\Phi, \partial \Phi) \quad \quad \partial_\mu \epsilon^a \neq 0 \rightarrow$ transformaciór local (Gauge)
\end{definition}
\subsubsection{Transformaciones de Lorentz}

\begin{example}[Campo escalar]
  
\end{example}
\begin{example}[Espinores de Weyl]
  
\end{example}
\begin{example}[Campo vectorial]
  
\end{example}
\begin{example}[Campo tensorial]
  
\end{example}
\section{Dinámica de los campos clásicos}

Por la mecánica lagrangiana sabemos que para estudiar la dinámica usamos las llamadas coordenadas generalizadas $q_{i}(t)$. En el formalismo de campos utilizamos el propio campo para definir la dinámica, es decir, supongamos que tenemos un campo $\Phi$ el cual esta compuesto por el valor del campo en un punto que denotamos $\phi_{i}(\vec{x},t)$ estas $\phi_{i}$ son las coordenadas generalizadas en el formalismo de campos. Esto se puede ver como:

\[q_{i}\to\phi_{i}(\vec{x},t)\]

Esto quiere decir, que vamos a tener tantas coordenadas como puntos del espacio esten definidos en nuestro campo y cada punto del espacio $\phi_{i}$ esta definido por unas coordeandas. 

De esta forma, la teoría de campos que buscamos para entender la dinámica de un sistema es una \textit{Teoría de campos local}, es decir, que lo que le ocurre a un punto del campo solo afecte a otros puntos cercanos y no a todo el campo y que además sea relativista.
\subsection{Formalismo Lagrangiano}
Para ello definimos la densidad lagrangiana $\mathcal{L}$ la cual depende de las coordenadas generalizadas $\phi_{i}$ y de sus velocidades generalizadas $\partial_{\mu}\phi_{i}$
\subsection{Formalismo Hamiltoniano}

Una vez descrito el formalismo Lagrangiano, podemos ir al formalismo Hamiltoniano de forma análoga a como lo haciamos en mecánica clásica, para ello definimos el \textit{Momento canónico generalizado} como 

\[\Pi(x)=\pdv{\mathcal{L}}{\partial(\partial_{0}\phi_{i})}\text{ con }\partial_{0}\phi_{i}= \dot{\phi}\]

De forma que el Hamiltoniano queda definido como 

\[\mathcal{H}=\Pi\phi_{i}-\mathcal{L}\]

\subsection{Ejemplos}

\begin{example}[Campo escalar real]
  
\end{example}
\begin{example}[Campo escalar real]
  
\end{example}
\begin{example}[Campo escalar real]
  
\end{example}
\begin{example}[Campo escalar real]
  
\end{example}
\begin{example}[Campo escalar real]
  
\end{example}
\section{Teorema de Noether}
Este teorema relaciona las simetrías de la integral de acción con la existencia de leyes de conservación. Consideremos una transformación infinitesimal que puede afectar tanto a las coordenadas como a los campos, esto es

\begin{aligned}
x^{\mu} \rightarrow x^{\prime \mu} & =x^{\mu}+\delta x^{\mu} \\
\phi_{i}(x) \rightarrow \phi_{i}^{\prime}\left(x^{\prime}\right) & =\phi_{i}(x)+\delta \phi_{i}(x) \tag{3.43}
\end{aligned}

Se dice que estas transformaciones definen una simetría de la acción si dejan $S$ invariante para todo $\phi_{i}(x)$ (sea o no solución de las ecuaciones de movimiento).

Una observación importante es que una simetría requiere que la acción $S$ sea invariante, pero no necesariamente la densidad Lagrangiana $\mathscr{L}$. Es decir, la condición de simetría es $S^{\prime}\left[\phi^{\prime}\right]=S[\phi]$ y por tanto

\begin{equation*}
\delta S[\phi]=\int_{V^{\prime}} d^{4} x^{\prime} \mathscr{L}\left(\phi^{\prime}\left(x^{\prime}\right), \partial_{\mu} \phi^{\prime}\left(x^{\prime}\right)\right)-\int_{V} d^{4} x \mathscr{L}\left(\phi(x), \partial_{\mu} \phi(x)\right)=0 \tag{3.44}
\end{equation*}

donde $V^{\prime}$ denota el mismo volumen de integración $V$ pero expresado en coordendas $x^{\prime}$. Haciendo el cambio de variable a las coordenadas $x$ en la primera integral, tendremos que usar el determinante jacobiano que a primer orden es

\begin{equation*}
d^{4} x^{\prime}=d^{4} x\left(1+\partial_{\mu} \delta x^{\mu}\right) \tag{3.45}
\end{equation*}

y por tanto

\begin{aligned}
\delta S[\phi] & =\int_{V} d^{4} x\left(1+\partial_{\mu} \delta x^{\mu}\right)\left(\mathscr{L}\left(\phi^{\prime}(x), \partial_{\mu} \phi^{\prime}(x)\right)+\partial_{\mu} \mathscr{L}\left(\phi(x), \partial_{\mu} \phi(x)\right) \delta x^{\mu}\right) \\
& -\int_{V} d^{4} x \mathscr{L}\left(\phi(x), \partial_{\mu} \phi(x)\right) \tag{3.46}
\end{aligned}

Usando la definición (3.5)

\begin{equation*}
\phi_{i}^{\prime}(x)=\phi_{i}(x)+\delta_{0} \phi_{i}(x) \tag{3.47}
\end{equation*}

con

\begin{equation*}
\delta_{0} \phi_{i}(x)=\delta \phi_{i}(x)-\delta x^{\lambda} \partial_{\lambda} \phi_{i}(x) \tag{3.48}
\end{equation*}

podemos escribir

\begin{equation*}
\mathscr{L}\left(\phi^{\prime}(x), \partial_{\mu} \phi^{\prime}(x)\right)=\mathscr{L}\left(\phi(x), \partial_{\mu} \phi(x)\right)+\left(\frac{\partial \mathscr{L}}{\partial \phi_{i}} \delta_{0} \phi_{i}+\frac{\partial \mathscr{L}}{\partial\left(\partial_{\mu} \phi_{i}\right)} \delta_{0}\left(\partial_{\mu} \phi_{i}\right)\right) \tag{3.49}
\end{equation*}


Ahora usando $\delta_{0}\left(\partial_{\mu} \phi_{i}\right)=\partial_{\mu}\left(\delta_{0} \phi\right)$ tenemos

\begin{equation*}
\mathscr{L}\left(\phi^{\prime}(x), \partial_{\mu} \phi^{\prime}(x)\right)=\mathscr{L}\left(\phi(x), \partial_{\mu} \phi(x)\right)+\left(\frac{\partial \mathscr{L}}{\partial \phi_{i}}-\partial_{\mu}\left(\frac{\partial \mathscr{L}}{\partial\left(\partial_{\mu} \phi_{i}\right)}\right)\right) \delta_{0} \phi_{i}+\partial_{\mu}\left(\frac{\partial \mathscr{L}}{\partial\left(\partial_{\mu} \phi_{i}\right)} \delta_{0} \phi_{i}\right) \tag{3.50}
\end{equation*}


Sustituyendo en (3.46) y quedándonos a primer orden obtenemos

\begin{aligned}
\delta S[\phi] & =\int_{V} d^{4} x\left[\left(\partial_{\mu} \delta x^{\mu}\right) \mathscr{L}+\left(\partial_{\mu} \mathscr{L}\right) \delta x^{\mu}\right. \\
& \left.+\left(\frac{\partial \mathscr{L}}{\partial \phi_{i}}-\partial_{\mu}\left(\frac{\partial \mathscr{L}}{\partial\left(\partial_{\mu} \phi_{i}\right)}\right)\right) \delta_{0} \phi_{i}+\partial_{\mu}\left(\frac{\partial \mathscr{L}}{\partial\left(\partial_{\mu} \phi_{i}\right)} \delta_{0} \phi_{i}\right)\right]=0 \tag{3.51}
\end{aligned}
que se puede reescribir como


\begin{equation*}
\delta S[\phi]=\int_{V} d^{4} x\left[\left(\frac{\partial \mathscr{L}}{\partial \phi_{i}}-\partial_{\mu}\left(\frac{\partial \mathscr{L}}{\partial\left(\partial_{\mu} \phi_{i}\right)}\right)\right) \delta_{0} \phi_{i}+\partial_{\mu}\left(\frac{\partial \mathscr{L}}{\partial\left(\partial_{\mu} \phi_{i}\right)} \delta_{0} \phi_{i}+\mathscr{L} \delta x^{\mu}\right)\right]=0 \tag{3.52}
\end{equation*}


Puesto que esta variación debe ser cero para cualquier valor del volumen de integración $V$, el integrando debe ser cero por lo que

\begin{equation*}
\left(\frac{\partial \mathscr{L}}{\partial \phi_{i}}-\partial_{\mu}\left(\frac{\partial \mathscr{L}}{\partial\left(\partial_{\mu} \phi_{i}\right)}\right)\right) \delta_{0} \phi_{i}+\partial_{\mu}\left(\frac{\partial \mathscr{L}}{\partial\left(\partial_{\mu} \phi_{i}\right)} \delta_{0} \phi_{i}+\mathscr{L} \delta x^{\mu}\right)=0 \tag{3.53}
\end{equation*}


Ahora bien, el primer paréntesis no son más que las ecuaciones de Euler-Lagrange (3.37), y por tanto sobre soluciones de las ecuaciones clásicas de movimiento $\phi_{i}(x)=\phi_{i}^{c l}(x)$ llegamos a

\begin{equation*}
\partial_{\mu} j^{\mu}\left(\phi^{c l}\right)=0 \tag{3.54}
\end{equation*}

donde

\begin{equation*}
j^{\mu}=\left(\frac{\partial \mathscr{L}}{\partial\left(\partial_{\mu} \phi_{i}\right)} \delta_{0} \phi_{i}+\mathscr{L} \delta x^{\mu}\right) \tag{3.55}
\end{equation*}


Supongamos ahora que las transformaciones dependen de una serie de parámetros $\theta^{a}(x) \operatorname{con} a=1, \ldots, N$ de forma que

\begin{aligned}
\delta x^{\mu} & =\theta^{a}(x) X_{a}^{\mu}(x) \\
\delta \phi_{i}(x) & =\theta^{a}(x) Y_{i, a}\left(\phi, \partial_{\mu} \phi\right) \tag{3.56}
\end{aligned}

Estas transformaciones pueden corresponder a cada uno de los parámetros de un grupo de Lie. En este caso, (3.48) se escribirá

\begin{equation*}
\delta_{0} \phi_{i}(x)=\theta^{a}(x)\left(Y_{i, a}(x)-X_{a}^{\lambda}(x) \partial_{\lambda} \phi_{i}(x)\right) \tag{3.57}
\end{equation*}

y por tanto la corriente (3.55) se escribirá

\begin{equation*}
j^{\mu}=\left(\frac{\partial \mathscr{L}}{\partial\left(\partial_{\mu} \phi_{i}\right)}\left(Y_{i, a}(x)-X_{a}^{\lambda}(x) \partial_{\lambda} \phi_{i}(x)\right)+\mathscr{L} X_{a}^{\mu}(x)\right) \theta^{a}(x) \tag{3.58}
\end{equation*}


Puesto que los parámetros $\theta^{a}$ son independientes, podemos definir una corriente conservada para cada uno de ellos:

\begin{equation*}
j_{a}^{\mu}=\frac{\partial \mathscr{L}}{\partial\left(\partial_{\mu} \phi_{i}\right)}\left(Y_{i, a}(x)-X_{a}^{\lambda}(x) \partial_{\lambda} \phi_{i}(x)\right)+\mathscr{L} X_{a}^{\mu}(x), \quad a=1, \ldots, N \tag{3.59}
\end{equation*}

de forma que sobre soluciones de las ecuaciones del movimiento

\begin{equation*}
\partial_{\mu} j_{a}^{\mu}\left(\phi^{c l}\right)=0, \quad a=1, \ldots, N \tag{3.60}
\end{equation*}


Es decir, si la acción es invariante bajo las transformaciones de simetría (3.56) existen $N$ corrientes de Noether $j_{a}^{\mu}$ conservadas cuando se evalúan sobre las soluciones clásicas. Es decir, hay una corriente conservada por cada generador de las transformaciones de simetría. Este es el Teorema de Noether.

Para cada corriente conservada tenemos una carga conservada dada por:

\begin{equation*}
Q_{a} \equiv \int d^{3} x j_{a}^{0}(x) \tag{3.61}
\end{equation*}


Efectivamente, usando (3.60) tenemos

\begin{equation*}
\partial_{0} Q_{a}=\int d^{3} x \partial_{0} j_{a}^{0}(x)=-\int d^{3} x \partial_{i} j_{a}^{i}(x)=0 \tag{3.62}
\end{equation*}

donde en la última integral hemos usado el teorema de Stokes y asumido que los campos se anulan en el infinito.
\section{Transformaciones de simetria espacio-temporales}
Dependiendo de la forma de las transformaciones (3.56), podemos distinguir distintos tipos de simetrías:

Simetrías globales: los parámetros $\theta^{a}$ son independientes de las coordenadas; las simetrías globales son un caso especial de las simetrías locales.

Simetrías locales: los parámetros $\theta^{a}(x)$ dependen de las coordenadas
Simetrías internas: $X_{a}^{\mu}(x)=0$, es decir, las coordenadas no cambian.
Simetrías espacio-temporales: $X_{a}^{\mu}(x) \neq 0$, es decir, involucran cambios en las coordenadas.
\subsection{Traslaciones}
El ejemplo más sencillo es la simetría bajo traslaciones, dadas en el caso infinitesimal por

\begin{aligned}
x^{\prime \mu} & =x^{\mu}+\theta^{\mu} \\
\phi_{i}^{\prime}\left(x^{\prime}\right) & =\phi_{i}(x) \tag{3.63}
\end{aligned}

En este caso, el índice $a=1, \ldots, 4$ podemos sustituirlo por un índice griego $v$, de forma que

\begin{aligned}
X_{v}^{\mu} & =\delta^{\mu}{ }_{v} \\
Y_{i, v} & =0 \tag{3.64}
\end{aligned}

La corriente conservada se lee en este caso

\begin{equation*}
\theta^{\mu v}=\frac{\partial \mathscr{L}}{\partial\left(\partial_{\mu} \phi_{i}\right)} \partial^{v} \phi_{i}-\eta^{\mu v} \mathscr{L} . \tag{3.65}
\end{equation*}
donde hemos usado $\theta^{\mu \nu}=\eta^{\nu \rho} \theta_{\rho}^{\mu}$. Esta corriente conservada se conoce como tensor energíamomento.

La conservación de la corriente implica

\begin{equation*}
\partial_{\mu} \theta^{\mu v}\left(\phi^{c l}\right)=0 \tag{3.66}
\end{equation*}
y las cargas son
}begin{equation*}
P^{v} \equiv \int d^{3} x \theta^{0 v}\left(\phi^{c l}\right) \tag{3.67}
\end{equation*}
que es el cuadrimomento del campo.
Nótese que $\theta^{\mu \nu}$ no es simétrico en sus índices. Sin embargo, podemos construir un tensor simétrico

\begin{equation*}
T^{\mu v}=\theta^{\mu v}+\partial_{\rho} A^{\rho \mu v} \tag{3.68}
\end{equation*}
con $A$ un tensor antisimétrico en los índices $\rho, \mu$ elegido adecuadamente. Precisamente debido a la antisimetría de $A$, el nuevo tensor también es conservado
$$
\partial_{\mu} T^{\mu v}=\partial_{\mu} \partial_{\rho} A^{\rho \mu v}=0
$$

Para $\mu=0$ tenemos $\partial_{\rho} A^{\rho 0 v}=\partial_{i} A^{i 0 v}$ que es una divergencia espacial y por tanto no cambia las cargas conservadas en (3.67). En otras palabras, $\theta^{\mu v}$ y $T^{\mu v}$ son físicamente equivalentes y podemos trabajar con cualquiera de ellos. $T^{\mu v}$ se conoce como tensor energía-momento simetrizado de Belinfante-Rosenfeld.
\subsection{Transformaciones de Lorentz}
Consideremos ahora el caso en el que la acción es invariante bajo transformaciones de Lorentz dadas anteriormente en (3.6) y (3.31)

\begin{aligned}
x^{\prime \mu} & =x^{\mu}+\omega_{v}^{\mu} x^{v} \\
\phi_{i}^{\prime}\left(x^{\prime}\right) & =\phi_{i}(x)-\frac{i}{2} \omega_{\mu v}\left[S^{\mu v}\right]_{i}^{j} \phi_{j}(x) \tag{3.69}
\end{aligned}
de forma que ahora el índice $a$ se convierte en un par de índices griegos $v, \rho$

\begin{gather*}
X^{\mu v \rho}=\frac{1}{2}\left(\eta^{\mu v} x^{\rho}-\eta^{\mu \rho} x^{v}\right) \\
Y_{i}^{v \rho}=-\frac{i}{2}\left[S^{v \rho}\right]_{i}^{j}{ }_{i} \phi_{j}(x) \tag{3.70}
\end{gather*}
donde los generadores $\left[S^{v \rho}\right]_{i}^{j}$ se escribirán en la representación correspondiente a los campos $\phi_{i}$.

La correspondiente corriente de Noether será

\begin{equation*}
j^{\mu}=\left[\frac{\partial \mathscr{L}}{\partial\left(\partial_{\mu} \phi_{i}\right)}\left(-\frac{i}{2}\left[S^{v \rho}\right]_{i}^{j} \phi_{j}(x)-X^{\lambda v \rho}(x) \partial_{\lambda} \phi_{i}(x)\right)+\mathscr{L} X^{\mu v \rho}(x)\right] \omega_{v \rho} \tag{3.71}
\end{equation*}

Esta corriente se puede escribir como

\begin{equation*}
j^{\mu}=\frac{1}{2} M^{\mu \nu \rho} \omega_{v \rho} \tag{3.72}
\end{equation*}
donde $M^{\mu v \rho}$ se puede escribir en función del tensor energía-momento como

\begin{equation*}
M^{\mu v \rho}=x^{v} \theta^{\mu \rho}-x^{\rho} \theta^{\mu v}+S^{\mu v \rho} \tag{3.73}
\end{equation*}
con

\begin{equation*}
S^{\mu v \rho}=-i \frac{\partial \mathscr{L}}{\partial\left(\partial_{\mu} \phi_{i}\right)}\left[S^{v \rho}\right]_{i}^{j} \phi_{j} \tag{3.74}
\end{equation*}

La conservación de la corriente implica

\begin{equation*}
\partial_{\mu} M^{\mu v \rho}=0 \tag{3.75}
\end{equation*}

Por tanto, tenemos 6 corrientes conservadas, $M^{\mu v \rho}$ una por cada uno de los 6 parámetros independientes de la transformación de Lorentz $\omega_{v \rho}$. Nótese que por construcción $M^{\mu v \rho}$ es antisimétrico en ( $v \rho$ ). Las correspondientes cargas conservadas son:

\begin{equation*}
M^{v \rho}=\int d^{3} x M^{0 v \rho}=\int d^{3} x\left(x^{v} p^{\rho}-x^{\rho} p^{v}+S^{0 v \rho}\right) \tag{3.76}
\end{equation*}
$\operatorname{con} p^{v}=\theta^{0 v}$ la densidad de cuadrimomento.
Nótese que las 3 cargas conservadas $M^{i j}$, asociadas a las rotaciones, proporcionan el momento angular del campo como

\begin{equation*}
J^{i}=\frac{1}{2} \epsilon^{i j k} M^{j k} \tag{3.77}
\end{equation*}
que como vemos tienen la contribución del momento angular orbital ( $\left.x^{i} p^{j}-x^{j} p^{i}\right)$ y de spin $S^{0 i j}$. Las $M^{0 i}$ corresponderían a las 3 cargas conservadas asociadas a los boosts.\sidenote{La interpretación física de estas cargas conservados no es tan directa y puede verse que corresponden a la posición del centro de momentos del campo en el instante $t=0$.}