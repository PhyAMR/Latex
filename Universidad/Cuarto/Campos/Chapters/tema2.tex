\setchapterpreamble[u]{\margintoc}
\chapter{Teoría clásica de campos}
\labch{Part}

\begin{center}
  \large Todo esto lo he sacado del \cite{Dobdado}
\end{center}
\section{Campos clásicos relativistas}
\section{Dinámica de los campos clásicos}

Por la mecánica lagrangiana sabemos que para estudiar la dinámica usamos las llamadas coordenadas generalizadas $q_{i}(t)$. En el formalismo de campos utilizamos el propio campo para definir la dinámica, es decir, supongamos que tenemos un campo $\Phi$ el cual esta compuesto por el valor del campo en un punto que denotamos $\phi_{i}(\vec{x},t)$ estas $\phi_{i}$ son las coordenadas generalizadas en el formalismo de campos. Esto se puede ver como:

\[q_{i}\to\phi_{i}(\vec{x},t)\]

Esto quiere decir, que vamos a tener tantas coordenadas como puntos del espacio esten definidos en nuestro campo y cada punto del espacio $\phi_{i}$ esta definido por unas coordeandas. 

De esta forma, la teoría de campos que buscamos para entender la dinámica de un sistema es una \textit{Teoría de campos local}, es decir, que lo que le ocurre a un punto del campo solo afecte a otros puntos cercanos y no a todo el campo y que además sea relativista.
\subsection{Formalismo Lagrangiano}
Para ello definimos la densidad lagrangiana $\mathcal{L}$ la cual depende de las coordenadas generalizadas $\phi_{i}$ y de sus velocidades generalizadas $\partial_{\mu}\phi_{i}$
\subsection{Formalismo Hamiltoniano}

Una vez descrito el formalismo Lagrangiano, podemos ir al formalismo Hamiltoniano de forma análoga a como lo haciamos en mecánica clásica, para ello definimos el \textit{Momento canónico generalizado} como 

\[\Pi(x)=\pdv{\mathcal{L}}{\partial(\partial_{0}\phi_{i})}\text{ con }\partial_{0}\phi_{i}= \dot{\phi}\]

De forma que el Hamiltoniano queda definido como 

\[\mathcal{H}=\Pi\phi_{i}-\mathcal{L}\]

\subsection{Ejemplos}

\begin{example}[Campo escalar real]
  
\end{example}
\begin{example}[Campo escalar real]
  
\end{example}
\begin{example}[Campo escalar real]
  
\end{example}
\begin{example}[Campo escalar real]
  
\end{example}
\begin{example}[Campo escalar real]
  
\end{example}
\section{Teorema de Noether}
\section{Transformaciones de simetria espacio-temporales}
\subsection{Traslaciones}
\subsection{Transformaciones de Lorentz}