
\documentclass[
	a4paper, 
	fontsize=10pt,
	twoside=true, 
	%open=any,
	%chapterentrydots=true, 
	numbers=noenddot, 
]{kaobook}

\ifxetexorluatex
	\usepackage{polyglossia}
	\setmainlanguage{spanish}
\else
	\usepackage[spanish]{babel} 
\fi
\usepackage[spanish]{csquotes}	

\usepackage{pdfpages}
\usepackage{blindtext}
%\usepackage{showframe} % Uncomment to show boxes around the text area, margin, header and footer
%\usepackage{showlabels} % Uncomment to output the content of \label commands to the document where they are used

% Load the bibliography package
\usepackage{kaobiblio}
\addbibresource{main.bib} % Bibliography file

% Load mathematical packages for theorems and related environments
\usepackage[framed=true]{kaotheorems}

% Load the package for hyperreferences
\usepackage{kaorefs}

%%%%%%%%%%%%%%%%Paquetes%%%%%%%%%%%%%%%%%%%%%%%
\usepackage{witharrows}
%%%%%%%%%%%%%%%%%%%%%%%%%%%%%%%%%%%%%%%

\graphicspath{{../Images}{images/}} % Paths in which to look for images

\makeindex[columns=3, title=Alphabetical Index, intoc] % Make LaTeX produce the files required to compile the index

\makeglossaries % Make LaTeX produce the files required to compile the glossary

    
    \newglossaryentry{computer}{
      name=computer,
      description={is a programmable machine that receives input, stores and manipulates data, and provides output in a useful format}
    }

    % Glossary entries (used in text with e.g. \acrfull{fpsLabel} or \acrshort{fpsLabel})
    \newacronym[longplural={Frames per Second}]{fpsLabel}{FPS}{Frame per Second}
 % Include the glossary definitions

\makenomenclature % Make LaTeX produce the files required to compile the nomenclature

% Reset sidenote counter at chapters
\counterwithin*{sidenote}{chapter}

\begin{document}

\titlehead{Apuntes de }
\subject{Asignatura}

\title[]{Título}
\subtitle{Customise this page according to your needs}

\author[]{}

\date{\today}

\publishers{An Awesome Publisher}

\frontmatter

\makeatletter
\uppertitleback{\@titlehead}

\makeatletter
\uppertitleback{\@titlehead} % Header

\lowertitleback{
  \textbf{Disclaimer}\\
  You can edit this page to suit your needs. For instance, here we have a no copyright statement, a colophon and some other information. This page is based on the corresponding page of Ken Arroyo Ohori's thesis, with minimal changes.

  \medskip

  \textbf{No copyright}\\
  \cczero\ This book is released into the public domain using the CC0 code. To the extent possible under law, I waive all copyright and related or neighbouring rights to this work.

  To view a copy of the CC0 code, visit: \\\url{http://creativecommons.org/publicdomain/zero/1.0/}

  \medskip

  \textbf{Colophon} \\
  This document was typeset with the help of \href{https://sourceforge.net/projects/koma-script/}{\KOMAScript} and \href{https://www.latex-project.org/}{\LaTeX} using the \href{https://github.com/fmarotta/kaobook/}{kaobook} class.

  The source code of this book is available at:\\\url{https://github.com/fmarotta/kaobook}

  (You are welcome to contribute!)

  \medskip

  \textbf{Publisher} \\
  First printed in May 2019 by \@publishers
}
\makeatother

\maketitle

\begingroup % Local scope for the following commands

% Define the style for the TOC, LOF, and LOT
%\setstretch{1} % Uncomment to modify line spacing in the ToC
%\hypersetup{linkcolor=blue} % Uncomment to set the colour of links in the ToC
\setlength{\textheight}{230\hscale} % Manually adjust the height of the ToC pages

% Turn on compatibility mode for the etoc package
\etocstandarddisplaystyle % "toc display" as if etoc was not loaded
\etocstandardlines % "toc lines" as if etoc was not loaded

\tableofcontents % Output the table of contents

\listoffigures % Output the list of figures

% Comment both of the following lines to have the LOF and the LOT on different pages
\let\cleardoublepage\bigskip
\let\clearpage\bigskip

\listoftables % Output the list of tables

\endgroup

%----------------------------------------------------------------------------------------
%	MAIN BODY
%----------------------------------------------------------------------------------------

\mainmatter % Denotes the start of the main document content, resets page numbering and uses arabic numbers
\setchapterstyle{kao} % Choose the default chapter heading style



\pagelayout{wide}

\addpart{Mecánica Newtoniana}

\pagelayout{wide}

\pagelayout{wide}

\addpart{Principios dinámicos}

\pagelayout{wide}

\pagelayout{wide}

\addpart{Geometria: Rieman y minkowsky}

\pagelayout{wide}

\pagelayout{wide}

\addpart{La gravitación en la relatividad especial}

\pagelayout{wide}

\pagelayout{wide}

\addpart{Geometria de Schwarzchild}

\pagelayout{wide}

\pagelayout{wide}

\addpart{Pruebas clásicas de la relatividad de Einstein}

\pagelayout{wide}

\pagelayout{wide}

\addpart{Agujeros negros}

\pagelayout{wide}





\end{document}
