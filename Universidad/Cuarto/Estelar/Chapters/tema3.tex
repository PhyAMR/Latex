\setchapterpreamble[u]{\margintoc}
\chapter{Ecuación de estado del interior estelar}
\labch{Part}

\section{El estado termodinámico del interior estelar}
Las propiedades macroscópicas estelares están relacionadas con los fenómenos que ocurren a nivel microscópico. Estos dependen del estado termodinámico del interior estelar.\\
¿Están las estrellas en equilibrio termodinámico? (estado de equilibrio cuando hay suficientes interacciones entre partículas en el que el campo de radiación es isótropo, la distribución de energía de los fotones se describe por la ley de Planck y la distribución de partículas materiales y fotones depende de una única temperatura). ¿Son sistemas isotermos encerrados adiabáticamente?

En principio NO, pues:
\begin{itemize}
  \item La temperatura interna no es constante
  \item No son sistemas aislados (generan y emiten energía)
\end{itemize}

Pero, localmente, se puede considerar que las desviaciones son suficientemente pequeñas para asumir equilibrio termodinámico local (ETL): en regiones de tamaño pequeño comparado con la estrella, pero grande comparado con el recorrido libre medio de las partículas $\rightarrow$ se puede definir una temperatura local bien definida para describir la distribución de partículas.\\

Esto se debe a que las variciones de temperatura son muy pequeñas 
Recorrido libre medio de los fotones:
\section{Presión mecánica de un gas perfecto}
\section{Conceptos de física estadística}
\subsection{Descripción Mecanico-cuántica del gas}
\section{Ecuación de estado}
\subsection{Gas perfecto clásico}
\subsection{Peso molecular medio}
\subsection{Degeneración electrónica}
\subsubsection{Degeneración completa}
\subsubsection{Degeneración parcial}
\subsection{Presión de radiación}
\subsection{Resumen}

\section{Relaciones termodinámicas}
\section{Ionización}
\section{Otros efectos de la ecuación de estado}