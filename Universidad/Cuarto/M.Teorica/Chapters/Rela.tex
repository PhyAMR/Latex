\setchapterpreamble[u]{\margintoc}
\chapter{Repaso de la cinemática relativista}
\labch{Part}


\section{Postulados de la teoría especial de la relatividad}

Un sistema de referencia es un sistema de coordenadas para señalar la posición espacial de un suceso y un reloj que indique cuándo ha ocurrido.

Un sistema de referencia inercial es aquel en el que una partícula libre, sobre la que no actúa ninguna fuerza, permanece siempre en el mismo estado de movimiento, que llamaremos estado de movimiento inercial y que determinaremos experimentalmente.

\begin{proposition}[Principio de relatividad]
  Las leyes de la física, en ausencia de fuerzas gravitatorias, son idénticas en todos los sistemas de referencia inerciales.
\end{proposition}

Según este principio de relatividad, las ecuaciones que describen las leyes de la física tienen la misma forma en todos los sistema de referencia inerciales. Sin embargo, las expresiones "leyes de la física" y "sistemas de referencia inerciales" tiene distinto significado según el contexto.

En la física aristotélica, el estado de movimiento inercial es el reposo. Los sistemas de referencia inerciales quedan así determinados por la condición de que las partículas libres permanezcan en reposo y, por tanto, todos los sistemas de referencia inerciales están en reposo unos con respecto a otros. En estos sistemas de referencia, los cuerpos tan solo poseen una velocidad no nula cuando actúa algún agente externo sobre ellos. El principio de relatividad aristotélico atañe solo a las leyes de la estática.

En la física newtoniana, el estado de movimiento inercial es el desplazamiento con velocidad constante. Los cuerpos libres se mueven con velocidad constante en cualquier sistema de referencia inercial. Todos los sistemas de referencia inerciales se mueven, por tanto, con velocidad relativa constante. Los cuerpos solo cambian de velocidad si actúa algún agente externo sobre ellos y la propagación de señales es instantánea, es decir, las fuerzas actúan instantáneamente y producen aceleraciones. En este contexto, el principio de relatividad (galileano) se refiere a las leyes de la dinámica de los cuerpos materiales, es decir, a la mecánica.

Sin embargo, no existen interacciones instantáneas. Al introducir el campo electromagnético, es necesario tener este hecho en cuenta. La velocidad de la
luz en el vacío es la velocidad máxima que puede alcanzar una interacción. Esta es una ley física y, por tanto, debe ser válida en todos los sistemas de referencia inerciales.

\begin{proposition}[Velocidad de la luz]
  La velocidad de la luz en el vacío $c$ es constante e igual en todos los sistemas de referencia inerciales.
\end{proposition}

Estos dos postulados constituyen la base de la teoría especial de la relatividad. El principio de relatividad einsteniano generaliza así todos los anteriores y extiende su validez a todas las leyes de la física conocidas hasta hoy, excluyendo la interacción gravitatoria. La mecánica newtoniana se recupera en el límite $c \rightarrow \infty$, es decir, en el límite de interacciones instantáneas.

En la mecánica newtoniana, el espacio es relativo: la distancia entre dos sucesos no simultáneos depende del sistema de referencia. En efecto, sean $\left(t_{1}, \vec{x}_{1}\right)$ y $\left(t_{2}, \vec{x}_{2}\right)$ dos sucesos en el sistema de referencia inercial $S$. En otro sistema de referencia inercial $S^{\prime}$, que se mueve con velocidad $\vec{v}$, la posición es $\vec{x}^{\prime}(t)=\vec{x}(t)-\vec{v} t$ y, por tanto,

\[\begin{aligned}
  \left|\vec{x}_{2}^{\prime}\left(t_{2}\right)-\vec{x}_{1}^{\prime}\left(t_{1}\right)\right|^{2}= & \left|\vec{x}_{2}\left(t_{2}\right)-\vec{x}_{1}\left(t_{1}\right)\right|^{2}+v^{2}\left(t_{2}-t_{1}\right)^{2}- \\
  & -2\left(t_{2}-t_{1}\right) \vec{v} \cdot\left[\vec{x}_{2}\left(t_{2}\right)-\vec{x}_{1}\left(t_{1}\right)\right] .
  \end{aligned}\]

  En la mecánica newtoniana el tiempo es absoluto 

  \begin{lemma}[Tiempo absoluto]
    Si el tiempo es absoluto, $t^{\prime}=t$, entonces dos sucesos simultáneos en un sistema inercial lo son en cualquier otro ya que la propagación de señales es instantánea.
  \end{lemma}

  Como consecuencia, tenemos la ley de suma de velocidades galileana:
  \begin{definition}[Ley galileana de suma de voelocidades]

    Si en $S$ una partícula tiene velocidad $\vec{V}$ y $S^{\prime}$ se mueve con velocidad $\vec{v}$ con respecto a $S$, entonces la velocidad de la partícula en $S^{\prime}$ es $\vec{V}^{\prime}=\vec{V}-\vec{v}$. En efecto, para dos sucesos próximos $\left(t_{1}, \vec{x}_{1}\right)$ y $\left(t_{2}=t_{1}+\mathrm{d} t, \vec{x}_{2}=\vec{x}_{1}+\mathrm{d} \vec{x}\right)$,
    \[\vec{V}^{\prime}=\frac{\mathrm{d} \vec{x}^{\prime}}{\mathrm{d} t^{\prime}}=\frac{\mathrm{d} \vec{x}^{\prime}}{\mathrm{d} t}=\frac{\mathrm{d} \vec{x}}{\mathrm{~d} t}-\vec{v}=\vec{V}-\vec{v}\]
    
    
    Esta ley de composición es incompatible con el carácter universal y finito de la velocidad de la luz. De hecho, debido a la constancia y finitud de la velocidad de la luz, en relatividad especial, el tiempo es relativo, es decir, depende del sistema de referencia en el que se mida: dos sucesos simultáneos en un sistema de referencia inercial no son necesariamente simultáneos en otro, como veremos a continuación.

  En relatividad especial, el tiempo y el espacio son relativos, pero no todo es relativo como a menudo se dice. Veremos que existen cantidades absolutas que son de gran importancia. Entre ellas, el intervalo espaciotemporal ocupa un lugar sobresaliente.

  Finalmente, podemos eliminar la limitación que se refiere a las fuerzas gravitatorias y enunciar el principio de relatividad general: todas las leyes de la física son idénticas en todos los sistemas de referencia. Notemos que, ahora, la universalidad de las leyes es completa y no está restringida a los sistemas de referencia inerciales. Este es uno de los principios en los que está basada la teoría general de la relatividad, sobre la que no comentaremos nada más
  \end{definition}
\section{Transformaciones de Lorentz}

En un sistema de referencia inercial $S$, consideremos los sucesos "emisión de una señal luminosa en $\vec{x}_{1}$ en el instante $t_{1}$ " y "recepción de la señal en $\vec{x}_{2}$ en el instante $t_{2}$ ". Puesto que la velocidad de propagación de la señal es $c$, se satisface la relación:

\begin{equation*}
-c^{2}\left(t_{2}-t_{1}\right)^{2}+\left(\vec{x}_{2}-\vec{x}_{1}\right)^{2}=0 \tag{C.1}
\end{equation*}


En otro sistema de referencia inercial $S^{\prime}$, estos dos sucesos estarán caracterizados por sus vectores de posición $\vec{x}_{1}^{\prime}$ y $\vec{x}_{2}^{\prime}$ y los instantes en que se producen $t_{1}^{\prime}$ y $t_{2}^{\prime}$ respectivamente. Como la velocidad de propagación de la señal es también $c$, se satisface la relación

\begin{equation*}
-c^{2}\left(t_{2}^{\prime}-t_{1}^{\prime}\right)^{2}+\left(\vec{x}_{2}^{\prime}-\vec{x}_{1}^{\prime}\right)^{2}=0 \tag{C.2}
\end{equation*}


La transformación lineal de coordenadas y tiempos que satisfacen la condición de invariancia que acabamos de exponer recibe el nombre de \textit{transformación de Lorentz pura} :

\begin{equation*}
t^{\prime}=\gamma\left(t-c^{-2} \vec{v} \cdot \vec{x}\right), \quad \vec{x}^{\prime}=\vec{x}+(\gamma-1)(\hat{v} \cdot \vec{x}) \hat{v}-\gamma \vec{v} t \tag{C.3}
\end{equation*}

donde $\gamma^{-1}=\sqrt{1-\vec{v}^{2} / c^{2}}$ y $\hat{v}=\vec{v} / v$ siendo $v=|\vec{v}|$. Si elegimos el sistema de referencia $S$ de forma que $\vec{v}=v \hat{x}$, esta transformación adquiere la forma familiar

\[t^{\prime}=\gamma\left[t-\left(v / c^{2}\right) x\right], \quad x^{\prime}=\gamma(x-v t), \quad y^{\prime}=y, \quad z^{\prime}=z\]

Sea $\vec{V}$ la velocidad de una partícula en $S, \vec{V}^{\prime}$ su velocidad de $S^{\prime}$ y $\vec{v}$ la velocidad de $S^{\prime}$ con respecto a $S$. Entonces

\[\vec{V}^{\prime}=\frac{\mathrm{d} \vec{x}^{\prime}}{\mathrm{d} t^{\prime}}=\frac{\mathrm{d} \vec{x}^{\prime}}{\mathrm{d} t} \frac{\mathrm{~d} t}{\mathrm{~d} t^{\prime}}=\frac{\mathrm{d} \vec{x}^{\prime}}{\mathrm{d} t} / \frac{\mathrm{d} t^{\prime}}{\mathrm{d} t}
\]

Derivando las ecuaciones C. 3 con respecto a $t$ y teniendo en cuenta que $\vec{V}=\mathrm{d} \vec{x} / \mathrm{d} t$, obtenemos la ley de adición de velocidades:

\[\vec{V}^{\prime}=\frac{\vec{V}+(\gamma-1)(\hat{v} \cdot \vec{V}) \hat{v}-\gamma \vec{v}}{\gamma\left(1-c^{-2} \vec{v} \cdot \vec{V}\right)}\]


Es ilustrativo escribir las leyes de transformación para la componente paralela $V_{\|}$a $\vec{v}$ y la componente perpendicular $\vec{V}_{\perp}$ de $\vec{V}=V_{\|} \hat{v}+\vec{V}_{\perp}$ :

\[V_{\|}^{\prime}=\frac{V_{\|}-v}{1-c^{-2} v V_{\|}}, \quad \vec{V}_{\perp}^{\prime}=\frac{\vec{V}_{\perp}}{\gamma\left(1-c^{-2} v V_{\|}\right)}\]

\section{Elemento de linea}

Podemos definir el intervalo espaciotemporal entre dos sucesos $\left(t_{1}, \vec{x}_{1}\right)$ y $\left(t_{2}, \vec{x}_{2}\right)$ cualesquiera (no necesariamente conectados mediante una señal luminosa) como la cantidad $s_{12}$ tal que
\[s_{12}^{2}=-c^{2}\left(t_{2}-t_{1}\right)^{2}+\left(\vec{x}_{2}-\vec{x}_{1}\right)^{2}\]
Resulta útil introducir el elemento de línea $\mathrm{ds}^{2}$ entre dos sucesos próximos $(t, \vec{x})$ y $(t+\mathrm{d} t, \vec{x}+\mathrm{d} \vec{x}):$

\[\mathrm{d} s^{2}=-c^{2} \mathrm{~d} t^{2}+\mathrm{d} \vec{x}^{2}\]

que es invariante bajo las transformaciones de Lorentz.

\begin{itemize}
  \item Decimos que dos sucesos están separados temporalmente o que su intervalo es de género tiempo cuando el cuadrado de su intervalo $s^{2}$ es negativo. Entonces existe un sistema de referencia inercial en el que ambos sucesos ocurren en el mismo lugar pero en distintos instantes de tiempo.\todo{Uno de los sucesos esta fuera del cono de luz}
  \item Decimos que dos sucesos están separados espacialmente o que su intervalo es de género espacio cuando el cuadrado de su intervalo $s^{2}$ es positivo. Entonces existe un sistema de referencia inercial en el que ambos sucesos ocurren en el mismo instante pero en distintos lugares. \todo{Ambos sucesos están dentro del cono de luz}
  \item Decimos que el intervalo de dos sucesos es de género luz o nulo cuando su intervalo se anula. Entonces ambos sucesos están conectados mediante una señal luminosa.\todo{Uno de los sucesos están en el borde del cono de luz}
\end{itemize}

Es importante notar que esta clasificación de los intervalos en género tiempo, espacio o luz es independiente del sistema de referencia inercial elegido $y$, por tanto, es absoluta.

En cada instante de tiempo, llamaremos sistema de referencia propio de una partícula al sistema de referencia inercial cuya velocidad coincide en ese instante con la de la partícula, es decir, tal que $\vec{V}=\vec{v}$ y cuyo origen coincide con la posición de la partícula. El tiempo propio $\tau$ de una partícula es el tiempo medido por un reloj que se mueve con la partícula, es decir, el tiempo medido en el sistema de referencia propio. En términos del tiempo $t$ medido en otro sistema de referencia $S$ con respecto al cual el sistema de referencia propio $S^{\prime}$ se mueve con una velocidad instantánea $\vec{v}$, el tiempo propio $\tau:=t^{\prime}$ se puede obtener a partir de la ley de transformación de Lorentz:


\[\mathrm{d} \tau:=\mathrm{d} t^{\prime}=\gamma\left(1-v^{2} / c^{2}\right) \mathrm{d} t=\mathrm{d} t / \gamma
\]

Por otro lado, el tiempo propio y el intervalo espaciotemporal entre dos sucesos están íntimamente relacionados. En efecto,

\[\mathrm{d} s^{2} / c^{2}=-\mathrm{d} t^{2}\left(1-\frac{1}{c^{2}} \frac{\mathrm{~d} \vec{x}^{2}}{\mathrm{~d} t^{2}}\right)=-\mathrm{d} t^{2} / \gamma^{2}=\mathrm{d} \tau^{2}\]

\begin{corollary}
El tiempo propio es siempre menor que el tiempo medido en cualquier otro sistema de referencia inercial.
  
\end{corollary}







