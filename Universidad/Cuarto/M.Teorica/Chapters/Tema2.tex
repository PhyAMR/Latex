\setchapterpreamble[u]{\margintoc}
\chapter{Mecánica Hamiltoniana}
\labch{Part}


\section{Motivación de sistemas Hamiltonianos}

Uno podría imaginar, de forma más general, la construcción de una variedad diferente $\mathbb{F}$ añadiendo un tipo diferente de fibra a cada punto de $\mathbb{Q}$ (el resultado sería un haz de fibras sobre $\mathbb{Q}$ ). Si se pudiera hacer que $\mathbb{F}$ llevara de alguna manera la dinámica (para separar las trayectorias), y si las trayectorias sobre $\mathbb{F}$ pudieran proyectarse hasta las trayectorias físicas sobre $\mathbb{Q}$, entonces $\mathbb{F}$ serviría para nuestros propósitos tan bien como $\mathbf{TQ}$. En el formalismo hamiltoniano esto es precisamente lo que se hace: $\mathbb{F}$ es el haz cuyas fibras consisten en los posibles valores del momento $p$, una variedad cuyos puntos son de la forma ( $q, p$ ). Se denomina haz cotangente o colector (o espacio) de fase $\mathbf{T}^{*}. \mathbf{Q}$; sus fibras no son los espacios tangentes a los puntos de $\mathbb{Q}$, sino sus espacios duales (definidos en el apartado 3.4.1), llamados espacios cotangentes. Estas afirmaciones se aclararán a medida que avancemos.

\subsection{Ecuaciones de Hamilton}

A continuación vamos a derviar las ecuaciones de Hamilton a partir de lo que conocemos, es decir, el lagrangiano. Para ello vamos a hacer uso de la transformada de Legendre, que ya vimos en Termodinámica de segundo (Ver apéndice \ref{Legendre}). Lo que vamos a ver es análogo a lo del apéndice. 

El momento generalizado $p_{\alpha}$ conjugado con $q^{\alpha}$ ha sido definido (Sección 2.2.1) como
\begin{equation*}
  p_{\alpha}=\frac{\partial L}{\partial \dot{q}^{\alpha}} \tag{5.1}
\end{equation*}

donde $L: \mathbb{Q} \times \mathbb{R} \rightarrow \mathbb{R}$ es, como siempre, la función lagrangiana. (El dominio de $L$ está formado por $\mathbf{T}$ y el tiempo $t \in \mathbb{R}$). Por tanto, las ecuaciones de E-L y las ecuaciones diferenciales para $q^{\alpha}$ pueden escribirse de la forma

\begin{DispWithArrows}[format=ll, displaystyle]
  \frac{d p_{\alpha}}{d t} & =\frac{\partial L}{\partial q^{\alpha}}  \tag{5.2a}\\
  \frac{d q^{\alpha}}{d t} & =\dot{q}^{\alpha} \tag{5.2b}
\end{DispWithArrows}

\todo{Ahora no hay ecuaciones de segundo grado, solo $2n$ ecuaciones de primer grado}
Excepto por el hecho de que $L$ y $\dot{q}^{\alpha}$ son todas funciones de ( $q, \dot{q}, t$ ) ($\dot{q}^{\alpha}$ trivialmente) en lugar de $(q, p, t)$, estas ecuaciones tienen la forma simple deseada. Para obtener precisamente la forma deseada sólo se requiere que los lados derechos se escriban en términos de $(q, p, t)$.

En principio, esto es fácil de hacer para la Ec. (5.2b): escribir el $\dot{q}^{\alpha}$ en el lado derecho como funciones de $(q, p, t)$ invirtiendo (5.1). Supongamos que esto se puede hacer (vamos a encontrar las condiciones necesarias para ello más adelante), y llamar a las funciones resultantes $\dot{q}^{\alpha}(q, p, t)$. El lado derecho de la Ec. (5.2a) puede transformarse de forma similar: en todos los $\partial L / \partial q^{\beta}$ sustituir los $\dot{q}^{\alpha}$ dondequiera que aparezcan por $\dot{q}^{\alpha}(q, p, t)$. Sería más fácil hacer primero la sustitución en $L$, convirtiendo $L$ en una función de ( $q, p, t$ ), y luego tomar las derivadas. Esto requiere cuidado, sin embargo, porque la derivada parcial de $L(q, \dot{q}, t)$ con respecto a $q^{alpha}$ no es la misma que la de \todo{$\hat{L}$ es igual que $L$, pero en este caso $\dot{q}$ no es un parámetro libre, si no que es función de $(q,p,t)$} $\hat{L}(q, p, t) \equiv L(q, \dot{q}(q, p, t), t)$, ya que $L$ es una función de $(q, \dot{q}, t)$, y $\hat{L}$ es una función de $(q, p, t)$. Al tomar derivadas de $L$, las $\dot{q} s$ se mantienen fijas, mientras que en $\hat{L}$ las $p$ s se mantienen fijas.
fijas. Comparamos estos dos conjuntos de derivadas:

\[\frac{\partial \hat{L}}{\partial q^{\alpha}}=\frac{\partial L}{\partial q^{\alpha}}+\frac{\partial L}{\partial \dot{q}^{\beta}} \frac{\partial \dot{q}^{\beta}}{\partial q^{\alpha}}=\frac{\partial L}{\partial q^{\alpha}}+p_{\beta} \frac{\partial \dot{q}^{\beta}}{\partial q^{\alpha}}\]

donde hemos utilizado (5.1). El último término del lado derecho es la derivada de la función $\dot{q}^{\beta}(q, p, t)$. Ahora ponga todas las funciones de $(q, p, t)$ en un lado de la ecuación para obtener

\[\frac{\partial L}{\partial q^{\alpha}}=\frac{\partial}{\partial q^{\alpha}}\left[\hat{L}(q, p, t)-p_{\beta} \dot{q}^{\beta}(q, p, t)\right]\]

También será necesaria la derivada de $\hat{L}$ con respecto a $p_{\alpha}$. Esto es

\[\frac{\partial \hat{L}}{\partial p_{\alpha}}=\frac{\partial L}{\partial \dot{q}^{\beta}} \frac{\partial \dot{q}^{\beta}}{\partial p_{\alpha}}=p_{\beta} \frac{\partial \dot{q}^{\beta}}{\partial p_{\alpha}}\]

o (de nuevo poner todas las funciones de $(q, p, t)$ en un lado)

\begin{equation*}
  \frac{\partial}{\partial p_{\alpha}}\left[\hat{L}(q, p, t)-p_{\beta} \dot{q}^{\beta}(q, p, t)\right]=-\dot{q}^{\alpha} \tag{5.3}
  \end{equation*}

  Esta ecuación parece la solución que se obtendría para $\dot{q}^{text {ox }}$ invirtiendo (5.1), pero en realidad no lo es, ya que el lado izquierdo depende de $\dot{q}^{\beta}$. En su lugar es una ecuación diferencial para $\dot{q}^{\alpha}$ como una función de $(q, p, t)$.

  La función 
  
  \begin{equation*}
    H(q, p, t) \equiv p_{\beta} \dot{q}^{\beta}(q, p, t)-\hat{L}(q, p, t) \tag{5.4}
    \end{equation*}

aparece en las ecuaciones (5.3) y (5.4). Esta función tan importante se denomina función Hamiltoniana, o simplemente Hamiltoniana.


Con la ayuda del Hamiltoniano, las Ecs. (5.2a, b) pueden ahora escribirse enteramente en términos de $(q, p, t)$ :


\[\left.\begin{array}{l}
  \dot{q}^{\beta}=\frac{\partial H}{\partial p_{\beta}}  \tag{5.5}\\
  \dot{p}_{\beta}=-\frac{\partial H}{\partial q^{\beta}}
  \end{array}\right\}\]

Se denominan \textit{ecuaciones canónicas de Hamilton} \sidenote{El término «canónico» significa estándar o convencional (según los cánones). Se utiliza técnicamente para los sistemas dinámicos, en particular en el formalismo hamiltoniano.}. Son las ecuaciones del movimiento en el formalismo hamiltoniano. Su solución da expresiones locales para las trayectorias
$(q(t), p(t))$ en $\mathbf{T}^{*} \mathbb{Q}$, la multiplicidad $(q, p)$. Esta variedad se denomina haz cotangente (que se discutirá en la Sección 5.2). Las proyecciones de estas trayectorias hacia $\mathbb{Q}$ (esencialmente ignorando la parte $p(t)$) son las trayectorias del colector de configuración $q(t)$. Dejamos para la Sección 5.2 una discusión más detallada de $\mathbf{T}^{*} \mathbb{Q}$ y una explicación de la diferencia entre éste y $\mathbf{T Q}$.


\subsubsection{Energía}

Es de destacar que el Hamiltoniano es la variable dinámica que hemos llamado $E$, escrito ahora en términos de $(q, p)$ en lugar de $(q, \dot{q})$. Recordemos que para muchos sistemas dinámicos $E$ es la energía, y por tanto cuando $L=T-V$ el Hamiltoniano es $H=T+V$ expresado en términos de $(q, p)$. En los casos comunes en que $V$ es independiente de la $\dot{q}^{\beta}$ sólo es necesario escribir la energía cinética $T$ en términos de $(q, p)$. Sin embargo, no siempre es tan sencillo, como se verá en el ejemplo práctico 5.1 del Hamiltoniano para una partícula cargada en un campo electromagnético.

En el formalismo lagrangiano teniamos la cantidad conservada, asociada en algunos casos a la energía, que deonotabamos como $h(q,\dot{q},t)$ con la forma 
\[h(q,\dot{q},t)=p_{\alpha}\dot{q}^{\alpha}-L\]
En el formalismo Hamiltoniano, esta cantidad coincide con el Hamiltoniano en si, es decir,

\[h(q,\dot{q},t)=p_{\alpha}\dot{q}^{\alpha}-L=H \text{ con }\dot{q}^{\alpha}(q,p)\]

\begin{proof}
  Esto se puede demostrar sabiendo que $\pdv{H}{t}=-\pdv{L}{t}$ como 
  \[\dv{H}{t}=\overbrace{\pdv{H}{q^{a}}\pdv{q^{a}}}^{-p_{a}\dot{q}^{a}} + \overbrace{\pdv{H}{p_{a}}\pdv{}{}}^{\dot{q}^{a}p_{a}}+\pdv{H}{t}\Rightarrow \dv{H}{t}=\pdv{H}{t}=0 \]
\end{proof}



\subsubsection{Receta Para construir el Hamiltoniano}

La receta para escribir las ecuaciones de Hamilton para un sistema cuya Lagrangiana es conocida es entonces la siguiente\sidenote{Dentro de cada apartado pondré como explico Prado cada paso para un \textbf{caso de interés} que ya desarrolllamos en su momento es cuando el Lagrangiano tiene la forma \\
\[L=\underbrace{\frac{1}{2}T_{ab}(q,t) \dot{q}^{a}\dot{q}^{b}}_{L_{2}}+\underbrace{A_{a}(q, t)\dot{q}^{a}}_{L_{1}}-\underbrace{V(q,t)}_{L_{0}}\],\\ es decir
$L=L_{2}+L_{1}+L_{0}$}:
\begin{itemize}
  \item Utilizar (5.1) para calcular $p_{\alpha}$ en términos de $(q, \dot{q}, t)$.
  \item Invertir estas ecuaciones para obtener el $\dot{q}^{\alpha}$ en términos de $(q, p, t)$.
  \item Insertar el $\dot{q}^{{\alpha}}(q, p, t)$ en la expresión de $L$ para obtener $\hat{L}$.
  \item Utilizar (5.4) para obtener una expresión explícita para $H(q, p, t)$.
  \item Tomando las derivadas de $H$, escribir (5.5).
\end{itemize}

La receta funciona sólo si el paso $i i$ es posible, es decir, sólo si la Ecs. (5.1) se puede resolver de forma única para el $\dot{q}^{\alpha}$. De acuerdo con el teorema de la función implícita una condición necesaria y suficiente para la invertibilidad localmente es que el hessiano $\partial^{2} L / \partial \dot{q}^{\alpha} \partial \dot{q}^{\beta}$ sea no singular. Nos encontramos con el mismo requisito en la sección 2.2.2. Como regla general supondremos que se cumple, aunque habrá ocasiones en que falle en subconjuntos pequeños.

Las ecuaciones canónicas de Hamilton se han derivado de las ecuaciones de Euler-Lagrange, que a su vez se han derivado de las ecuaciones de movimiento de Newton. Hasta ahora el único método que tenemos para hallar la función hamiltoniana es pasando por la lagrangiana, pero a menudo es tan sencillo desde cero escribir una como la otra. Más adelante hablaremos de ello.



\begin{example}[Dado el lagrangiano: $L=\frac{1}{2} m\left[\dot{r}^{2}+r^{2} \dot{\theta}^{2}+\left(r^{2} \sin ^{2} \theta\right) \dot{\phi}^{2}\right]-V(r) $ \\
  
  (a) Obtenga el Hamiltoniano y (b) escriba las ecuaciones canónicas de Hamilton para una partícula cargada en un campo electromagnético. Proyectarlas hasta $\mathbb{Q}$ y demostrar que son las ecuaciones usuales para una partícula en un campo electromagnético.]

  Según la Ecuación (5.1): 
  \begin{DispWithArrows}[displaystyle, format=c] 
    p_{r}=m \dot{q}^{\gamma}+e A_{\gamma} \tag{5.6} 
  \end{DispWithArrows} 
  que es el momento generalizado canónicamente conjugado a $q^{\gamma}$. El momento generalizado no es el momento dinámico $m \dot{q}^{\gamma}$, sino que tiene una contribución que proviene del campo electromagnético.

  Para obtener el hamiltoniano, resuelve esta ecuación para $\dot{q}^{\gamma}$ e insértalo en la Ecuación (5.4). El resultado es (usamos $\delta^{\alpha \beta}$ para forzar que las sumas sean sobre pares superíndice-subíndice): \begin{DispWithArrows}[displaystyle, format=c] 
    H(q, p, t)=\frac{1}{2 m} \delta^{\alpha \gamma}\left\{p_{\alpha}-e A_{\alpha}(q, t)\right\}\left\{p_{\gamma}-e A_{\gamma}(q, t)\right\}+e \varphi(q, t) \tag{5.7} 
  \end{DispWithArrows}
  
  (b) Las ecuaciones canónicas obtenidas de este hamiltoniano son: 
  
  \begin{DispWithArrows}[format=c, displaystyle]
    \dot{q}^{\beta}=\frac{1}{m} \delta^{\alpha \beta}\left\{p_{\alpha}-e A_{\alpha}(q, t)\right\}, \tag{5.8a} \\
     \dot{p}{\beta}=\frac{e}{m} \delta^{\alpha \gamma} \frac{\partial A{\alpha}}{\partial q^{\beta}}\left\{p_{\gamma}-e A_{\gamma}(q, t)\right\}-\frac{\partial \varphi}{\partial q^{\beta}}. \tag{5.8b}
  \end{DispWithArrows}
  
  Las ecuaciones de movimiento en $\mathbb{Q}$ se obtienen eliminando $p_{\beta}$ de las Ecuaciones (5.8a, b). Esto se hace tomando las derivadas temporales de $\dot{q}^{\alpha}$ en (5.8a) y reemplazando los $\dot{p}{\alpha}$ que aparecen por sus expresiones en (5.8b). La Ecuación (5.8a) da como resultado: \begin{DispWithArrows}[displaystyle, format=c] \ddot{q}^{\beta}=\frac{1}{m} \delta^{\alpha \beta}\left(\dot{p}{\alpha}-e \partial_{y} A_{\alpha} \dot{q}^{\gamma}-e \partial_{y} A_{\alpha}\right) \end{DispWithArrows}
  
  donde $\partial_{\gamma}=\partial / \partial q^{\gamma}$ y $\partial_{t}=\partial / \partial t$. Ahora, se usa la Ecuación (5.8b) para eliminar $\dot{p}{\alpha}$, pero el $p_{y}$ en el lado derecho debe ser reemplazado por su expresión en términos de $(q, \dot{q})$, es decir, por la Ecuación (5.6). Entonces, la ecuación para $\ddot{q}^{\beta}$ se convierte en (aquí ya no nos preocupamos de si los índices son superíndices o subíndices): 
  \begin{DispWithArrows}[displaystyle, format=c] 
    m \ddot{q}^{\beta}=e\left\{\left(\partial_{\beta} A_{\gamma}-\partial_{\gamma} A_{\beta}\right) \dot{q}^{\gamma}-\partial_{t} A_{\beta}-\partial_{\beta} \varphi\right\} \tag{5.9} 
  \end{DispWithArrows}
  
  Se deduce de la Sección 2.2.4 que $\partial_{\beta} A_{\gamma}-\partial_{y} A_{\beta}=\epsilon_{\beta \gamma \alpha} B_{\alpha}$ y $-\partial_{\psi} A_{\alpha}-\partial_{\alpha} \varphi=E_{\alpha}$, donde $\mathbf{B}=\left(B_{1}, B_{2}, B_{3}\right)$ es el campo magnético y $\mathbf{E}=\left(E_{1}, E_{2}, E_{3}\right)$ es el campo eléctrico (no debe confundirse con la energía). Así que: 
  \begin{DispWithArrows}[displaystyle, format=c] 
    m \ddot{q}^{\beta}=e\left\{\epsilon_{\gamma \alpha \beta} \dot{q}^{\gamma} B_{\alpha}+E_{\beta}\right\} \end{DispWithArrows}
  
  lo cual es la forma en componentes de la fuerza de Lorentz $\mathbf{F} \equiv m \mathbf{a}=e(\mathbf{v} \wedge \mathbf{B}+\mathbf{E})$, que es el resultado requerido.
  
\end{example}


\begin{example}[Dado el lagrangiano: $L(q, \dot{q}, t)=\frac{1}{2} m \dot{q}^{\beta} \dot{q}^{\beta}-e \varphi(q, t)+e \dot{q}^{\beta} A_{\beta}(q, t)$ \\ (a) Obtén el Hamiltoniano y las ecuaciones canónicas para una partícula en un campo de fuerza central. (b) Toma dos de las condiciones iniciales como \(p_{\phi}(0)=0\) y \(\phi(0)=0\) (esto es esencialmente la elección de un sistema de coordenadas esféricas particular). Discute la simplificación resultante de las ecuaciones canónicas. ]

  Denota los momentos conjugados por \(p_{r}, p_{\theta}\) y \(p_{\phi}\). Las ecuaciones definitorias para ellos son
\begin{DispWithArrows}[format=ll, displaystyle]
  p_{r} & \equiv \frac{\partial L}{\partial \dot{r}}=m \dot{r}  \tag{5.11}\\
  p_{\theta} & \equiv \frac{\partial L}{\partial \dot{\theta}}=m r^{2} \dot{\theta} \\
  p_{\phi} & \equiv \frac{\partial L}{\partial \dot{\phi}}=m\left(r^{2} \sin ^{2} \theta\right) \dot{\phi}
\end{DispWithArrows}

Invertir estas ecuaciones da
\begin{DispWithArrows}[format=c, displaystyle]
  \dot{r}=\frac{p_{r}}{m}  \tag{5.12}\\
  \dot{\theta}=\frac{p_{\theta}}{m r^{2}} \\
  \dot{\phi}=\frac{p_{\phi}}{m r^{2} \sin ^{2} \theta}
\end{DispWithArrows}

El Hamiltoniano es
\begin{DispWithArrows}[format=c, displaystyle]
  H=p_{\alpha} \dot{q}^{\alpha}-L=\frac{p_{r}^{2}}{2 m}+\frac{p_{\theta}^{2}}{2 m r^{2}}+\frac{p_{\phi}^{2}}{2 m r^{2} \sin ^{2} \theta}+V(r) \equiv T+V \tag{5.13}
\end{DispWithArrows}
donde la energía cinética $T$ es la suma de los tres primeros términos. Las ecuaciones canónicas de Hamilton en estas variables son
\begin{DispWithArrows}[format=ll, displaystyle]
  \dot{r} \equiv \frac{\partial H}{\partial p_{r}}=\frac{p_{r}}{m}, & \dot{p}_{r} \equiv-\frac{\partial H}{\partial r}=\frac{1}{m r^{3}}\left[p_{\theta}^{2}+\frac{p_{\phi}^{2}}{\sin ^{2} \theta}\right]-V^{\prime}(r) \\
  \dot{\theta} \equiv \frac{\partial H}{\partial p_{\theta}}=\frac{p_{\theta}}{m r^{2}}, & \dot{p}_{\theta} \equiv-\frac{\partial H}{\partial \theta}=\frac{p_{\phi}^{2} \cos \theta}{m r^{2} \sin ^{3} \theta}  \tag{5.14}\\
  \dot{\phi} \equiv \frac{\partial H}{\partial p_{\phi}}=\frac{p_{\phi}}{m r^{2} \sin ^{2} \theta}, & \dot{p}_{\phi} \equiv-\frac{\partial H}{\partial \phi}=0
\end{DispWithArrows}

Comentario: Nota que las tres ecuaciones a la izquierda siguen inmediatamente de las definiciones de los momentos generalizados. Esto siempre ocurre con las ecuaciones canónicas de Hamilton: al estudiar su derivación queda claro que la mitad de ellas son simplemente inversiones de esas definiciones.

(b) Si $p_{\phi}(0)=0$, la última de las Eqs. (5.14) implica que $p_{\phi}(t)=0$ para todo $t$. Entonces las ecuaciones canónicas se simplifican.

De $\phi(0)=0$ y estas ecuaciones se sigue que $\phi(t)=0$ para todo $t$: el movimiento se mantiene en el plano $\phi=0$. El resultado es un sistema de dos grados de libertad que consiste en las dos primeras líneas de (5.15). La cuarta ecuación implica que $p_{\theta}$ es una constante del movimiento. Esta es el momento angular en el plano $\phi=0$, llamado $l$.
  
\end{example}

\subsection{
Coordenadas unificadas en $T* \mathbb{Q}^{*}$ y corchetes de Poisson}

Uno de los principales beneficios del formalismo hamiltoniano es que las variables $q$ y $p$ se tratan a la misma manera. Para hacer esto de manera consistente, los unimos en un conjunto de $2 n$ variables que serán llamadas $\xi^{\prime}$, con el índice $j$ corriendo desde 1 hasta $2 n$ (índices latinos van desde 1 hasta $2 n$ y índices griegos continúan corriendo desde 1 hasta $n$). El primer $n$ de las $\xi^{j}$ son las $q^{\beta}$, y el segundo $n$ son las $p_{\beta}$. De manera más explícita,

$$
\begin{array}{ll}
\xi^{j}=q^{j}, & j \in\{1, \ldots, n\} \\
\xi^{j}=p_{j-n}, & j \in\{n+1, \ldots, 2 n\}
\end{array}
$$
\todo{De estas dos ecuaciones, se deduce que $\xi^{j}$ es un conjunto de $2 n$ variables, donde $q^{\beta}$ y $p_{\beta}$ son las primeras $n$ y las últimas $n$ variables, respectivamente.}

Si ahora queremos escribir las ecuaciones canónicas de Hamilton en la forma unificada $\dot{\xi}^{\prime}=f^{\prime}(\xi)$, deben encontrarse los $f^{j}$ funciones a partir de la derecha de (5.5): los primeros $n$ del $f^{\prime}$ son las $\partial H / \partial p_{\jmath} \equiv \partial H / \partial \xi^{\prime+n}$, y los últimos $n$ son $-\partial H / \partial q^{J} \equiv-\partial H / \partial \xi^{j}$. Así que las ecuaciones canónicas son

Esta puede escribirse en una de las dos formas equivalentes

$$
\begin{array}{ll}
\dot{\xi}^{\prime}=\omega^{j k} \partial_{k} H \equiv \Delta_{H}\tag{5.42a}\\
\omega_{l,} \dot{\xi}^{\prime}=\partial_{l} H \tag{5.42b}
\end{array}
$$

Aquí $\partial_{k} \equiv \partial / \partial \xi^{k}$.

La ecuación (5.42a) introduce la matriz simétrica $2 n \times 2 n$ $\Omega$, con elementos de matriz $\omega^{\prime h}$ dados por
$$
\Omega=\left|\begin{array}{cc}
0_{n} & -\mathbb{I}_{n}  \tag{5.43}\\
\mathbb{I}_{n} & 0_{n}
\end{array}\right|
$$
donde $\mathbb{I}_{n}$ y $0_{n}$ son las matrices unitarias y de ceros $n \times n$, respectivamente. Dos propiedades importantes de $\Omega$ son que $\Omega^{-1}=-\Omega$ (usamos índices inferiores para los elementos de la matriz de $\Omega^{-1}$, escribiendo $\omega_{j l}$ ) y que es antisimétrica. En otras palabras,
$$
\begin{array}{l}
\Omega^{2}=-\mathbb{I}, \quad \Omega^{\mathrm{T}}=-\Omega \tag{5.44}\\
\omega^{\prime k} \omega_{k l}=\delta_{l}^{\prime}, \quad \omega_{k l} \omega_{l j} \delta^{l l}=-\delta_{k l}, \quad \omega^{k_{l}} \omega^{l /} \delta_{l l}=-\delta^{k j}
\end{array}
$$



\subsection{Corchetes de Poisson}

El término involucrado en $\omega^{\prime k}$ en el lado derecho de (5.46) es importante; se le llama el cuadrado Poisson de $f$ con $H$. En general, si $f, g \in \mathcal{F}\left(\mathbf{T}^{*} \mathbb{Q}\right)$, el cuadrado Poisson de $f$ con $g$\todo{$f, g$ son variables dinámicas} se define como

$$
\begin{aligned}
\{f, g\} & \equiv\left(\partial_{j} f\right) \omega^{\prime k}\left(\partial_{k} g\right) \equiv \frac{\partial f}{\partial \xi^{\prime}} \omega^{\prime k} \frac{\partial g}{\partial \xi^{k}} \\
& \equiv \frac{\partial f}{\partial q^{\alpha}} \frac{\partial g}{\partial p_{\alpha}}-\frac{\partial f}{\partial p_{\alpha}} \frac{\partial g}{\partial q^{\alpha}} \tag{5.47}
\end{aligned}
$$

si cambiamos la $f$ y la $g$ por $\xi^{j}$ y $\xi^{i}$, respectivamente.
\begin{DispWithArrows}[format=c, displaystyle]
\pb{\xi^{j}}{\xi^{i}}=\partial_{\gamma }\xi^{j} \omega^{\gamma  \delta } \partial_{\delta } \xi^{ i} = \delta_{\gamma}^{i}\omega\omega^{\gamma  \delta }\delta_{\delta}^{j} 
\end{DispWithArrows}

Si cambiamos f y g por $\xi^{i}$ y $\xi^{j}$, respectivamente, entonces
\begin{DispWithArrows}[format=ll, displaystyle]
\pb{\xi^{i}}{\xi^{j}}=& \partial_{\gamma }\xi^{i} \omega^{\gamma  \delta } \partial_{\delta } \xi^{ j} \\
=& \delta_{\gamma}^{j}\omega^{\gamma  \delta }\delta_{\delta}^{i}=\omega^{ij} 
\end{DispWithArrows}

Los corchetes de Poisson (CP) satisfacen una regla de producto de tipo Leibniz:

\begin{DispWithArrows}[displaystyle, format=c]
\{f, g h\}=g\{f, h\}+\{f, g\} h \tag{5.48}
\end{DispWithArrows}


\begin{remark}
  Bilinearidad
  \begin{DispWithArrows}[format=c, displaystyle]
  \pb{af+bg}{h}=a\{f, h\}+b\{g, h\} \notag
  \end{DispWithArrows}, 
  antisimetría
  \begin{DispWithArrows}[format=c, displaystyle]
  \pb{f}{g}=-\pb{g}{f} \notag
  \end{DispWithArrows} y la identidad Jacobi 
  \begin{DispWithArrows}[format=c, displaystyle]
  \pb{f}{\pb{g}{h}}+\pb{g}{\pb{h}{f}}+\pb{h}{\pb{f}{g}}=0 \notag
  \end{DispWithArrows}
  son las propiedades definidoras de una estructura algebraica importante llamada algebras Lie. El espacio funcional $\mathcal{F}\left(\mathbf{T}^{*} \mathrm{Q}\right)$ es por tanto una algebras Lie bajo el CP, y la dinámica hamiltoniana se estudia con éxito desde el punto de vista de las algebras Lie. El punto de vista algebraico Lie juega un papel importante también en la transición a mecánica cuántica.
\end{remark}

En términos del CP Eq. (5.46) se convierte
$$
\mathbf{L}_{\Delta} f \equiv \frac{d f}{d t}=\{f, H\}+\partial_{t} f
$$

La ecuación (5.49) se satisface por todas las variables dinámicas, incluyendo las coordenadas mismas. Aplicado a las coordenadas locales, se convierte en

\begin{DispWithArrows}[displaystyle, format=c]
\Delta_{H}^{j}=\dot{\xi}^{j}=\left\{\xi^{j}, H\right\} \left\{\begin{array}{c}
  \dot{q}^{a}=\pb{q^{a}}{H} \\
  \dot{o}_{a}=\pb{p_{a}}{H} \\
\end{array}\right.
    \tag{5.50}
\end{DispWithArrows}

que es solo otra forma de escribir las ecuaciones hamiltonianas canónicas. Otra aplicación de Eq. (5.49) es a la función hamiltoniana. Si hacemos la derivada de una función $f$ cualquiera obtenemos 
\begin{DispWithArrows}[format=c, displaystyle]
  \dv{f}{t}=\dot{f}=\partial_{i}f\dot{\xi}^{i}=\partial_{i}f\omega^{ij}\partial_{j}H+\partial_{t} f=\{f, H\}+\partial_{t} f\equiv \mathbf{L}_{\Delta_{H}}
\end{DispWithArrows}


La antisimetría del CP implica que $\{H, H\}=0$, y luego

\begin{DispWithArrows}[displaystyle, format=c]
\dot{H}=\{H, H\}+\partial_{t} H=\partial_{t} H \tag{5.51}
\end{DispWithArrows}

es decir, la derivada temporal de la función hamiltoniana es igual a su derivada parcial temporal. Esto refleja la conservación de energía para lagrangianos invarientes en el tiempo, ya que se puede demostrar desde la definición original de la función hamiltoniana que $\partial_{t} H=-\partial_{t} L$.

Los paréntesis de Poisson (PB) de funciones tomadas con las coordenadas jugarán un papel especial en gran parte de lo que sigue, así que los presentamos aquí. Si \( f \in \mathcal{F}\left(\mathbf{T}^{*} \mathbb{Q}\right) \), entonces se tienen las siguientes relaciones:

\[
\Delta_{f}^{\alpha }= \underbrace{\partial_{\gamma }\xi^{j}}_{\delta _{\gamma }^{j}}\omega^{jk}\partial_{k}f\left\{\xi^{j}, f\right\} = \omega^{j k} \partial_{k} f, \text{ o } 
\]

\[
\left\{p_{\alpha}, f\right\} = \frac{\partial f}{\partial q^{\alpha}}, \quad \left\{q^{\alpha}, f\right\} = -\frac{\partial f}{\partial p_{\alpha},} 
\]

\[
\left\{\xi^{j}, \xi^{k}\right\} = \omega^{j k}, \text{ o }
\]

\[
\left\{p_{\beta}, q^{\alpha}\right\} = \delta_{\beta}^{\alpha}, \quad \left\{q^{\alpha}, q^{\beta}\right\} = \left\{p_{\alpha}, p_{\beta}\right\} = 0.
\]



\begin{example}[(a) Encuentra las correciones Poisson de las componentes $j_\alpha$ del momento angular $\mathbf{j}$ con variables dinámicas (usamos $\mathbf{j}$ en lugar de $\mathbf{L}$ para evitar la confusión con el Lagrangiano y la derivada de Lie). \\  (b) Encuentra las PBs de las componentes $j_\alpha$ con la posición y momentum en espacio euclidiano 3. \\
  (c) La única forma de formar escalar es mediante productos de dot. Encuentra las correciones Poisson de las $j_\alpha$ con los momientos y posiciones. \\
  (d) La única forma de formar vectores es multiplicando $\mathbf{x}$ y $\mathbf{p}$ por escalar y tomando productos cruzados. Encuentra las PBs de las $j_\alpha$ con vectores. \\
  Especializa en los CP  de las $j_\alpha$ entre sí.]

  Configuración del espacio de la cuadrícula $\mathbb{Q}$ es el espacio vectorial $\mathbb{E}^{3}$ con coordenadas $\left(x^{1}, x^{2}, x^{3}\right)$. El anticommutador entre componentes de la momenta angular no es cero, sino que es igual a la componente $x^{3}$ del producto escalar. Por lo tanto, para cualquier vector escalar $\mathbf{w} = f \mathbf{x} + g \mathbf{p} + h \mathbf{j}$,

  El espacio de configuración $\mathbb{Q}$ es el espacio vectorial $\mathbb{E}^{3}$ con coordenadas $\left(x^{1}, x^{2}, x^{3}\right) \equiv \mathbf{x}$. El momento angular $\mathbf{j}$ de una sola partícula se define como $\mathbf{j}=\mathbf{x} \wedge \mathbf{p}$. Aquí $\mathbf{p}$ es el momento, que se encuentra en otro $\mathbb{E}^{3}$. El momento angular $\mathbf{j}$, en un tercer $\mathbb{E}^{3}$, tiene componentes
  \begin{DispWithArrows}[displaystyle, format=c]
  j_{\alpha}=\epsilon_{\alpha \beta \gamma} x^{\beta} p_{\gamma} \tag{5.53}
  \end{DispWithArrows}
  (Como hacemos a menudo al usar coordenadas cartesianas, violamos las convenciones sobre índices superiores e inferiores).
  (a) De la segunda línea de la Ecuación (5.52) tenemos
  \begin{aligned}
  \left\{j_{\alpha}, f\right\} & =\epsilon_{\alpha \beta Y}\left\{x^{\beta} p_{Y}, f\right\}=\epsilon_{\alpha \beta_{Y}} x^{\beta}\left\{p_{\gamma}, f\right\}+\epsilon_{\alpha \beta \gamma}\left\{x^{\beta}, f\right\} p_{\gamma} \\
  & =\epsilon_{\alpha \beta Y}\left[-x^{\beta} \partial f / \partial x^{\gamma}+p_{\gamma} \partial f / \partial p_{\beta}\right] .
  \end{aligned}
  
  (b) Usa el resultado anterior con $f \equiv x^{\lambda}$ y $f=p_{\lambda}$ para obtener
  \begin{aligned}
  & \left\{j_{\alpha}, x^{\lambda}\right\}=-\epsilon_{\alpha \beta \gamma} x^{\beta} \delta_{\gamma}^{\lambda}=\epsilon_{\alpha \lambda \beta} x^{\beta} \\
  & \left\{j_{\alpha}, p_{\lambda}\right\}=\epsilon_{\alpha \beta \beta \gamma} p_{\gamma} \delta_{\lambda}^{\beta}=\epsilon_{\alpha \lambda \gamma} p_{\gamma}
  \end{aligned}
  
  (c) Usa la regla de Leibniz, Ecuación (5.48), para obtener
  \begin{aligned}
  & \left\{j_{\alpha}, x^{\lambda} x^{\lambda}\right\}=2 x^{\lambda} \epsilon_{\alpha \lambda \beta} x^{\beta}=0 \\
  & \left\{j_{\alpha}, p_{\lambda} p_{\lambda}\right\}=2 p_{\lambda} \epsilon_{\alpha \lambda \gamma} p_{Y}=0 \\
  & \left\{j_{\alpha}, x^{\lambda} p_{\lambda}\right\}=\epsilon_{\alpha \lambda \gamma} x^{\lambda} p_{\gamma}+\epsilon_{\alpha \lambda \beta} x^{\beta} p_{\lambda}=0
  \end{aligned}
  
  (d) Los únicos vectores son productos de funciones escalares con $\mathbf{x}, \mathbf{p}$ y $\mathbf{j}=\mathbf{x} \wedge \mathbf{p}$, es decir, son de la forma
  $$
  \mathbf{w}=f \mathbf{x}+g \mathbf{p}+h \mathbf{j}
  $$
  donde $f, g$ y $h$ son escalares. Por la regla de Leibniz y la Parte (a)
  \begin{aligned}
  \left\{j_{\alpha}, f x^{\lambda}\right\} & =f \epsilon_{\alpha \lambda} x^{\beta} \\
  \left\{j_{\alpha}, g p_{\lambda}\right\} & =g \epsilon_{\alpha \lambda \gamma} p_{\gamma}
  \end{aligned}
  
  Solo necesitan calcularse los corchetes de Poisson de los componentes de $\mathbf{j}$ entre sí:
  \begin{aligned}
  \left\{j_{\alpha}, j_{\beta}\right\} & =\epsilon_{\alpha \mu \nu}\left\{x^{\mu} p_{\nu}, j_{\beta}\right\}=\epsilon_{\alpha \mu \nu} x^{\mu}\left\{p_{\nu}, j_{\beta}\right\}+\epsilon_{\alpha \mu \nu}\left\{x^{\mu}, j_{\beta}\right\} p_{v} \\
  & =\epsilon_{\alpha \mu \nu} \epsilon_{\nu \beta \gamma} x^{\mu} p_{\gamma}+\epsilon_{\alpha \mu \nu} \epsilon_{\mu \beta \gamma} x^{\gamma} p_{\nu} \\
  & =\left(\epsilon_{\alpha \mu \nu} \epsilon_{\nu \beta \gamma}+\epsilon_{\alpha \nu \gamma} \epsilon_{\nu \beta \mu}\right) x^{\mu} p_{\gamma} \\
  & =\left[\left(\delta_{\alpha \beta} \delta_{\mu \nu}-\delta_{\alpha \gamma} \delta_{\mu \beta}\right)+\left(\delta_{\gamma \beta} \delta_{\mu \alpha}-\delta_{\gamma \mu} \delta_{\alpha \beta}\right)\right] x^{\mu} p_{\gamma} \\
  & =\left(\delta_{\gamma \beta} \delta_{\alpha \mu}-\delta_{\alpha \gamma} \delta_{\mu \beta}\right) x^{\mu} p_{\gamma}=\epsilon_{\alpha \beta \nu} \epsilon_{\nu \mu \gamma} x^{\mu} p_{\gamma}
  \end{aligned}
  
  o
  \begin{DispWithArrows}[displaystyle, format=c]
  \left\{j_{\alpha}, j_{\beta}\right\}=\epsilon_{\alpha \beta \nu} j_{\nu} \tag{5.54}
  \end{DispWithArrows}
  
  Dado que $h$ es un escalar, la regla de Leibniz implica que $\left\{j_{\alpha}, h j_{\beta}\right\}=h \epsilon_{\alpha \beta v} j_{v}$. Así, para cualquier vector $\mathbf{w}$,
  $$
  \left\{j_{\alpha}, w_{\beta}\right\}=\epsilon_{\alpha \beta v} w_{v}
  $$
  
  Resumiendo, tenemos
  \begin{aligned}
  & \left\{j_{1}, w_{1}\right\}=\left\{j_{2}, w_{2}\right\}=\left\{j_{3}, w_{3}\right\}=0 \\
  & \left\{j_{1}, w_{2}\right\}=w_{3},\left\{j_{2}, w_{3}\right\}=w_{1}, \quad \text { y permutaciones cíclicas }
  \end{aligned}
  
  En particular,
  \begin{aligned}
  & \left\{j_{1}, j_{2}\right\}=j_{3} \quad \text { y permutaciones cíclicas } \\
  & \left\{j_{1}, j_{1}\right\}=\left\{j_{2}, j_{2}\right\}=\left\{j_{3}, j_{3}\right\}=0
  \end{aligned}
  
  Los componentes cartesianos del momento angular no conmutan entre sí (el término se toma de álgebra matricial). Veremos más adelante que esto tiene consecuencias importantes.
  
\end{example}

\subsubsection{Oscilador armónico istrópico}

Consideremos el oscilador armónico isotrópico en \( n \) grados de libertad (para simplificar, tomemos tanto la masa \( m = 1 \) como la constante del resorte \( k = 1 \)). El lagrangiano es 

\[
L = \frac{1}{2} \delta_{\alpha \beta} \left( \dot{q}^{\alpha} \dot{q}^{\beta} - q^{\alpha} q^{\beta} \right)
\]
\todo{Al pasar a la nueva notación hemos perdido como se transformar los momentos (forma covariante), se verá como se hace}

lo cual lleva a

\[
H = \frac{1}{2} (\delta^{\alpha \beta }p_{\alpha }p_{\beta }+\delta_{\alpha \beta }q^{\alpha \beta })= \frac{1}{2} \delta_{y k} \xi^{y} \xi^{k}
\]

Luego, a partir de (5.50) con bilinealidad y Leibniz, se sigue que

\[
\dot{\xi}^t = \frac{1}{2} \delta_{j k} \left\{ \xi^t, \xi^j \xi^k \right\} = \delta_{1 k} \omega^l \xi^k \tag{5.55}
\]

Una forma de resolver estas ecuaciones es derivarlas con respecto al tiempo y usar la propiedad de que \( \Omega^{2} = -\mathbb{I} \), obteniendo \( \ddot{\xi}^{k} = -\xi^{k} \). Estas \( 2n \) ecuaciones de segundo orden dan lugar a \( 4n \) constantes del movimiento que están conectadas por las ecuaciones de primer orden (5.55), de modo que solo \( 2n \) de ellas son independientes. Una manera diferente, y quizás más interesante, de resolver estas ecuaciones es escribirlas en la forma

\[
\dot{\vec{\xi}} = \Lambda \vec{\xi} \tag{5.56}
\]

donde \( \xi \) es el vector de \( 2n \) dimensiones con componentes \( \xi^{k} \) y \( \Lambda =\Omega^{-1}\mathrm{i}\) es la matriz con elementos \( \lambda_{k}^{y} = \delta_{j k} \omega^{t} \). La solución de la ecuación (5.56) es

\[
\vec{}{\xi}(t) = e^{\Lambda t} \vec{\xi_{0}} \tag{5.57}
\]

donde \( \xi_{0} \) es un vector constante cuyas componentes son las condiciones iniciales, y

\[
e^{\Lambda t} \equiv \sum_{0}^{\infty} \frac{(\Lambda t)^n}{n!} \tag{5.58}
\]

Calcular \( e^{\Lambda t} \) se simplifica por el hecho de que \( \Lambda^{2} = -\mathbb{I} \); de hecho,

\[
\left( \Lambda^{2} \right)_{l}^{l} \equiv \lambda_{k}^{l} \lambda_{l}^{k} = \omega^{l} \delta_{l k} \omega^{k r} \delta_{r l} = -\delta^{r} \delta_{r l} = -\delta_{l}^{l}
\]

Esto se puede usar para escribir todas las potencias de \( \Lambda \):

\[
\Lambda^{3} = -\Lambda, \quad \Lambda^{4} = \mathbb{I}, \quad \Lambda^{5} = \Lambda, \quad \Lambda^{6} = -\mathbb{I}, \ldots
\]

Luego, al expandir la suma en (5.58) se obtiene

\[
e^{\Lambda t} = \mathbb{I} \left( 1 - \frac{t^2}{2!} + \frac{t^4}{4!} + \cdots \right) + \Lambda \left( t - \frac{t^3}{3!} + \frac{t^5}{5!} + \cdots \right)
= \mathbb{I} \cos t + \Lambda \sin t
\]

(comparar con la expansión similar para \( e^{\prime t} \)). La solución de las ecuaciones de movimiento es entonces

\[
\xi(t) = \xi_{0} \cos t + \Lambda \xi_{0} \sin t \tag{5.59a}
\]

o

\[
\xi^{k}(t) = \xi_{0}^{k} \cos t + \lambda^{k}, \xi_{0}^{\prime} \sin t \tag{5.59~b}
\]

Las ecuaciones (5.59a, 5.59b) son solo una forma diferente de escribir (5.57). Debido a que solo se usaron ecuaciones de primer orden, solo las \( 2n \) constantes del movimiento \( \xi_{0}^{k} \) aparecen en estas soluciones. Finalmente, para completar el ejemplo, escribimos la forma \((q, p)\) de la solución:

\[
\begin{aligned}
q^{\alpha}(t) &= q_{0}^{\alpha} \cos t + p_{0 \alpha} \sin t \\
p_{\alpha}(t) &= -q_{0}^{\alpha} \sin t + p_{0 \alpha} \cos t
\end{aligned}
\]

\subsection{Corchetes de Poisson y dinámica Hamiltoniana}

Los corchetes de Poisson juegan un papel especial cuando el movimiento se deriva de las ecuaciones canónicas de Hamilton. No todo movimiento concebible en \(\mathbf{T}^{*} \mathbb{Q}\) es un sistema dinámico hamiltoniano, definido de la manera canónica por (5.42a) o (5.50). Cualquier ecuación de la forma

\[
\dot{\xi}^{j} = X^{\prime}(\xi) \tag{5.60}
\]

con \( X^{\prime} \in \mathcal{F}\left(\mathbf{T}^{*} \mathbb{Q}\right) \), define un sistema dinámico en \(\mathbf{T}^{*} \mathbb{Q}\) (los \( X^{f} \) son las componentes de un campo vectorial dinámico). Si no existe \( H \in \mathcal{F}\left(\mathbf{T}^{*} \mathbb{Q}\right) \) tal que \( X^{\prime} = \omega^{j k} \partial_{k} H \), puede ser imposible poner las ecuaciones de movimiento en la forma canónica de (5.42a).

\begin{example}
  Consideremos, por ejemplo, el sistema en dos grados de libertad dado por

\[
\dot{q} = q p, \quad \dot{p} = -q p \tag{5.61}
\]

Si este fuera un sistema hamiltoniano, \(\dot{q} = q p\) sería igual a \(\partial H / \partial p\), y \(\dot{p} = -q p\) sería igual a \(-\partial H / \partial q\). Pero entonces, \(\partial^{2} H / \partial q \partial p\) no sería igual a \(\partial^{2} H / \partial p \partial q\); por lo tanto, no existe una función \(H\) que sea el hamiltoniano de este sistema. No obstante, este es un sistema dinámico legítimo cuyas curvas integrales se encuentran fácilmente:

\[
q(t) = q_{0} \frac{C e^{C t}}{p_{0} + q_{0} e^{C t}}, \quad p(t) = p_{0} \frac{C}{p_{0} + q_{0} e^{C t}}
\]

donde \(C = q_{0} + p_{0} = q + p\) es una constante del movimiento.
\end{example}

Un papel especial del corchete de Poisson es que proporciona una prueba de si un sistema dinámico es hamiltoniano o no. 

\begin{proposition}
  Ahora mostramos que un sistema dinámico es hamiltoniano (o que los \(X^{\prime}\) son los componentes de un campo vectorial hamiltoniano) si y solo si la derivada temporal actúa sobre los paréntesis de Poisson como si fueran productos (es decir, mediante la regla de Leibniz). Es decir, el sistema es hamiltoniano si y solo si

\[
\frac{d}{d t}\{f, g\} = \{\dot{f}, g\} + \{f, \dot{g}\} \tag{5.62}
\]

\end{proposition}
\begin{proof}
  Para la demostración, asumamos que el sistema dinámico es hamiltoniano y que \( H(\xi, t) \) es la función hamiltoniana. Entonces,

\[
\frac{d}{d t}\{f, g\} = \{\{f, g\}, H\} + \partial_{t}\{f, g\}
\]

Según la identidad de Jacobi (y la antisimetría), el primer término se puede escribir como

\[
\{\{f, g\}, H\} = \{\{f, H\}, g\} + \{f, \{g, H\}\}
\]

El segundo término es

\[
\begin{aligned}
\partial_{t}\{f, g\} & = \partial_{t}\left[(\partial_j f) \, \omega_{j k} \, \partial_k g\right] \\
& = (\partial_j \, \partial_{t} f) \, \omega_{j k} \, \partial_k g + (\partial_j f) \, \omega_{j k} \, \partial_k \, \partial_{t} g \\
& = \{\partial_{t} f, g\} + \{f, \partial_{t} g\}.
\end{aligned}
\]

Combinando ambos resultados, se obtiene

\[
\begin{aligned}
\frac{d}{d t}\{f, g\} &= \left\{\{f, H\} + \partial_{t} f, g\right\} + \left\{f, \{g, H\} + \partial_{t} g\right\} \\
&= \{\dot{f}, g\} + \{f, \dot{g}\}.
\end{aligned}
\]

Esto prueba que, si existe una función hamiltoniana, la regla de Leibniz es válida para el paréntesis de Poisson.
\end{proof}

\begin{proof}
  Para demostrar la afirmación contraria, usaremos la siguiente lógica: si la regla de Leibniz de la Ecuación (5.62) se cumple para todos \( f, g \in \mathcal{F}\left(\mathbf{T}^{*} \mathbb{Q}\right) \), entonces se cumple en particular para el par \( \xi^{l}, \xi^{\prime} \). La demostración mostrará que si (5.62) se cumple para este par, entonces existe una función Hamiltoniana \( H \) tal que \( \omega_{j k} \dot{\xi}^{k} = \partial_{j} H \); es decir, la existencia de la función Hamiltoniana se deduce de la regla de Leibniz.

Dado que \( \left\{\xi^{l}, \xi^{\prime}\right\} = \omega^{l i} \), su derivada temporal es cero. Así, tenemos:

\[
\begin{aligned}
\frac{d}{d t}\{\xi^{l}, \xi^{\prime}\} &= 0 = \{\dot{\xi}^{l}, \xi^{\prime}\} + \{\xi^{l}, \dot{\xi}^{\prime}\} = \{X^{l}, \xi^{\prime}\} + \{\xi^{l}, X^{\prime}\} \\
&= (\partial_{j} X^{l}) \omega^{j k} \partial_{k} \xi^{\prime} + (\partial_{j} \xi^{l}) \omega^{j k} \partial_{k} X^{\prime} \\
&= \partial_{j}(X^{l} \omega^{j k}) + \partial_{k}(\omega^{j k} X^{l}),
\end{aligned}
\]

donde hemos usado la Ecuación (5.60). Ahora, multiplicamos por \( \omega_{l p} \omega_{i r} \) y sumamos sobre los índices repetidos, obteniendo

\[
\partial_{p} Z_{r} - \partial_{r} Z_{p} = 0,
\]

donde \( Z_{l} = \omega_{j k} X^{k} \). Esta es la condición de integrabilidad local (ver Problema 2.4) para la existencia de una función \( H \) que satisfaga el conjunto de ecuaciones diferenciales parciales

\[
\partial_{j} H = Z_{l} \equiv \omega_{l k} X^{k} \equiv \omega_{j k} \dot{\xi}^{k}.
\]

Esto completa la demostración. Hemos probado el resultado solo de forma local, lo cual es lo máximo que se puede lograr. Si el paréntesis de Poisson se comporta como un producto con respecto a la diferenciación temporal, entonces existe una función Hamiltoniana local \( H \); es decir, el sistema dinámico es localmente hamiltoniano. Puede no existir una única función \( H \) que sea válida en toda \( \mathbf{T}^{*} \mathbb{Q} \).
\end{proof}

\begin{corollary}
  
\end{corollary}

\section{Geometría simpléctica}

En esta sección se analiza la geometría de 
$\mathbf{T}^{*} \mathbb{Q}$ y cómo esta geometría contribuye a sus propiedades como una variedad portadora para la dinámica.

\subsection{Espacio cotangente}

Primero demostraremos la diferencia entre \( T \mathbb{Q} \) y \( \mathbf{T}^{*} \mathbb{Q} \). El haz tangente \( \mathbf{T} \mathbb{Q} \) consiste en la variedad de configuración \( \mathbb{Q} \) y el conjunto de los espacios tangentes \( \mathbf{T}_{q} \mathbb{Q} \), cada uno asociado a un punto \( q \in \mathbb{Q} \). Los puntos de \( \mathbf{T} \mathbb{Q} \) tienen la forma \( (q, \dot{q}) \), donde \( q \in \mathbb{Q} \) y \( \dot{q} \) es un vector en \( \mathbf{T}_{q} \mathbb{Q} \). Sin embargo, los puntos \( (q, p) \) de \( \mathbf{T}^{*} \mathbb{Q} \) no tienen esta forma, ya que \( p \) no es un vector en \( \mathbf{T}_{q} \mathbb{Q} \). Mostramos esto al considerar la forma diferencial \( \theta_{L} \equiv \left(\partial L / \partial \dot{q}^{\alpha}\right) d q^{\alpha} = p_{\alpha} d q^{\alpha} \) de la Ecuación (3.86) y al compararla con el campo vectorial \( \dot{q}^{\alpha}\left(\partial / \partial q^{\alpha}\right) \).

Los \( \dot{q}^{\alpha} \) son los componentes locales del campo vectorial \( \dot{q}^{\alpha}\left(\partial / \partial q^{\alpha}\right) \) en \( \mathbb{Q} \): para funciones dadas \( \dot{q}^{\alpha} \in \mathcal{F}(\mathbb{Q}) \), especifican el vector con componentes \( \dot{q}^{\alpha}(q) \) en el espacio tangente \( \mathbf{T}_{q} \mathbb{Q} \) en cada \( q \in \mathbb{Q} \). Sin embargo, los \( p_{\alpha} \) son los componentes locales de la forma diferencial \( \theta_{L} = p_{\alpha} d q^{\alpha} \). Aunque \( \theta_{L} \) se introdujo como una forma diferencial en \( \mathbb{T} \mathbb{Q} \), también puede verse como una forma diferencial en \( \mathbb{Q} \), ya que no contiene una parte \( d \dot{q} \) (cuando se ve de esta manera, sus componentes dependen, sin embargo, de los \( \dot{q}^{\alpha} \) como parámetros). Una forma diferencial no es un campo vectorial, y por lo tanto los \( p_{\alpha} = \partial L / \partial \dot{q}^{\alpha} \), componentes de una forma diferencial, no son componentes de un campo vectorial. En la Sección 3.4.1 mencionamos que las formas diferenciales son duales a los campos vectoriales, por lo que \( \theta_{L} \) pertenece a un espacio dual al \( \mathbf{T}_{q} \mathbb{Q} \). Este nuevo espacio se denota como \( \mathbf{T}_{q}^{*} \mathbb{Q} \) y se llama el espacio cotangente en \( q \in \mathbb{Q} \). Sus elementos, las formas diferenciales, mapean los vectores (es decir, se combinan con ellos en una especie de producto interno) hacia funciones.

\subsubsection{dos-formas}
Recordemos que en el tema 1, una 1-forma \( \alpha \) en \( T \mathbb{Q} \) se definió como un mapeo lineal de los campos vectoriales \( X \) a funciones, es decir, \( \alpha: \mathcal{X} \rightarrow \mathcal{F}: X \mapsto \langle \alpha, X \rangle \). (Esta definición es válida para cualquier variedad diferencial, tanto para \( \mathbf{T}^{*} \mathbb{Q} \) como para \( \mathbf{T Q} \)). Las 2-formas se definen como mapeos bilineales y antisimétricos de pares de campos vectoriales a funciones. Es decir, si \( \omega \) es una 2-forma en \( \mathbf{T}^{*} \mathbb{Q} \) y \( X \) y \( Y \) son campos vectoriales en \( \mathbf{T}^{*} \mathbb{Q} \), entonces

$$
\omega(X, Y)=-\omega(Y, X) \in \mathcal{F}\left(\mathbf{T}^{*} \mathbb{Q}\right) \tag{5.66}
$$

matemáticamente, \( \omega: \mathcal{X} \times \mathcal{X} \rightarrow \mathcal{F} \). Como el mapeo es bilineal, puede representarse localmente mediante una matriz cuyos elementos son (recordando que \( \partial_{j} \equiv \partial / \partial \xi^{\prime} \) es un campo vectorial)

$$
\omega_{j k} = \omega\left(\partial_{j}, \partial_{k}\right) = -\omega_{k j} = -\omega\left(\partial_{k}, \partial_{j}\right) \tag{5.67}
$$

de modo que, por linealidad, si (localmente) \( X = X^{j} \partial_{j} \) y \( Y = Y^{k} \partial_{j} \), entonces

$$
\omega(X, Y) = \omega_{j k} X^{j} Y^{k} \tag{5.68}
$$

¿Qué ocurre si una 2-forma se aplica solo a un campo vectorial \( Y^{k} \partial_{k} \), es decir, qué tipo de objeto es \( \omega_{j k} Y^{k} \)? No es una función, ya que su subíndice \( j \) implica que tiene \( n \) componentes. Sin embargo, se puede construir una función a partir de ella y de un segundo campo vectorial \( X^{j} \partial_{j} \), multiplicando los componentes y sumando sobre \( j \), obteniendo así \( \left(\omega_{j k} Y^{k}\right) X^{j} \), que es el lado derecho de la Ecuación (5.68). Esto significa que \( \omega_{j k} Y^{k} \) es el componente \( j \)-ésimo de una 1-forma, ya que puede utilizarse para mapear cualquier campo vectorial \( X^{j} \partial_{j} \) a una función. En una notación evidente, esta 1-forma puede llamarse \( \omega(\bullet, Y) = -\omega(Y, \bullet) \) (el \( \bullet \) indica dónde colocar el otro campo vectorial): cuando la 1-forma \( \omega(\bullet, Y) \) se aplica a un campo vectorial \( X = X^{j} \partial_{j} \), el resultado es \( \omega(X, Y) \).

Es conveniente introducir una terminología y notación uniformes para la acción de 1-formas y 2-formas sobre campos vectoriales. Se dice que los campos vectoriales se contraen con o se insertan en las formas, denotado por \( i_{X} \):

$$
i_{X} \alpha \equiv \alpha(X) \equiv \langle \alpha, X \rangle \quad \text { y } \quad i_{X} \omega \equiv \omega(\bullet, X) \tag{5.69}
$$

donde \( X \) es un campo vectorial, \( \alpha \) es una 1-forma, y \( \omega \) es una 2-forma. Entonces \( i_{X} \alpha \) y \( i_{Y} i_{X} \omega \equiv \omega(X, Y) = -i_{X} i_{Y} \omega \) son funciones, y \( i_{X} \omega \) es una 1-forma.
\subsubsection{k-formas}

De forma enteramente análoga a como hemos definidos las dos-formas, se definen las $k$-formas\sidenote{\href{https://www.youtube.com/watch?v=xRf9-hdxB0w}{En este video explica un un poco mejor que son las k-formas}} como tensores covariantes totalmente antisimétricos que actúan sobre $k$ vectores para dar un número real (o una función en el caso de campos). Mediante una asignación suave de una $k$-forma a cada punto de variedad diferenciables $\mathscr{M}$ obtenemos un campo de $k$-formas.

Los productos tensoriales antisimetrizados y ordenados $d \zeta^{\alpha_{1}} \dot{\Lambda} \cdots \dot{\lambda} d \zeta^{\alpha_{k}}$ forman una base de $k$-formas y cualquier $k$-forma puede escribir como combinación lineal de los mismos:
$$
\omega=\sigma_{\alpha_{1} \cdots \alpha_{k}} \mathrm{~d} \zeta^{\alpha_{1}} \dot{\wedge} \cdots \dot{\wedge} \mathrm{~d} \zeta^{\alpha_{k}}=\frac{1}{k!} \sigma_{\alpha_{1} \cdots \alpha_{k}} \mathrm{~d} \zeta^{\alpha_{1}} \wedge \cdots \wedge \mathrm{~d} \zeta^{\alpha_{k}}
$$

Definiremos la contracción $i_{X} \omega$ de una $k$-forma $\omega$ con un vector $X$ como
$$
i_{X} \omega=\omega(X, \diamond, \ldots, \diamond)=\frac{1}{(k-1)!} \sigma_{\alpha_{1} \cdots \alpha_{k}} X^{\alpha_{1}} \mathrm{~d} \zeta^{\alpha_{2}} \wedge \cdots \wedge \mathrm{~d} \zeta^{\alpha_{k}}
$$

Si $\sigma$ es una $k$-forma y $\rho$ una $l$-forma, definimos el producto exterior de ambas $\sigma \wedge \rho$ como su producto tensorial completamente antisimetrizado, cuyas componentes son:
$$
(\sigma \wedge p)_{\alpha_{1} \cdots \alpha_{k} \beta_{1} \cdots \beta_{l}}=\frac{(k+l)!}{k!!!} \sigma_{\left[\alpha_{1} \cdots \alpha_{k}\right.} \rho_{\left.\beta_{1} \cdots \beta_{2}\right]}
$$
Se puede demostrar que si $\sigma$ es una $k$-forma y $\rho$ una $l$-forma, entonces
$$
i_{X}(\omega \wedge \rho)=i_{X} \omega \wedge \rho+(-1)^{k} \omega \wedge i_{X} \rho
$$

\subsubsection{Derivada exterior}

Definimos la derivada exterior d como una operación que actúa sobre una $k$-forma $\omega$ para dar la $(k+1)$-forma
$$
\begin{aligned}
\mathrm{d} \omega & =\frac{1}{k!} \mathrm{d} \sigma_{\alpha_{1} \cdots \alpha_{k}} \wedge \mathrm{~d} \zeta^{\alpha_{1}} \wedge \cdots \wedge \mathrm{~d} \zeta^{\alpha_{k}}= \\
& =\frac{1}{k!} \partial_{\beta} \sigma_{\alpha_{1} \cdots \alpha_{k}} \mathrm{~d} \zeta^{\beta} \wedge \mathrm{d} \zeta^{\alpha_{1}} \wedge \cdots \wedge \mathrm{~d} \zeta^{\alpha_{k}}= \\
& =(k+1) \partial_{[\beta} \sigma_{\alpha_{1} \cdots \alpha_{k}} \mathrm{~d} \zeta^{\beta} \dot{\wedge} \zeta^{\alpha_{1}} \dot{\wedge} \dot{\wedge} \mathrm{~d} \zeta^{\alpha_{k}}
\end{aligned}
$$
cuyas componentes son $(k+1) \partial_{[\beta} \sigma_{\left.\alpha_{1} \cdots \alpha_{k}\right]}$. La derivada exterior satisface la "regla de Leibniz antisimetrizada"
$$
\mathrm{d}(\omega \wedge \rho)=\mathrm{d} \omega \wedge \rho+(-1)^{k} \omega \wedge \mathrm{~d} \rho
$$
donde $\omega$ es una $k$-forma y $\rho$ una $l$-forma.
Por ejemplo, la derivada exterior $\mathrm{d} \omega$ de una uno-forma $\omega=\sigma_{\alpha} \mathrm{d} \zeta^{\alpha}$ es 
$$
\mathrm{d} \omega=\mathrm{d}\left(\sigma_{\beta} \mathrm{d} \zeta^{\beta}\right)=\partial_{\alpha} \sigma_{\beta} \mathrm{d} \zeta^{\alpha} \wedge \mathrm{d} \zeta^{\beta}=2 \partial_{[\alpha} \sigma_{\beta]} \mathrm{d} \zeta^{\alpha} \dot{\lambda} \mathrm{d} \zeta^{\beta}
$$

Una forma es exacta si la derivada exterior de otra. Una forma es cerrada si su derivada exterior se anula. La condición necesaria y suficiente para que una $k$ forma sea exacta (localmente) es que sea una forma cerrada. En particular, para uno-formas $\omega$ y dos-formas $\rho$,
$$
\mathrm{d} \omega=0 \Leftrightarrow \exists f|\omega=\mathrm{d} f, \quad \mathrm{~d} \rho=0 \Leftrightarrow \exists \omega| \rho=\mathrm{d} \omega \quad \text { (localmente). }
$$

En componentes, $\partial_{[\alpha} \sigma_{\beta]}=0 \Leftrightarrow \sigma_{\alpha}=\partial_{\alpha} f, \quad \partial_{[\alpha} \rho_{\beta \gamma]}=0 \Leftrightarrow \rho_{\alpha \beta}=2 \partial_{[\alpha} \sigma_{\beta]}$.
\subsubsection{Ley de transformación II}

Como hemos visto, una $k$-forma es un tensor antisimétrico covariante. Por tanto, bajo cambios de coordenadas, una $k$-forma se transforma de forma inversa a como lo hacen los vectores en cada uno de sus índices:

\begin{DispWithArrows}[format=c, displaystyle]
  \omega_{\alpha}^{\prime}=\frac{\partial \zeta^{\beta}}{\partial \zeta^{\prime \alpha}} \omega_{\beta}, \tag{1-forma}\\  \omega_{\alpha \beta}^{\prime}=\frac{\partial \zeta^{\gamma}}{\partial \zeta^{\prime} \alpha} \frac{\partial \zeta^{\delta}}{\partial \zeta^{\prime \beta}} \omega_{\gamma \delta}, \tag{2-forma}\\  \omega_{\alpha_{1} \cdots \alpha_{k}}^{\prime}=\frac{\partial \zeta^{\gamma_{1}}}{\partial \zeta^{\prime \alpha_{1}}} \cdots \frac{\partial \zeta^{\gamma_{k}}}{\partial \zeta^{\prime \alpha_{k}}} \omega_{\gamma_{1} \cdots \gamma_{k}} \tag{k-forma}
\end{DispWithArrows}

\subsection{Geometria simppléctica}
En la sección anterior, solo hemos considerado coordenadas canónicas del espacio de fases $T^{*} \mathscr{Q}$ de la forma $\xi^{\alpha}=\left(q_{a}, p^{a}\right)$ que provienen directamente de
la transformación de Legendre. En estas coordenadas, la matriz simpléctica $\breve{\Omega}$ es constante y está dada por la ecuación 2.5. En esta sección, consideraremos una situación más general que lleva a su máxima expresión el hecho de que las variables de configuración y los momentos son tratados de igual manera. Reformularemos los principales resultados obtenidos hasta ahora de forma que sea independiente de las coordenadas en el espacio de fases. Para ello, serán necesarios algunos conceptos básicos de geometría diferencial que se incluyen en el apéndice B .

\subsubsection{Forma simpléctica}

Dadas las coordenadas canónicas $\xi^{\alpha}=\left(q^{a}, p_{a}\right)$ en el espacio fases que hemos usado hasta ahora, construimos la dos-forma $\Omega$ tal que, en esas coordenadas canónicas, adquiere la expresión
$$
\Omega=\frac{1}{2} \omega_{\alpha \beta} \mathrm{d} \xi^{\alpha} \wedge \mathrm{d} \xi^{\beta}=\mathrm{d} p_{a} \wedge \mathrm{~d} q^{a}
$$

Un cambio arbitrario de coordenadas $\xi^{\alpha} \rightarrow \zeta^{\alpha}$ convierte esta expresión en
$$
\Omega=\frac{1}{2} \omega_{\alpha \beta} \mathrm{d} \zeta^{\alpha} \wedge \mathrm{d} \zeta^{\beta}=\frac{1}{2} \omega_{\gamma \delta} \frac{\partial \xi^{\gamma}}{\partial \zeta^{\alpha}} \frac{\partial \xi^{\delta}}{\partial \zeta^{\beta}} \mathrm{d} \zeta^{\alpha} \wedge \mathrm{d} \zeta^{\beta} .
$$

Por ser una dos-forma, $\Omega$ es independiente de las coordenadas elegidas, aunque sus componentes sí dependen de las mismas.
$\omega$ es una matriz no singular y, por tanto, la dos-forma $\Omega$ es no degenerada, es decir, la aplicación $i \Omega: T\left(T^{*} \mathscr{Q}\right) \rightarrow T^{*}\left(T^{*} \mathscr{Q}\right)$ que a cada vector $X$ le asigna la uno-forma
$$
i_{X} \Omega=\Omega(X, \diamond)=\omega_{\alpha \beta} X^{\alpha} \mathrm{d} \zeta^{\beta}
$$
es un isomorfismo entre el espacio tangente y el cotangente del espacio de fases. $\diamond$ EJERCICIO: Comprobar que $\quad i_{\partial_{q a}} \boldsymbol{\Omega}=-\mathrm{d} p_{a}, \quad i_{\partial_{p_{a}}} \boldsymbol{\Omega}=\mathrm{d} q^{a}$.

Además, la dos-forma $\Omega$ es cerrada, es decir, $\mathrm{d} \Omega=0$, como se puede comprobar fácilmente (EJERCICIO). Por tanto, existe una uno-forma $\theta$ llamada forma canónica tal que, localmente, $\Omega=\mathrm{d} \theta$. La expresión de la forma canónica (en términos de las variables canónicas) es $\theta=p_{a} \mathrm{~d} q^{a}$ (EJERCICIO). Llamaremos forma simpléctica a esta dos-forma $\Omega$ definida sobre el espacio de fases, que es cerrada y no degenerada.

Los elementos $\Omega^{\alpha \beta}$ de la matriz inversa a la formada por las componentes de la forma simpléctica en unas ciertas coordenadas, es decir, tales que $\Omega^{\alpha \beta} \Omega_{\beta \gamma}=\delta_{\gamma}^{\alpha}$, son las componentes de un tensor antisimétrico contravariante.

Desde el punto de vista notacional, es importante notar que tanto la derivada exterior y las formas como los vectores están definidos sobre el espacio de fases $T^{*} \mathscr{Q}$ y son, por tanto, ajenos a los parámetros adicionales de los que una variable dinámica pueda depender, en particular, del tiempo. Así, dada una variable dinámica $f(\zeta, t)$ y un vector $X$,
$$
\mathrm{d} f=\partial_{\alpha} f \mathrm{~d} \zeta^{\alpha}+\partial_{f} f \mathrm{~d} t \quad \text { у } \quad X f=X^{\alpha} \partial_{\alpha} f+\partial_{t} f
$$

Además, utilizaremos el mismo símbolo $\partial_{\alpha}$ para denotar tanto a las derivadas parciales con respecto a las variables canónicas $\partial / \partial \xi^{\alpha}$ como con respecto a variables arbitrarias $\partial / \partial \zeta^{\alpha}$, según el contexto, salvo cuando exista posibilidad de confusión.

\subsubsection{Ecuaciones de Hamilton}

Las ecuaciones de Hamilton $\omega_{\alpha \beta} \dot{\xi} \beta=\partial_{\xi \alpha} H$ se pueden escribir también de forma independiente de las coordenadas. Para ello, consideremos un sistema dinámico $\dot{\xi}^{\alpha}=X^{\alpha}$ cuyo vector dinámico es $X$. Si este sistema es hamiltoniano, entonces el vector dinámico $X$ debe ser tal que
$$
\omega_{\alpha \beta} X^{\beta}=\partial_{\xi \alpha} H
$$
como se puede ver comparando ambas ecuaciones. Por tanto, las ecuaciones de Hamilton se traducen en encontrar los vectores $X$ que satisfacen esta ecuación $y$, después, encontrar sus curvas integrales. Notemos que el miembro de la izquierda son las componentes de la contracción $i_{X} \Omega$ de la forma simpléctica con el vector dinámico (cambiadas de signo) y que, en el miembro de la derecha, aparecen las componentes de la uno-forma $\mathrm{d} H$ en la base $\left\{\mathrm{d} \xi^{\alpha}\right\}$ adaptada a las coordenadas canónicas. Multiplicando ambos miembros por $\mathrm{d} \xi^{\beta}$, vemos que las ecuaciones de Hamilton se pueden escribir como una ecuación entre formas que es válida en unas coordenadas concretas; por tanto, es válida en cualquier sistema de coordenadas. Así, en una formulación independiente de las coordenadas, las ecuaciones de Hamilton determinan el vector dinámico $X$ cuyas curvas integrales son las trayectorias clásicas a través de la expresión
$$
i_{X} \Omega=-\mathrm{d} H
$$

Dado que la forma simpléctica es no degenerada, cada hamiltoniano determina su vector dinámico de forma única y cada vector dinámico determina el hamiltoniano unívocamente salvo por una cantidad aditiva independiente de las variables dinámicas que corresponde a la indeterminación de la energía potencial. En efecto, si existiesen dos vectores dinámicos $X_{H}$ y $Y_{H}$ correspondientes al mismo hamiltoniano, entonces $i_{X_{H}-Y_{H}} \Omega=0$, lo que a su vez implica que $X_{H}-Y_{H}=0$, ya que $\Omega$ es no degenerada. Por otro lado, si existen dos hamiltonianos $H_{1}$ y $H_{2}$ asociados al mismo vector dinámico, entonces se verifica que $\mathrm{d}\left(H_{1}-H_{2}\right)=0$, lo que implica que el hamiltoniano queda determinado salvo por una cantidad aditiva que no depende del espacio de fases.

\subsubsection{Flujos hamiltonianos}

\paragraph{Flujo de un campo vectorial}

Sea $X$ un campo vectorial sobre el espacio de fases y consideremos la aplicación $\varphi_{s}^{X}$ en el espacio de fases tal que a cada punto $\zeta \in T^{*} \mathscr{Q}$ le asigna el punto $\zeta_{s}$ que se halla a una distancia paramétrica $s$ sobre la curva integral de $X$, es decir, sobre la solución de la ecuación $\mathrm{d} \zeta_{s} / \mathrm{d} s=X$, que pasa por $\zeta$. El conjunto $\varphi^{X}=\left\{\varphi_{s}^{X}, s \in \mathbb{R}\right\}$ de todas las transformaciones de este tipo tiene obviamente estructura de grupo bajo la ley de composición $\varphi_{s_{1}}^{X} \circ \varphi_{s_{2}}^{X}=\varphi_{s_{1}+s_{2}}^{X}$ y recibe el nombre de flujo del campo vectorial $X$.
$\diamond$ EJEMPLO: Consideremos el campo vectorial $\boldsymbol{X}_{v}=\left(v^{a}, 0\right)$ en unas coordenadas canónicas. Este vector genera un flujo en el espacio de fases que corresponde a traslaciones en el espacio de configuración. En efecto, las curvas integrales de $X_{v}$ son
$$
q^{a}(s)=q_{0}^{a}+v^{a} s, \quad p_{a}(s)=p_{a 0}
$$
donde $q_{0}^{a}$ y $p_{a 0}$ son constantes.
Consideremos un pequeño desplazamiento a lo largo del flujo del vector $X$ caracterizado por una distancia paramétrica infinitesimal $\delta s$. La evolución de cualquier tensor (función escalar, vector, forma...) $W$ definido sobre el espacio de fases a lo largo de este flujo estará determinada por su derivada de Lie y por su posible dependencia explícita en el parámetro de la transformación:
$$
\varphi_{\delta s}^{X} W=W+\delta s \mathfrak{L}_{X} W+\delta s \partial_{s} W
$$

En otras palabras, la transformación infinitesimal $\varphi_{\delta s}^{X}$ se puede escribir de la forma
$$
\varphi_{\delta s}^{X}=\mathbb{I}+\delta s\left(\mathfrak{L}_{X}+\partial_{s}\right)
$$

Cualquier transformación finita $\varphi_{s}^{X}$ se puede descomponer en transformaciones infinitesimales. Una posible manera de llevar a cabo esta descomposición se obtiene si escribimos $s=k \cdot(s / k) \operatorname{con} k \rightarrow \infty$. Entonces,
$$
\varphi_{s}^{X}=\lim _{k \rightarrow \infty} \varphi_{k \cdot(s / k)}^{X}=\lim _{k \rightarrow \infty}\left(\varphi_{s / k}^{X}\right)^{k}=\lim _{k \rightarrow \infty}\left[\mathbb{I}+(s / k)\left(\mathfrak{L}_{X}+\partial_{s}\right)\right]^{k}=e^{s\left(\mathfrak{L}_{X}+\partial_{s}\right)}
$$

Es importante notar que la actuación de la exponencial de un operador se define a través del proceso de límite que acabamos de describir o, equivalentemente, mediante un desarrollo en serie:
$$
\varphi_{s}^{X}=e^{s\left(\mathfrak{(}_{X}+\partial_{s}\right)}=\mathbb{I}+\sum_{k=1}^{\infty} \frac{s^{k}}{k!}\left(\mathfrak{L}_{X}+\partial_{s}\right)^{k}
$$

\paragraph{Campos vectoriales hamiltonianos}

Dada una variable dinámica $f$, definimos el campo vectorial hamiltoniano $X_{f}$ asociado a ella como el único vector que satisface la ecuación
$$
i_{X_{f}} \Omega=-\mathrm{d} f
$$
y que, por tanto, se puede escribir también de la forma $X_{f}=\Omega^{\alpha \beta} \partial_{\beta} f \partial_{\alpha}$.
No todos los campos vectoriales son hamiltonianos, como es obvio a partir de estas expresiones. De hecho, es fácil ver que un campo vectorial $X$ es hamiltoniano si y solo si la forma $i_{X} \Omega$ es cerrada (es decir, $\mathrm{d} i_{X} \Omega=0$ ) si y solo si $\partial_{\lambda}\left(\Omega_{\alpha \sigma} X^{\alpha}\right)-\partial_{\sigma}\left(\Omega_{\beta \lambda} X^{\beta}\right)=0$, que es la condición de integrabilidad local de la ecuación $\omega_{\alpha \beta} X^{\beta}=\partial_{\alpha} f$.
$\diamond$ EJEMPLO: El campo vectorial $\boldsymbol{X}_{v}=\left(v^{a}, 0\right)$ del ejemplo anterior (asociado a las traslaciones) es hamiltoniano ya que $i_{X_{v}} \Omega=-v^{a} \mathrm{~d} p_{a}$ y, por tanto, $\mathrm{d} i_{X_{v}} \Omega=0$. De hecho su variable dinámica asociada es $v(q, p)=v^{a} p_{a}$.

Cada variable dinámica $f$ determina de forma única un campo vectorial hamiltoniano $X_{f} \mathrm{y}$, consecuentemente, un flujo bamiltoniano
$$
\varphi^{f}=\varphi^{X_{f}}=e^{s\left(\mathcal{L}_{X_{f}}+\mathcal{Z}_{s}\right)} .
$$

La variable $f$ recibe el nombre de generatriz infinitesimal del flujo $\varphi^{f}$.
Notemos que sobre variables dinámicas (funciones escalares sobre el espacio de fases), $\mathfrak{L}_{X_{f}} g=X_{f} g=\{g, f\} y$, por tanto,
$$
\varphi^{f} g=e^{s\left(\mathfrak{I}_{X_{f}}+\partial_{s}\right)} g=e^{s\left(\{, f\}+\partial_{s}\right)} g .
$$

EJEMPLO: Sobre variables dinámicas $g(q, p)$ el flujo de traslaciones del ejemplo anterior es
$$
\varphi^{v} g=e^{s v^{2} \partial_{a^{a}}} g
$$

Así, la proyección del momento canónico en una cierta dirección $v^{a}$ del espacio de configuración genera un flujo hamiltoniano de traslaciones en esa dirección y , sobre variables dinámicas, actúa por medio del operador $v^{a} \partial_{q^{a}}$.
$\diamond$ EJERCICIO: Demostrar que el campo vectorial hamiltoniano del corchete de Poisson de dos variables dinámicas es el conmutador (cambiado de signo) de sus respectivos campos vectoriales hamiltonianos:
$$
X_{\{f, g\}}=-\left[X_{f}, X_{g}\right]
$$


Las ecuaciones de Hamilton se reducen, en este lenguaje independiente de las coordenadas, a encontrar el campo vectorial hamiltoniano $X_{H}$ asociado al hamiltoniano $H$ y su flujo, es decir, sus curvas integrales. La evolución clásica de cualquier tensor $W$ definido sobre el espacio de fases está dada por su derivada a lo largo de la trayectoria clásica, es decir, con respecto al vector dinámico $X_{H}$ (más la posible dependencia temporal explícita):
$$
\dot{W}=\mathfrak{L}_{X_{H}} W+\partial_{t} W
$$

En particular, la evolución clásica de cualquier variable dinámica está dada por
$$
\dot{f}=X_{H} f+\partial_{t} f=\{f, H\}+\partial_{t} f
$$

\paragraph{Teorema de Noether}
Sea $f$ una variable dinámica que no depende explícitamente del tiempo. Entonces, el hamiltoniano es invariante bajo la acción del grupo de transformaciones generado por $f$ si y solo si $f$ es una cantidad conservada.
Demostración. Para demostrar este teorema, basta darse cuenta que si $\partial_{t} f=0$, entonces, $\mathfrak{L}_{X_{f}} H=X_{f} H=\{H, f\}=-\dot{f}$\todo{$\dot{f}=\Omega(X_{f},X_{g})$}. Por tanto, $\dot{f}=0$ si y solo si el hamiltoniano no cambia a lo largo del flujo de $X_{f}$.
\subsubsection{Variedades simplécticas}
En este apartado, abandonamos el espacio de fases para considerar situaciones más generales. Sea $\mathscr{M}$ una variedad diferenciable de dimensión $2 n$. Llamaremos forma simpléctica en $\mathscr{M}$ a cualquier dos-forma $\Omega$ no degenerada y cerrada y llamaremos variedad simpléctica al par $(\mathscr{M}, \Omega)$.

Cualquier función $H$ en la variedad simpléctica $(\mathscr{M}, \Omega)$ define una evolución hamiltoniana, donde el parámetro $t$ de evolución es un parámetro de las curvas integrales del campo vectorial hamiltoniano $X_{H}$ asociado a la función $H$, que llamaremos hamiltoniano. Más explícitamente, dada la función $H$,
\begin{enumerate}
  \item Construimos el campo hamiltoniano $X_{H}$ solución de la ecuación de Hamilton $i_{X_{H}} \Omega=-\mathrm{d} H$ y que, en coordenadas arbitrarias, se puede escribir $\operatorname{como} X_{H}=\Omega^{\alpha \beta} \partial_{\beta} H \partial_{\alpha}$;
  \item Encontramos las curvas integrales $\gamma_{H}$ 
  \item Mediante condiciones iniciales, elegimos una de ellas y la describimos mediante un parámetro $t$ tal que $\dot{\gamma}_{H}=X_{H}$.
\end{enumerate}


Esta curva $\gamma_{H}(t)$ es la trayectoria clásica en el tiempo $t$ generada por el hamiltoniano $H$, que satisface las ecuaciones de Hamilton correspondientes a este hamiltoniano.

El corchete de Poisson de dos funciones $f$ y $g$ en la variedad simpléctica se define a partir de la forma simpléctica de la forma
$$
\{f, g\}:=X_{g} f=-\Omega\left(X_{f}, X_{g}\right)=\Omega^{\alpha \beta} \partial_{\alpha} f \partial_{\beta} g .
$$

Es obvio que el corchete de Poisson, así definido, es bilineal y antisimétrico. También se puede demostrar  que la identidad de Jacobi es equivalente a que la forma simpléctica es cerrada.
\paragraph{Teorema de Darboux}
El teorema de Darboux garantiza que, dada una variedad simpléctica $(\mathscr{M}, \Omega)$, existe una carta de coordenadas $\xi^{\alpha}=\left(q^{a}, p_{a}\right)$ alrededor de cada punto tal que, en esa carta, la forma simpléctica adquiere la forma
$$
\Omega=\frac{1}{2} \breve{\Omega}_{\alpha \beta} \mathrm{d} \xi^{\alpha} \wedge \mathrm{d} \xi^{\beta}=\mathrm{d} p_{a} \wedge \mathrm{~d} q^{a}
$$
\begin{theorem}[Teorema de Darboux]
  El teorema de Darboux nos garantiza que cualquier coordenada en el entrono local, podrá ser canónica, aunque una vez elegida esa, el resto ya no se puede elegir libremente.

  Es decir, este teorema nos permite elegir las coordenadas que más simplifiquen la dinámica del problema
\end{theorem}
\begin{marginfigure}[]
  \includegraphics{}
  \caption[]{Ilustración del teorema de Darboux en una variedad simpléctica unidimensional.}
  \labfig{fig:darboux}
\end{marginfigure}
En otras palabras, el teorema de Darboux garantiza que, localmente, todas las variedades simplécticas admiten coordenadas canónicas y son, por tanto, localmente isomorfas a $T^{*} \mathscr{Q}$. Así, gracias al teorema de Darboux, todos los resultados y desarrollos llevados a cabo en este tema y, en particular, en esta sección $\$ 2.2$ son válidos (localmente) en cualquier variedad simpléctica. En este curso, estamos estudiando solo los aspectos locales de la dinámica hamiltoniana, por lo que, para nosotros no existe ninguna diferencia entre una variedad simpléctica arbitraria y el espacio de fases, en virtud del teorema de Darboux.

El teorema de Darboux no solo nos asegura la existencia de coordenadas canónicas locales en cualquier variedad simpléctica. Además nos dice que cualquier variable dinámica puede ser una coordenada canónica. Elegida una, las demás ya no se pueden escoger libremente aunque el margen de elección es grande.

Consideremos como ejemplo una variedad simpléctica bidimensional. Dada una variable dinámica arbitraria $g$, tenemos que encontrar otra $f$ tal que $\{f, g\}=1$. Para ello seguiremos el siguiente procedimiento (ver figura \ref{fig:darboux}):
\begin{enumerate}
  \item Consideremos el flujo hamiltoniano $\varphi^{g}$ generado por $g$, es decir, el conjunto de todas las curvas integrales de $X_{g}$. Notemos que, en general, estas curvas están contenidas en subvariedades $g=$ constante, ya que $X_{g} g=0$ o, lo que es lo mismo, $g$ se preserva a lo largo de cada curva integral. En este caso, cada curva del flujo es de la forma $g=$ constante, por ser estas subvariedades unidimensionales.
  \item Elijamos un punto cualquiera $\zeta_{0}$ de la variedad y construyamos otra subvariedad $\mathscr{N}$ que lo contenga y que corte a todas las trayectorias del flujo $\varphi^{g}$. En este caso, $\mathscr{N}$ también es unidimensional.
  \item Construyamos la variable dinámica $f$ tal que a cada punto de la variedad simpléctica le asigna su distancia paramétrica $s$ al punto de referencia a lo largo del flujo de $g$.
  \item El corchete de Poisson de estas dos variables es $\{f, g\}=\boldsymbol{X}_{g} f=1$, puesto que $X_{g}$ es el generador infinitesimal del flujo, es decir, es igual a $\mathrm{d} / \mathrm{d} s \mathrm{y}$ $f=s$.
\end{enumerate}


Consideremos ahora una variedad ( $2 n$ )-dimensional. Ahora, cada curva del flujo $\varphi^{g}$ está contenida en una subvariedad $g=$ constante pero no coincide con ella ya que esta subvariedad es $(2 n-1)$-dimensional. El mismo procedimiento descrito para el caso bidimensional se puede seguir en este caso. Construimos otra subvariedad $(2 n-1)$-dimensional $\mathscr{N}$ que corte a todas las líneas de flujo y definimos $f$ de la misma manera. Entonces, es claro que $\{f, g\}=1$. Sin embargo, nada sabemos de los corchetes de Poisson con las otras $2 n-2$ coordenadas restantes. El teorema de Darboux lo que nos asegura es, precisamente, que la forma simpléctica se puede separar (localmente) en dos partes:
$$
\begin{equation*}
\boldsymbol{\Omega}=\mathrm{d} g \wedge \mathrm{~d} f+\left.\boldsymbol{\Omega}\right|_{\mathscr{L}_{g}} \tag{2.9}
\end{equation*}
$$
donde $\left(\mathscr{L}_{g},\left.\Omega\right|_{\mathscr{L}_{g}}\right)$ es una variedad simpléctica  $\left(2 n-2\right)$-dimensional. Aplicando el mismo procedimiento sucesivamente podemos obtener todas las coordenadas canónicas.
\paragraph{Reducción por simetría}
Una aplicación especialmente interesante de este procedimiento consiste en utilizar las cantidades conservadas como variables canónicas y reducir así el espacio de fases en dos dimensiones (un grado de libertad) por cada cantidad conservada.

Si $g$ es una cantidad conservada, el teorema de Darboux nos permite encontrar otra variable dinámica $f$ y escribir la forma simpléctica de la forma 2.9 con
la peculiaridad de que $f$ es una coordenada cíclica ya que $0=\dot{g}=-\partial_{f} H$. De esta manera, hemos reducido la dinámica a un espacio de fases $(2 n-2)$-dimensional $\mathscr{L}_{g}$. El teorema de Darboux nos asegura que esta dinámica reducida es también hamiltoniana, como se demuestra a continuación.

La restricción a la subvariedad $\mathscr{L}_{g}$ de cualquier campo vectorial y, en particular, del vector dinámico $X$ dependerá no solo de las coordenadas $\zeta^{\bar{\alpha}}, \bar{\alpha}=2, \ldots, 2 n$ sino también de $f$ y de $g$. Sin embargo, $g$ es constante sobre $\mathscr{L}_{g}$ y $f$ no aparece en el vector dinámico $X$ por lo que tampoco aparece en $\left.X\right|_{\mathscr{L}_{g}}$. En resumen, tanto $\left.\Omega\right|_{\mathscr{L}_{g}}$ como $\left.X\right|_{\mathscr{L}_{g}}$ dependen solo de las variables de $\mathscr{L}_{g}$.

Nos ocupamos ahora de las ecuaciones de Hamilton. Para ello, notemos que
$$
\begin{aligned}
i_{X} \Omega & =-X^{f} \mathrm{~d} g+X^{g} \mathrm{~d} f+\left.i_{\left.X\right|_{\mathscr{g}}} \Omega\right|_{\mathscr{L}_{g}} \\
\mathrm{~d} H & =\partial_{f} H \mathrm{~d} f+\partial_{g} H \mathrm{~d} g+\left.\mathrm{d} H\right|_{\mathscr{g}_{g}}
\end{aligned}
$$

Entonces, las ecuaciones de Hamilton $i_{X} \Omega=-\mathrm{d} H$ se separan en
$$
X^f=\partial_g H, \quad X^g=-\partial_f H,\left.\quad i_{X \mid \mathscr{x}_g} \Omega\right|_{\mathscr{L}_g}=-\left.\mathrm{d} H\right|_{\mathscr{L}_g}
$$
Las dos primeras ecuaciones son las ecuaciones de Hamilton para las variables $f$ y $g$; nos indican que $g$ es el momento canónico conjugado a $f$; además, por ser $g$ conservada, $f$ es cíclica ya que $X^{g}=\dot{g}$. La última ecuación constituye las ecuaciones de Hamilton sobre $\mathscr{L}_{g}$, por lo que la dinámica reducida es hamiltoniana.

Supongamos que el sistema posee dos cantidades conservadas $g_{1}$ y $g_{2}$. Con la primera podemos proceder de la misma manera. Sin embargo, la segunda cantidad conservada no será en general solo función de las coordenadas $\zeta^{\bar{\alpha}}$ de $\mathscr{L}_{g_{1}}$ sino que dependerá también de $g_{1}$ y de $f_{1}$. En realidad, puesto que estamos suponiendo que $g_{1}$ y $g_{2}$ son funcionalmente independientes, en la situación más general, $g_{2}$ dependerá de $f_{1} \mathrm{y}$ de $\zeta^{\bar{\alpha}}$. Así, para poder reducir el sistema nuevamente, es decir, para poder utilizar $g_{1}$ y $g_{2}$ como coordenadas canónicas, la condición necesaria y suficiente es que ambas estén en involución:
$$
\partial_{f_{1}} g_{2}=-\left\{g_{1}, g_{2}\right\}=0
$$
\section{Transformaciones canónicas}
\subsection{Simplectomorfismos}
Una transformación continua en el espacio de fases puede verse en términos pasivos (si cambiamos las coordenadas en un mismo punto) o desde el punto de vista activo (desplazando el punto). En esta sección, adoptaremos el punto de vista activo. Veamos a continuación cuatro definiciones equivalentes de transformación canónica.

\begin{enumerate}
  \item Llamaremos transformación canónica o simplectomorfismo a cualquier transformación en el espacio de fases que preserve la forma simpléctica $\Omega$. Si tenemos una transformación finita $\zeta \rightarrow \zeta^{\prime}$, la condición de que se preserve la forma simpléctica es que su valor en el punto $\zeta^{\prime}$ sea el mismo que el punto $\zeta$. Sin embargo, para poder comparar dos formas, necesitamos evaluarlas en el mismo punto. Por tanto, trasladamos la forma simpléctica $\Omega\left(\zeta^{\prime}\right)$ evaluada en $\zeta^{\prime}$ al punto original $\zeta \sin$ modificarla y en este punto la comparamos con su valor original $\boldsymbol{\Omega}(\zeta)$. Así la condición de que se preserve la forma simpléctica bajo la transformación $\zeta \rightarrow \zeta^{\prime}$ es que
  $$
  \Omega\left[\zeta^{\prime}(\zeta)\right]-\Omega(\zeta)=0
  $$
  
  Las expresiones para $\Omega$ en los puntos inicial $\zeta$ y final $\zeta^{\prime}$ son
  $$
  \Omega(\zeta)=\frac{1}{2} \Omega_{\alpha \beta}(\zeta) \mathrm{d} \zeta^{\alpha} \wedge \mathrm{d} \zeta^{\beta}, \quad \Omega\left(\zeta^{\prime}\right)=\frac{1}{2} \Omega_{\gamma \delta}\left(\zeta^{\prime}\right) \mathrm{d} \zeta^{\prime \gamma} \wedge \mathrm{d} \zeta^{\prime \delta}
  $$
  por lo que
  $$
  \Omega\left[\zeta^{\prime}(\zeta)\right]=\frac{1}{2} \Omega_{\gamma \delta}\left[\zeta^{\prime}(\zeta)\right] \partial_{\alpha} \zeta^{\prime \gamma}(\zeta) \partial_{\beta} \zeta^{\prime \delta}(\zeta) \mathrm{d} \zeta^{\alpha} \wedge \mathrm{d} \zeta^{\beta}
  $$
  
  Por tanto, la transformación $\zeta \rightarrow \zeta^{\prime}(\zeta)$ es canónica si y solo si
  $$
  \Omega_{\gamma \delta}\left(\zeta^{\prime}\right) \partial_{\alpha} \zeta^{\prime} \gamma \partial_{\beta} \zeta^{\prime \delta}-\Omega_{\alpha \beta}(\zeta)=0 .
  $$
  \item Una transformación generada por el vector $X$ es canónica si y solo si la derivada de Lie de la forma simpléctica a lo largo del vector $X$ es cero:
  $$
  \mathfrak{L}_{X} \Omega=\frac{1}{2}\left(X^{\gamma} \partial_{\gamma} \Omega_{\alpha \beta}+\Omega_{\alpha \gamma} \partial_{\beta} X^{\gamma}+\Omega_{\gamma \beta} \partial_{\alpha} X^{\gamma}\right) \mathrm{d} \zeta^{\alpha} \wedge \mathrm{d} \zeta^{\beta}=0 .
  $$
  \begin{proof}Para verificar la equivalencia de ambas condiciones basta con comprobar que, bajo la transformación infinitesimal $\zeta^{\prime}=\zeta+\delta \zeta$, donde $\delta \zeta=X \delta s$, la forma simpléctica varía en $\delta \Omega=\mathfrak{L}_{X} \Omega \delta s$
  \end{proof}
  \item Una transformación es canónica si y solo si su vector generador $X$ es hamiltoniano, es decir, si y solo si existe una función $f$ tal que $i_{X} \Omega=-\mathrm{d} f$.
  \item \begin{proof}
    La identidad de Cartan nos permite escribir
$$
\mathfrak{L}_{X} \Omega=i_{X} \mathrm{~d} \Omega+\mathrm{d} i_{X} \Omega
$$

El primer término se anula por ser $\Omega$ cerrada. Por tanto, $X$ genera una transformación canónica si y solo si $0=\mathfrak{L}_{X} \Omega=\mathrm{d} i_{X} \Omega$. Esta condición es necesaria y suficiente para que exista (localmente) una función $f$ tal que $i_{X} \Omega=-\mathrm{d} f$.
  \end{proof}
  \item Una transformación $\zeta \rightarrow \zeta^{\prime}$ es canónica si y solo si preserva los corchetes de Poisson entre las variables originales, es decir, si y solo si
  $$
  \left\{\zeta^{\alpha}, \zeta^{\beta}\right\}_{\zeta}=\left\{\zeta^{\alpha}, \zeta^{\beta}\right\}_{\zeta^{\prime}}
  $$
  \begin{proof}
    El corchete de Poisson entre dos funciones $f$ y $g$ en las variables $\zeta^{\alpha}$ y $\zeta^{\prime \alpha}$ son respectivamente
    $$
    \{f, g\}_{\zeta}=\Omega^{\alpha \beta}(\zeta) \partial_{\alpha} f \partial_{\beta} g, \quad\{f, g\}_{\zeta^{\prime}}=\Omega^{\alpha \beta}\left(\zeta^{\prime}\right) \partial_{\alpha}^{\prime} f \partial_{\beta}^{\prime} g
    $$
    (donde $\partial_{\alpha}^{\prime}=\partial / \partial \zeta^{\prime \alpha}$ ). Por tanto, la igualdad de los corchetes de Poisson de las variables originales en ambos conjuntos de variables $\left\{\zeta^{\alpha}, \zeta^{\beta}\right\}_{\zeta}=\left\{\zeta^{\alpha}, \zeta^{\beta}\right\}_{\zeta^{\prime}}$ es equivalente a la condición $\Omega^{\alpha \beta}(\zeta)=\Omega^{\gamma \delta}(\zeta) \partial_{\gamma}^{\prime} \zeta^{\alpha} \partial_{\delta}^{\prime} \zeta^{\beta}$. Esta condición es, a su vez, equivalente a que la forma simpléctica se preserve por ser esta no degenerada.
  \end{proof}
\end{enumerate}
El conjunto de todas las transformaciones diferenciables en $T^{*} \mathscr{Q}$ tiene estructura de grupo como ya vimos. El conjunto de todas las transformaciones canónicas también tiene estructura de grupo puesto que corresponden a la descripción pasiva (cambios de coordenadas) de flujos hamiltonianos y la composición de dos flujos hamiltonianos es hamiltoniano
\subsection{Variables canónicas}
Todos estos resultados y definiciones se aplican, en particular, a transformaciones sobre las variables canónicas $\xi^{\alpha}=\left(q^{a}, p_{a}\right)$. En este caso, las transformaciones canónicas $\xi \rightarrow \xi^{\prime}$ son aquellas que verifican
$$
\check{\Omega}_{\gamma \delta} \partial_{\alpha} \xi^{\prime \gamma} \partial_{\beta} \xi^{\prime \delta}=\breve{\Omega}_{\alpha \beta}
$$

Por tanto, en términos de variables canónicas, una transformación es canónica si y solo si preserva la matriz simpléctica o, equivalentemente, si y solo si el corchete de Poisson de las nuevas variables es $\left\{\xi^{\prime \alpha}, \xi^{\prime \beta}\right\}_{\xi}=\check{\Omega}^{\alpha \beta}$. Si escribimos esta condición explícitamente para las variables de configuración y los momentos canónicos, vemos que la condición necesaria y suficiente para que una transformación de las variables canónicas $(q, p) \rightarrow\left(q^{\prime}, p^{\prime}\right)$ sea canónica es que
$$
\left\{q^{\prime a}, q^{\prime b}\right\}_{q, p}=0, \quad\left\{q^{\prime a}, p_{b}^{\prime}\right\}_{q, p}=\delta_{b}^{a}, \quad\left\{p^{\prime a}, p^{\prime b}\right\}_{q, p}=0
$$
\subsection{Transformaciones canonoides}
ea $H(\zeta, t)$ el hamiltoniano del sistema en las variables $\zeta^{\alpha}$. Si realizamos una transformación canónica, entonces obtendremos un nuevo hamiltoniano $H^{\prime}\left(\zeta^{\prime}, t\right)$ para las nuevas variables $\zeta^{\prime \alpha}$. En efecto, dadas dos variables dinámicas arbitrarias $f$ y $g$, la evolución de su corchete de Poisson en las nuevas variables $\zeta^{\prime}$ es
$$
\frac{\mathrm{d}}{\mathrm{~d} t}\{f, g\}_{\zeta^{\prime}}=\frac{\mathrm{d}}{\mathrm{~d} t}\{f, g\}_{\zeta}=\{\dot{f}, g\}_{\zeta}+\{f, \dot{g}\}_{\zeta}=\{\dot{f}, g\}_{\zeta^{\prime}}+\{f, \dot{g}\}_{\zeta^{\prime}}
$$

Como vimos en $\$ 2.1$.4, esta ley de evolución para el corchete de Poisson garantiza la existencia de un nuevo hamiltoniano $H^{\prime}\left(\zeta^{\prime}, t\right)$ tal que las ecuaciones de evolución en la nuevas variables son las ecuaciones de Hamilton correspondientes a este hamiltoniano:
$$
\Omega_{\alpha \beta} \dot{\zeta}^{\prime \beta}=\partial_{\alpha}^{\prime} H^{\prime}, \quad \dot{\zeta}^{\prime \alpha}=\left\{\zeta^{\prime \alpha}, H^{\prime}\right\}_{\zeta^{\prime}}
$$

La afirmación inversa no es cierta: una transformación que convierte un hamiltoniano dado en otro no es necesariamente una transformación canónica. Este tipo de transformaciones reciben el nombre de transformaciones canonoides.

\begin{example}
  Por ejemplo, la transformación $Q=q, P=\sqrt{p}-\sqrt{q}$ convierte el hamiltoniano $H=p^{2} / 2$ en el hamiltoniano $K=\left(P+Q^{2}\right)^{2} / 2$. Sin embargo los corchetes de Poisson de las viejas variables en las nuevas no es el adecuado: $\{q, p\}_{P, Q}=2(P+\sqrt{Q}) \neq 1$ .
\end{example}

Es posible demostrar  que una transformación preserva la forma simpléctica (salvo una constante multiplicativa que puede absorberse mediante un escalado trivial de las variables) si es canonoide para cualquier hamiltoniano que contiene a lo sumo términos cuadráticos en las variables del espacio de fases.

Así, acabamos de ver que una transformación es canónica si y solo si es canonoide para cualquier hamiltoniano.
\subsection{Función generatriz}
Para determinar una transformación arbitraria, necesitamos las $2 n$ funciones $\zeta^{\prime \alpha}(\zeta, t)$. Sin embargo, las transformaciones canónicas no son arbitrarias sino que cumplen propiedades restrictivas (preservan la matriz simpléctica) y, por tanto, cabe esperar que sea necesaria menos información para caracterizarlas. En esta sección, veremos que una transformación canónica queda unívocamente determinada por una función en el espacio de fases y por el tipo de transformación.

Consideremos una variedad simpléctica $(\mathscr{M}, \Omega)$ arbitraria. Utilizaremos como punto de partida las coordenadas $\xi^{\alpha}=\left(q^{a}, p_{a}\right)$ que, gracias al teorema de Darboux, sabemos que existen localmente. En estas coordenadas, la forma simpléctica tiene componentes $\breve{\Omega}_{\alpha \beta}$ y supondremos que el hamiltoniano del sistema es $H(\xi, t)$.

Resultará conveniente introducir la matriz
$$
\check{\Gamma}=\left(\begin{array}{ll}
0_{n} & \mathbb{I}_{n} \\
0_{n} & 0_{n}
\end{array}\right)
$$
en términos de la cual la matriz simpléctica es $\breve{\Omega}=\breve{\Gamma}^{\mathrm{T}}-\breve{\Gamma}$, es decir, en componentes, $\breve{\Omega}_{\alpha \beta}=-2 \breve{\Gamma}_{[\alpha \beta]}$.

Sea $\xi \rightarrow \xi^{\prime}$ una transformación canónica. En el punto $\xi$, la forma canónica es
$$
\theta=p_{a} \mathrm{~d} q^{a}=\check{\Gamma}_{\alpha \beta} \xi^{\beta} \mathrm{d} \xi^{\alpha} ;
$$
análogamente, en $\breve{\xi}^{\prime}$ tendremos
$$
\theta^{\prime}=p_{a}^{\prime} \mathrm{d} q^{\prime a}=\breve{\Gamma}_{\alpha \beta} \xi^{\prime \beta} \mathrm{d} \xi^{\prime \alpha}
$$

La forma simpléctica se preserva bajo transformaciones canónicas y, por tanto,
$$
\mathrm{d} \theta=\boldsymbol{\Omega}(\xi)=\boldsymbol{\Omega}\left(\xi^{\prime}\right)=\mathrm{d} \theta^{\prime} \quad \Rightarrow \quad \mathrm{d}\left(\theta-\theta^{\prime}\right)=0
$$

Puesto que $\theta-\theta^{\prime}$ es una forma cerrada, localmente, existe una función $F$ tal que $\theta-\theta^{\prime}=\mathrm{d} F$. Esta función $F$ puede ser escrita en términos de cualquier par de variables independientes, no necesariamente $\left(q^{a}, p_{a}\right)$ y recibe el nombre de función generatriz de la transformación canónica. Explícitamente,
$$
\begin{equation*}
\mathrm{d} F=\breve{\Gamma}_{\alpha \beta}\left(\xi^{\beta} \mathrm{d} \xi^{\alpha}-\xi^{\prime \beta} \mathrm{d} \xi^{\prime \alpha}\right)=p_{a} \mathrm{~d} q^{a}-p_{a}^{\prime} \mathrm{d} q^{\prime a} \tag{2.10}
\end{equation*}
$$

Esta ecuación es integrable localmente puesto que la condición de integrabilidad local es precisamente que la transformación sea canónica. La función generatriz queda unívocamente determinada (salvo una función aditiva del tiempo) por la transformación canónica ya que, dada esta, quedan determinadas todas sus derivadas salvo la temporal. Sin embargo, dada una cierta función generatriz, esta ecuación es una ecuación diferencial para $\xi^{\prime \alpha}$ cuya solución no es única.
$\diamond$ EJEMPLO: Las dos transformaciones canónicas $q^{\prime}=p, p^{\prime}=-q$, por un lado, $\mathrm{y} q^{\prime \prime}=p^{2} / 2, p^{\prime \prime}=-q / p$, por el otro, tienen la misma función generatriz. En efecto, mediante sustitución directa de estas expresiones en la ecuación 2.10 obtenemos
$$
\begin{array}{lll}
\mathrm{d} F^{\prime}=p \mathrm{~d} q-p^{\prime} \mathrm{d} q^{\prime}=p \mathrm{~d} q+q \mathrm{~d} p & \Rightarrow & F^{\prime}(q, p)=q p \\
\mathrm{~d} F^{\prime \prime}=p \mathrm{~d} q-p^{\prime \prime} \mathrm{d} q^{\prime \prime}=p \mathrm{~d} q+q \mathrm{~d} p & \Rightarrow & F^{\prime \prime}(q, p)=q p
\end{array}
$$

Establezcamos, a continuación, la relación entre la función generatriz $F$ y los hamiltonianos $H$ y $H^{\prime}$ asociados a las variables originales $\xi^{\alpha}$ y a las nuevas $\xi^{\prime \alpha}$, respectivamente. Consideremos, por el momento, que la función generatriz es función de las variables canónicas originales y del tiempo $F(\xi, t)$.

A partir de las ecuaciones de Hamilton en cada conjunto de variables, podemos calcular $\partial_{\alpha}\left(H^{\prime}-H\right)$. En efecto,
$$
\begin{aligned}
\partial_{\alpha} H^{\prime} & =\partial_{\alpha} \xi^{\prime \beta} \partial_{\beta}^{\prime} H^{\prime}=\partial_{\alpha} \xi^{\prime \beta} \breve{\Omega}_{\beta \gamma} \dot{\xi}^{\prime \gamma}=\partial_{\alpha} \xi^{\prime \beta} \breve{\Omega}_{\beta \gamma}\left(\partial_{\delta} \xi^{\prime \gamma} \dot{\xi}^{\delta}+\partial_{t} \xi^{\prime \gamma}\right) \\
& =\breve{\Omega}_{\alpha \delta} \dot{\xi}^{\delta}+\partial_{\alpha} \xi^{\prime \beta} \breve{\Omega}_{\beta \gamma} \partial_{t} \xi^{\prime \gamma}=\partial_{\alpha} H+\partial_{\alpha} \xi^{\prime \beta} \breve{\Omega}_{\beta \gamma} \partial_{t} \xi^{\prime \gamma}
\end{aligned}
$$
de forma que
$$
\partial_{\alpha}\left(H^{\prime}-H\right)=\partial_{\alpha} \xi^{\prime \beta} \breve{\Omega}_{\beta \gamma} \partial_{t} \xi^{\prime \gamma}
$$

En esta ecuación, aparecen dos derivadas parciales: una con respecto a las variables originales $\xi^{\alpha}$ y otra con respecto al tiempo $t$. Por tanto, para relacionar esta expresión con la función generatriz, calculemos $\partial_{t} \partial_{\alpha} F$. La ecuación 2.10, nos proporciona las derivadas parciales de la función generatriz con respecto a las variables canónicas, de forma que
$$
\partial_{t} \partial_{\alpha} F=\partial_{t}\left[\breve{\Gamma}_{\beta \gamma}\left(\xi^{\gamma} \delta_{\alpha}^{\beta}-\xi^{\prime \gamma} \partial_{\alpha} \xi^{\prime \beta}\right)\right]=-\breve{\Gamma}_{\beta \gamma} \partial_{t}\left(\xi^{\prime \gamma} \partial_{\alpha} \xi^{\prime \beta}\right)
$$

Restando ambas expresiones se obtiene 
$$
\begin{aligned}
\partial_{\alpha}\left(\partial_{t} F+H-H^{\prime}\right) & =-\breve{\Gamma}_{\beta \gamma} \partial_{t}\left(\xi^{\prime \gamma} \partial_{\alpha} \xi^{\prime \beta}\right)+\partial_{\alpha} \xi^{\prime \beta} \breve{\Omega}_{\beta \gamma} \partial_{t} \xi^{\prime \gamma}= \\
& =-\partial_{\alpha}\left(\breve{\Gamma}_{\gamma \beta} \xi^{\prime} \partial_{t} \xi^{\prime \beta}\right)
\end{aligned}
$$

Vemos que la cantidad $\partial_{t} F+H-H^{\prime}+\breve{\Gamma}_{\beta \gamma} \partial_{t} \xi^{\prime \beta} \xi^{\prime \gamma}$ es independiente de las variables canónicas y, por tanto, solo puede depender del tiempo. Esta dependencia puede absorberse en la definición de $F$ (que está determinada salvo una función aditiva del tiempo). De esta manera, hemos obtenido la relación que buscábamos entre la función generatriz y los hamiltonianos

\begin{equation*}
\partial_{t} F=H^{\prime}-H-\check{\Gamma}_{\beta \gamma} \xi^{\prime \gamma} \partial_{t} \xi^{\prime \beta}=H^{\prime}-H-p_{a}^{\prime} \partial_{t} q^{\prime a} \tag{2.11}
\end{equation*}

 Se puede demostrar que $\dot{F}=\left(p_{a} \dot{q}^{a}-H\right)-\left(p_{a}^{\prime} \dot{q}^{\prime a}-H^{\prime}\right)$. 
 
 Notar que las transformaciones canónicas - vía una transformación de Legendre- generan transformaciones en el espacio de fases en velocidades $T \mathscr{Q}$ que cambian el lagrangiano en una derivada total.
\subsection{Tipos de transformaciones canónicas}
Hemos visto en la sección anterior que, dada una transformación canónica, su función generatriz queda unívocamente determinada, si bien pueden existir varias transformaciones canónicas con la misma función generatriz. Sin embargo, es posible determinar la transformación canónica generada por una cierta función generatriz con una información adicional: el tipo de la transformación
\subsubsection{Transformaciones canónicas de tipo 1}
Supongamos que una transformación canónica $q^{\prime a}(q, p, t), p_{a}^{\prime}(q, p, t)$ es tal que el conjunto de variables $\left(q^{a}, q^{\prime a}\right)$ son independientes, es decir, que son tan buenas coordenadas del espacio de fases como $\left(q^{a}, p_{a}\right)$. Entonces, cualquier variable dinámica se puede escribir en función de ellas y, en particular, la función generatriz. Diremos que esta transformación es de tipo 1. Por ejemplo, la transformación canónica $q^{\prime}=p, p^{\prime}=-q$, es obviamente de tipo 1. Las transformaciones de tipo 1 son útiles, por ejemplo, en la teoría de Hamilton-Jacobi.

La función generatriz de tipo 1, definida mediante la expresión
$$
F_{1}\left(q, q^{\prime}, t\right)=F\left[q, p\left(q, q^{\prime}, t\right), t\right]
$$
satisface la ecuación
$$
\mathrm{d} F=\frac{\partial F_{1}}{\partial q^{a}} \mathrm{~d} q^{a}+\frac{\partial F_{1}}{\partial q^{\prime a}} \mathrm{~d} q^{\prime a}=p_{a} \mathrm{~d} q^{a}-p_{a}^{\prime} \mathrm{d} q^{\prime a}
$$
de donde se deduce inmediatamente la expresión de $p_{a}$ y $p_{a}^{\prime}$ en función de las variables independientes $\left(q^{a}, q^{\prime a}\right)$ :
$$
\begin{equation*}
p_{a}=\frac{\partial F_{1}}{\partial q^{a}}, \quad p_{a}^{\prime}=-\frac{\partial F_{1}}{\partial q^{\prime a}} \tag{2.12}
\end{equation*}
$$

Dada una función generatriz de tipo 1, estas ecuaciones proporcionan dependencias funcionales únicas para $p_{a}$ y $p_{a}^{\prime}$. Si $F_{1}$ satisface la condición de invertibilidad
$$
\operatorname{det}\left(\partial^{2} F_{1} / \partial q^{a} \partial q^{\prime b}\right) \neq 0
$$
la primera ecuación se puede invertir para obtener $q^{\prime a}(q, p, t)$ y, si introducimos esta expresión en la segunda, obtenemos $p_{a}^{\prime}(q, p, t)$. De esta forma, cada función $F_{1}\left(q, q^{\prime}, t\right)$ invertible caracteriza una transformación canónica de tipo 1 y viceversa.

El hamiltoniano en las nuevas variables se obtiene directamente mediante el uso de la ecuación 2.11:
$$
H^{\prime}=H+\partial_{t} F+p_{a}^{\prime} \partial_{t} q^{\prime a}=H+\partial_{t} F-\frac{\partial F_{1}}{\partial q^{\prime a}} \partial_{t} q^{\prime a}
$$

Los dos últimos términos se reducen a $\partial_{t} F_{1}$ como es fácil de ver si introducimos las dependencias en las variables originales $q$ y $p$ explícitamente, es decir, si escribimos $F(q, p, t)=F_{1}\left[q, q^{\prime}(q, p, t), t\right]$. Entonces,
$$
\begin{aligned}
\left.\partial_{t} F_{1}\left(q, q^{\prime}, t\right)\right|_{q, q^{\prime}} & =\left.\partial_{t} F_{1}\left[q, q^{\prime}(q, p, t), t\right]\right|_{q, p}-\left.\frac{\partial F_{1}\left(q, q^{\prime}, t\right)}{\partial q^{\prime a}} \partial_{t} q^{\prime a}(q, p, t)\right|_{q, p} \\
& =\partial_{t} F(q, p, t)-\frac{\partial F_{1}\left(q, q^{\prime}, t\right)}{\partial q^{\prime a}} \partial_{t} q^{\prime a}(q, p, t)
\end{aligned}
$$

Por tanto, los hamiltonianos en las nuevas variables y en las antiguas satisfacen la relación
$$
\begin{equation*}
H^{\prime}=H+\partial_{t} F_{1} \tag{2.13}
\end{equation*}
$$

No todas las transformaciones canónicas son de tipo 1. En efecto, la transformación canónica $q^{\prime}=q, p^{\prime}=q+p$ no es de tipo 1 ya que $q$ y $q^{\prime}$ no son independientes.
\subsubsection{Transformaciones canónicas de tipo 2}
Diremos que una transformación canónica es de tipo 2 si las variables $q^{a}$ y $p_{a}^{\prime}$ son independientes $y$, por tanto, proporcionan coordenadas del espacio de fases. Como ejemplo, la transformación anterior, $q^{\prime}=q$, $p^{\prime}=q+p$, es de tipo 2. Las transformaciones de tipo 2 son útiles, por ejemplo, en la teoría de perturbaciones canónicas y en la construcción de variables de acción-ángulo.

Para obtener la función generatriz de tipo 2 , seguimos un procedimiento similar al empleado en el caso anterior. La función generatriz satisface la ecuación
$$
\mathrm{d} F=p_{a} \mathrm{~d} q^{a}-p_{a}^{\prime} \mathrm{d} q^{\prime a}=p_{a} \mathrm{~d} q^{a}+q^{\prime a} \mathrm{~d} p^{\prime a}-\mathrm{d}\left(p_{a}^{\prime} q^{\prime a}\right)
$$

Como consecuencia, la función generatriz de tipo 2, definida mediante la expresión
$$
F_{2}\left(q, p^{\prime}, t\right)=F\left[q, p\left(q, p^{\prime}, t\right), t\right]+p_{a}^{\prime} q^{\prime a}\left(q, p^{\prime}, t\right),
$$
satisface la ecuación
$$
\mathrm{d} F_{2}=\mathrm{d} F+\mathrm{d}\left(p_{a}^{\prime} q^{\prime a}\right)=p_{a} \mathrm{~d} q^{a}+q^{\prime a} \mathrm{~d} p^{\prime a}=\frac{\partial F_{2}}{\partial q^{a}} \mathrm{~d} q^{a}+\frac{\partial F_{2}}{\partial p_{a}^{\prime}} \mathrm{d} p_{a}^{\prime}
$$
de donde obtenemos las relaciones
$$
\begin{equation*}
p_{a}=\frac{\partial F_{2}}{\partial q^{a}}, \quad q^{\prime a}=\frac{\partial F_{2}}{\partial p_{a}^{\prime}} . \tag{2.14}
\end{equation*}
$$

Dada una función generatriz $F_{2}$ de tipo 2 que satisfaga la condición de invertibilidad
$$
\operatorname{det}\left(\partial^{2} F_{2} / \partial q^{a} \partial p_{b}^{\prime}\right) \neq 0
$$
la primera ecuación permite obtener de forma única $p_{a}^{\prime}(q, p, t)$ que, introducida en la segunda ecuación, proporciona $q^{\prime a}(q, p, t)$. De esta forma, cada función $F_{2}\left(q, p^{\prime}, t\right)$ invertible caracteriza una transformación canónica de tipo 2 y viceversa.
$\diamond$ EJERCICIO: Demostrar que el hamiltoniano en las nuevas variables se puede obtener a partir del hamiltoniano en las variables originales y la función generatriz de tipo 2 mediante la expresión
$$
H^{\prime}=H+\partial_{t} F_{2}
$$
$\diamond$ EJEMPLO: Consideremos una transformación de contacto dependiente del tiempo definida por $q^{\prime}=f(q, t)$. Para caracterizar completamente esta transformación debemos encontrar el nuevo momento canónico $p^{\prime}$. Para ello, podemos considerarla de tipo 2. Entonces la función generatriz $F_{2}\left(q, p^{\prime}, t\right)$ debe satisfacer las ecuaciones 2.14. Integrando la segunda ecuación vemos que $F_{2}=p^{\prime} f(q, t)+g(q, t)$ donde $g(q, t)$ es una función arbitraria. Si escogemos, por ejemplo, $g=0$, la primera ecuación nos permite obtener $p^{\prime}=p / \partial_{q} f(q, t)$.
Entonces, el nuevo hamiltoniano será la suma del antiguo y del término $\partial_{t} F_{2}=p^{\prime} \partial_{t} f(q, t)$.
\subsubsection{Transformaciones canónicas de tipo 3, 4 y otros}
Diremos que una transformación canónica es de tipo 3 si las variables $p_{a}$ y $q^{\prime a}$ son independientes y , por tanto, proporcionan coordenadas del espacio de fases.

Diremos que una transformación canónica es de tipo 4 si las variables $p_{a}$ y $p_{a}^{\prime}$ son independientes y, por tanto, proporcionan coordenadas del espacio de fases.
$\diamond$ EJERCICIO: Estudiar las transformaciones canónicas de tipos 3 y 4.
Por último, queda señalar que no todas las transformaciones canónicas son de estos tipos y que existen $2^{n}$ tipos diferentes resultantes de las $2^{n}$ posibles maneras de elegir las variables independientes $\left(q^{a}, q^{\prime i}, p_{j}^{\prime}\right)$ con $i=1, \ldots, m$ y $j=m+1, \ldots n$, para todo $m \leq n$ 
\subsubsection{Resumen y ejemplo}
Los tipos de^\top formaciones tienen las siguientes características: 

\begin{enumerate}
\item \textbf{Tipo 1.} $(q, Q)$ son independientes.
\item \textbf{Tipo 2.} $(q, P)$ son independientes.
\item \textbf{Tipo 3.} $(p, Q)$ son independientes.
\item \textbf{Tipo 4.} $(p, P)$ son independientes. 
\end{enumerate}



\begin{example}
  Por ejemplo, la transformación $Q=q, P=q+p$ en dos grados de libertad es canónica y pertenece simultáneamente a los Tipos 2, 3 y 4, pero no al Tipo 1. Las transformaciones canónicas (CT) de otros tipos también pueden clasificarse mediante funciones de $2n+1$ variables.  

Como ejemplo, considere el Tipo 2. Escriba $f^{2}(q, P, t)=F(q, p(q, P, t), t)$. Entonces, si todo en la Ec. (5.110) se escribe en términos de $(q, P, t)$ y $dF$ se iguala a $df^{2}$, esa ecuación se convierte en:  

$$
\begin{aligned}
d F & =p_{\alpha} d q^{\alpha}-P_{\beta}\left(\frac{\partial Q^{\beta}}{\partial q^{\alpha}} d q^{\alpha}+\frac{\partial Q^{\beta}}{\partial P_{\alpha}} d P_{\alpha}\right) \\
& =d f^{2} \equiv \frac{\partial f^{2}}{\partial q^{\alpha}} d q^{\alpha}+\frac{\partial f^{2}}{\partial P_{\alpha}} d P_{\alpha}.
\end{aligned}
$$  

Al igualar los coeficientes de $d q^{\alpha}$ y $d P_{\alpha}$ se obtiene:  

$$
\frac{\partial f^{2}}{\partial q^{\alpha}}=p_{\alpha}-P_{\beta} \frac{\partial Q^{\beta}}{\partial q^{\alpha}} \quad \text{y} \quad \frac{\partial f^{2}}{\partial P_{\alpha}}=-P_{\beta} \frac{\partial Q^{\beta}}{\partial P_{\alpha}}
$$  

o bien:  


\begin{equation*}
p_{\alpha}=\frac{\partial F^{2}}{\partial q^{\alpha}} \quad \text{y} \quad Q^{\alpha}=\frac{\partial F^{2}}{\partial P_{\alpha}} \tag{5.123}
\end{equation*}
 

donde  

$$
\begin{aligned}
F^{2}(q, P, t) & =f^{2}(q, P, t)+P_{\alpha} Q^{\alpha}(q, P, t) \\
& =F(q, Q(q, P, t), t)+P_{\alpha} Q^{\alpha}(q, P, t).
\end{aligned}
$$  

Para obtener ecuaciones explícitas de la propia CT, el primer conjunto de ecuaciones (5.123) se invierte (de nuevo, hay una condición de invertibilidad) para obtener $P_{\beta}(q, p, t)$, y esto se inserta en la segunda mitad para obtener una expresión explícita de $Q^{\beta}(q, p, t)$. Por lo tanto, las CT de Tipo 2, al igual que las de Tipo 1, también pueden clasificarse mediante funciones de $2n+1$ variables. [Tenga en cuenta que $F^{2}$ no es simplemente la función generadora $F$ escrita en las variables $(q, P, t)$.] Es sencillo mostrar que, de manera análoga al Tipo 1,  


\begin{equation*}
K=H+\frac{\partial F^{2}}{\partial t} \tag{5.124}
\end{equation*}
\end{example}

\begin{example}
  
(a) En una libertad, considera la transformación compleja  
$$
Q=\frac{m \omega q+i p}{\sqrt{2 m \omega}}, \quad P=i \frac{m \omega q-i p}{\sqrt{2 m \omega}} \equiv i Q^{*} \tag{5.125}
$$  
donde el asterisco denota el conjugado complejo. Demuestra que esta transformación es una TC de Tipo 1 y encuentra su función generadora de Tipo 1. Aplica esta TC al oscilador armónico cuya hamiltoniana es  
$$
H=\frac{1}{2} m \omega^{2} q^{2}+\frac{p^{2}}{2 m}
$$  
Es decir, encuentra la nueva hamiltoniana, resuelve el movimiento en términos de $(Q, P)$ y vuelve a transformarla a $(q, p)$.

(b) Haz lo mismo para la transformación compleja  
$$
Q_{\mathrm{t}}=\frac{m \omega q+i p}{\sqrt{2 m \omega}} e^{z \omega t}, \quad P_{\mathrm{t}}=i \frac{m \omega q-i p}{\sqrt{2 m \omega}} e^{-i \omega t} \equiv i Q_{\mathrm{t}}^{*} \tag{5.126}
$$  
(el subíndice $t$ se usa para evitar confusión con la Ec. (5.125) y para enfatizar la dependencia temporal).
\hrulefill
(a) La canonicidad se puede comprobar calculando que $\{Q, P\}=1$. (Es importante en mecánica cuántica que $\left\{Q^{*}, Q\right\}=-i$). A partir de la dependencia de $Q$ tanto en $q$ como en $p$, se deduce que $Q$ y $q$ son independientes, por lo que la TC es de Tipo 1 (por el mismo argumento, también es de Tipos 2, 3 y 4). Las ecuaciones (5.121) se pueden usar para encontrar la función generadora de Tipo 1 $F^{\mathrm{i}}(q, Q)$:  
$$
\begin{aligned}
& p \equiv i(m \omega q-\sqrt{2 m \omega} Q)=\frac{\partial F^{1}}{\partial q}, \\
& P \equiv i(\sqrt{2 m \omega} q-Q)=-\frac{\partial F^{1}}{\partial Q}.
\end{aligned}
$$  
La solución (hasta una constante aditiva) es  
$$
F^{1}(q, Q)=i\left(\frac{Q^{2}}{2}-\sqrt{2 m \omega} q Q+\frac{m \omega q^{2}}{2}\right) \tag{5.127}
$$  

Ahora considera la hamiltoniana del oscilador armónico. Debido a que la TC de la Ec. (5.125) es independiente del tiempo, la nueva hamiltoniana $K$ es simplemente $H$ expresada en las nuevas coordenadas:  
$$
K=H=-i \omega Q P \tag{5.128}
$$  

Las ecuaciones canónicas de Hamilton son  
$$
\dot{Q}=-i \omega Q, \quad \dot{P}=i \omega P.
$$  

Las ecuaciones para $Q(t)$ y $P(t)$ parecen estar desacopladas, pero como $P=i Q^{*}$, el movimiento de $P$ se puede encontrar a partir del movimiento de $Q$. Las soluciones son inmediatas:  
$$
Q(t)=Q_{0} e^{-i \omega t}, \quad P(t)=P_{0} e^{i \omega t} \tag{5.129}
$$  
donde $Q_{0}$ y $P_{0}$ son constantes complejas determinadas por las condiciones iniciales (cada ecuación de primer orden necesita solo una condición inicial). Estas son dos constantes complejas, por lo que las condiciones iniciales parecen depender de cuatro constantes reales. Pero $P=i Q^{*}$, así que $P_{0}=i Q_{0}^{*}$; es decir, las condiciones iniciales dependen solo de las dos constantes reales de $Q_{0}$: si $Q_{0}=\left(m \omega q_{0}+i p_{0}\right) / \sqrt{2 m \omega}$, entonces $P_{0}=i\left(m \omega q_{0}-i p_{0}\right) / \sqrt{2 m \omega}$. El movimiento $(q, p)$ se puede obtener invirtiendo la Ec. (5.125) o tomando las partes real e imaginaria:  
$$
\left.\begin{array}{rl}
q(t) & =\sqrt{\frac{2}{m \omega}} \mathfrak{R}\{Q(t)\} \equiv \sqrt{\frac{1}{2 m \omega}}\left[Q(t)+Q^{*}(t)\right] \\
& =\sqrt{\frac{1}{2 m \omega}}\left(Q_{0} e^{-t \omega t}+Q_{0}^{*} e^{i \omega t}\right) \equiv q_{0} \cos \omega t+\frac{p_{0}}{m \omega} \sin \omega t  \tag{5.130}\\
p(t) & =\sqrt{2 m \omega} \Im\{Q(t)\} \equiv-i \sqrt{\frac{m \omega}{2}}\left[Q(t)-Q^{*}(t)\right] \\
& =-i \sqrt{\frac{m \omega}{2}}\left(Q_{0} e^{-t \omega t}-Q_{0}^{*} e^{t \omega t}\right) \equiv-m \omega q_{0} \sin \omega t+p_{0} \cos \omega t.
\end{array}\right\}
$$  

**Nota:** Es imposible obtener un lagrangiano a partir de la hamiltoniana de la Ec. (5.128), porque no satisface la condición Hessiana (ver Problema 5; en este caso de una libertad, la condición es $\partial^{2} H / \partial^{2} Q \neq 0$). Por lo tanto, la TC compleja de este ejemplo hace imposible obtener el formalismo lagrangiano para la coordenada generalizada $Q$. No obstante, esta transformación se usa a menudo en mecánica cuántica.

**(b)** Ahora consideramos la transformación de la Ec. (5.126). Nuevamente, la canonicidad está garantizada por el hecho de que $\left\{Q_{t}, P_{t}\right\}=1$, y la TC es de Tipo 1 por el mismo argumento que en la parte (a). Las ecuaciones (5.121) ahora son  
$$
p=i\left(m \omega q-\sqrt{2 m \omega} Q_{\mathrm{e}} e^{-i \omega t}\right)=\frac{\partial F_{t}}{\partial q},
$$  
cuya solución (hasta una función aditiva de $t$) es  
$$
F_{\mathrm{t}}\left(q, Q_{\mathrm{t}}, t\right)=i\left(\frac{Q_{\mathrm{t}}}{2} e^{-2 t \omega t}-\sqrt{2 m \omega} q Q_{\mathrm{t}} e^{-t \omega t}+\frac{m \omega q^{2}}{2}\right) \tag{5.131}
$$  

Esta TC también se puede invertir calculando las partes real e imaginaria. Los resultados son  
$$
\left.\begin{array}{l}
q(t)=\sqrt{\frac{1}{2 m \omega}}\left\{Q_{\mathrm{t}}(t) e^{-i \omega t}-i P_{\mathrm{t}}(t) e^{i \omega t}\right\}  \tag{5.132}\\
p(t)=-i \sqrt{\frac{m \omega}{2}}\left\{Q_{\mathrm{t}}(t) e^{-i \omega t}+i P_{\mathrm{t}}(t) e^{i \omega t}\right\}.
\end{array}\right\}
$$  

Aunque la hamiltoniana original se da en las nuevas coordenadas por el análogo de la Ec. (5.128), es decir, $H=-i \omega Q_{\mathrm{t}} P_{\mathrm{t}}$, la función generadora ahora depende de $t$, por lo que la nueva hamiltoniana es  
$$
\begin{aligned}
K_{\mathrm{t}} & =H+\frac{\partial F_{\mathrm{t}}}{\partial t}=H+\omega Q_{\mathrm{t}} e^{-2 i \omega t}-\omega \sqrt{2 m \omega} q Q_{\mathrm{t}} e^{-t \omega t} \\
& =0.
\end{aligned}
$$  

La nueva hamiltoniana desaparece completamente. ¡El movimiento es ahora trivial! $Q_{\mathrm{t}}=$ const. $\equiv Q_{0}, P_{\mathrm{t}}=$ const. $\equiv P_{0}$. Esto da la solución correcta para $(q, p)$ comparando con (5.130).  

Nota: Aunque este

 sistema no tiene un hamiltoniano explícito, sí tiene un lagrangiano explícito.
\end{example}
\section{Invariantes}
\subsection{Teorema de Liouville}
\subsection{Distribuciones estadísticas}
Consideremos un sistema de $N$ partículas con $n$ grados de libertad del que nos interesa no la dinámica de cada una de las partículas sino el comportamiento promediado (como ocurre, por ejemplo, en el caso de un gas en una caja para el que $N \sim 10^{28}$ ). Entonces podemos utilizar una distribución de densidad en el espacio de fases $\rho(\xi, t)$ que asigna un número de partículas a cada elemento de volumen infinitesimal en el espacio de fases para describir el sistema, de forma que
$$
\int \rho v=\int \rho(\xi, t) \mathrm{d}^{2 n} \xi=N
$$

Si tenemos un sistema del cual no conocemos precisamente su estado, también podemos utilizar una distribución de densidad $\rho(\xi, t)$ para describirlo. En este caso, $\rho$ se interpreta como la probabilidad de que el sistema se halle en un estado contenido en un volumen infinitesimal en el espacio de fase y satisface
$$
\int \rho v=\int \rho(\xi, t) \mathrm{d}^{2 n} \xi=1
$$

En ambos casos, puesto que ni las partículas ni la probabilidad se crean o se destruyen, se satisface
$$
0=\frac{\mathrm{d}}{\mathrm{~d} t} \int_{\mathscr{U}} \rho v=\int_{\mathscr{U}} \frac{\mathrm{d} \rho}{\mathrm{~d} t} v
$$
puesto que $v$ es invariante. La región $\mathscr{U}$ del espacio de fases que estamos considerando es arbitraria y, por tanto, podemos concluir que $\dot{\rho}=0$. Como ocurre con cualquier variable dinámica,
$$
\dot{\rho}=\{\rho, H\}+\partial_{t} \rho
$$
por lo que $\rho$ satisface la ecuación de evolución de Liouville:
$$
\partial_{t} \rho=-\{\rho, H\}
$$