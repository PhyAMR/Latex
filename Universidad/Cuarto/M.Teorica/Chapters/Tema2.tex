\setchapterpreamble[u]{\margintoc}
\chapter{Mecánica Hamiltoniana}
\labch{Part}


\section{Motivación de sistemas Hamiltonianos}

Uno podría imaginar, de forma más general, la construcción de una variedad diferente $\mathbb{F}$ añadiendo un tipo diferente de fibra a cada punto de $\mathbb{Q}$ (el resultado sería un haz de fibras sobre $\mathbb{Q}$ ). Si se pudiera hacer que $\mathbb{F}$ llevara de alguna manera la dinámica (para separar las trayectorias), y si las trayectorias sobre $\mathbb{F}$ pudieran proyectarse hasta las trayectorias físicas sobre $\mathbb{Q}$, entonces $\mathbb{F}$ serviría para nuestros propósitos tan bien como $\mathbf{TQ}$. En el formalismo hamiltoniano esto es precisamente lo que se hace: $\mathbb{F}$ es el haz cuyas fibras consisten en los posibles valores del momento $p$, una variedad cuyos puntos son de la forma ( $q, p$ ). Se denomina haz cotangente o colector (o espacio) de fase $\mathbf{T}^{*}. \mathbf{Q}$; sus fibras no son los espacios tangentes a los puntos de $\mathbb{Q}$, sino sus espacios duales (definidos en el apartado 3.4.1), llamados espacios cotangentes. Estas afirmaciones se aclararán a medida que avancemos.

\subsection{Ecuaciones de Hamilton}

A continuación vamos a derviar las ecuaciones de Hamilton a partir de lo que conocemos, es decir, el lagrangiano. Para ello vamos a hacer uso de la transformada de Legendre, que ya vimos en Termodinámica de segundo (Ver apéndice \ref{Legendre}). Lo que vamos a ver es análogo a lo del apéndice. 

El momento generalizado $p_{\alpha}$ conjugado con $q^{\alpha}$ ha sido definido (Sección 2.2.1) como
\begin{equation*}
  p_{\alpha}=\frac{\partial L}{\partial \dot{q}^{\alpha}} \tag{5.1}
\end{equation*}

donde $L: \mathbb{Q} \times \mathbb{R} \rightarrow \mathbb{R}$ es, como siempre, la función lagrangiana. (El dominio de $L$ está formado por $\mathbf{T}$ y el tiempo $t \in \mathbb{R}$). Por tanto, las ecuaciones de E-L y las ecuaciones diferenciales para $q^{\alpha}$ pueden escribirse de la forma

\begin{DispWithArrows}[format=ll, displaystyle]
  \frac{d p_{\alpha}}{d t} & =\frac{\partial L}{\partial q^{\alpha}}  \tag{5.2a}\\
  \frac{d q^{\alpha}}{d t} & =\dot{q}^{\alpha} \tag{5.2b}
\end{DispWithArrows}

\todo{Ahora no hay ecuaciones de segundo grado, solo $2n$ ecuaciones de primer grado}
Excepto por el hecho de que $L$ y $\dot{q}^{\alpha}$ son todas funciones de ( $q, \dot{q}, t$ ) ($\dot{q}^{\alpha}$ trivialmente) en lugar de $(q, p, t)$, estas ecuaciones tienen la forma simple deseada. Para obtener precisamente la forma deseada sólo se requiere que los lados derechos se escriban en términos de $(q, p, t)$.

En principio, esto es fácil de hacer para la Ec. (5.2b): escribir el $\dot{q}^{\alpha}$ en el lado derecho como funciones de $(q, p, t)$ invirtiendo (5.1). Supongamos que esto se puede hacer (vamos a encontrar las condiciones necesarias para ello más adelante), y llamar a las funciones resultantes $\dot{q}^{\alpha}(q, p, t)$. El lado derecho de la Ec. (5.2a) puede transformarse de forma similar: en todos los $\partial L / \partial q^{\beta}$ sustituir los $\dot{q}^{\alpha}$ dondequiera que aparezcan por $\dot{q}^{\alpha}(q, p, t)$. Sería más fácil hacer primero la sustitución en $L$, convirtiendo $L$ en una función de ( $q, p, t$ ), y luego tomar las derivadas. Esto requiere cuidado, sin embargo, porque la derivada parcial de $L(q, \dot{q}, t)$ con respecto a $q^{alpha}$ no es la misma que la de \todo{$\hat{L}$ es igual que $L$, pero en este caso $\dot{q}$ no es un parámetro libre, si no que es función de $(q,p,t)$} $\hat{L}(q, p, t) \equiv L(q, \dot{q}(q, p, t), t)$, ya que $L$ es una función de $(q, \dot{q}, t)$, y $\hat{L}$ es una función de $(q, p, t)$. Al tomar derivadas de $L$, las $\dot{q} s$ se mantienen fijas, mientras que en $\hat{L}$ las $p$ s se mantienen fijas.
fijas. Comparamos estos dos conjuntos de derivadas:

\[\frac{\partial \hat{L}}{\partial q^{\alpha}}=\frac{\partial L}{\partial q^{\alpha}}+\frac{\partial L}{\partial \dot{q}^{\beta}} \frac{\partial \dot{q}^{\beta}}{\partial q^{\alpha}}=\frac{\partial L}{\partial q^{\alpha}}+p_{\beta} \frac{\partial \dot{q}^{\beta}}{\partial q^{\alpha}}\]

donde hemos utilizado (5.1). El último término del lado derecho es la derivada de la función $\dot{q}^{\beta}(q, p, t)$. Ahora ponga todas las funciones de $(q, p, t)$ en un lado de la ecuación para obtener

\[\frac{\partial L}{\partial q^{\alpha}}=\frac{\partial}{\partial q^{\alpha}}\left[\hat{L}(q, p, t)-p_{\beta} \dot{q}^{\beta}(q, p, t)\right]\]

También será necesaria la derivada de $\hat{L}$ con respecto a $p_{\alpha}$. Esto es

\[\frac{\partial \hat{L}}{\partial p_{\alpha}}=\frac{\partial L}{\partial \dot{q}^{\beta}} \frac{\partial \dot{q}^{\beta}}{\partial p_{\alpha}}=p_{\beta} \frac{\partial \dot{q}^{\beta}}{\partial p_{\alpha}}\]

o (de nuevo poner todas las funciones de $(q, p, t)$ en un lado)

\begin{equation*}
  \frac{\partial}{\partial p_{\alpha}}\left[\hat{L}(q, p, t)-p_{\beta} \dot{q}^{\beta}(q, p, t)\right]=-\dot{q}^{\alpha} \tag{5.3}
  \end{equation*}

  Esta ecuación parece la solución que se obtendría para $\dot{q}^{text {ox }}$ invirtiendo (5.1), pero en realidad no lo es, ya que el lado izquierdo depende de $\dot{q}^{\beta}$. En su lugar es una ecuación diferencial para $\dot{q}^{\alpha}$ como una función de $(q, p, t)$.

  La función 
  
  \begin{equation*}
    H(q, p, t) \equiv p_{\beta} \dot{q}^{\beta}(q, p, t)-\hat{L}(q, p, t) \tag{5.4}
    \end{equation*}

aparece en las ecuaciones (5.3) y (5.4). Esta función tan importante se denomina función Hamiltoniana, o simplemente Hamiltoniana.


Con la ayuda del Hamiltoniano, las Ecs. (5.2a, b) pueden ahora escribirse enteramente en términos de $(q, p, t)$ :


\[\left.\begin{array}{l}
  \dot{q}^{\beta}=\frac{\partial H}{\partial p_{\beta}}  \tag{5.5}\\
  \dot{p}_{\beta}=-\frac{\partial H}{\partial q^{\beta}}
  \end{array}\right\}\]

Se denominan \textit{ecuaciones canónicas de Hamilton} \sidenote{El término «canónico» significa estándar o convencional (según los cánones). Se utiliza técnicamente para los sistemas dinámicos, en particular en el formalismo hamiltoniano.}. Son las ecuaciones del movimiento en el formalismo hamiltoniano. Su solución da expresiones locales para las trayectorias
$(q(t), p(t))$ en $\mathbf{T}^{*} \mathbb{Q}$, la multiplicidad $(q, p)$. Esta variedad se denomina haz cotangente (que se discutirá en la Sección 5.2). Las proyecciones de estas trayectorias hacia $\mathbb{Q}$ (esencialmente ignorando la parte $p(t)$) son las trayectorias del colector de configuración $q(t)$. Dejamos para la Sección 5.2 una discusión más detallada de $\mathbf{T}^{*} \mathbb{Q}$ y una explicación de la diferencia entre éste y $\mathbf{T Q}$.


\subsubsection{Energía}

Es de destacar que el Hamiltoniano es la variable dinámica que hemos llamado $E$, escrito ahora en términos de $(q, p)$ en lugar de $(q, \dot{q})$. Recordemos que para muchos sistemas dinámicos $E$ es la energía, y por tanto cuando $L=T-V$ el Hamiltoniano es $H=T+V$ expresado en términos de $(q, p)$. En los casos comunes en que $V$ es independiente de la $\dot{q}^{\beta}$ sólo es necesario escribir la energía cinética $T$ en términos de $(q, p)$. Sin embargo, no siempre es tan sencillo, como se verá en el ejemplo práctico 5.1 del Hamiltoniano para una partícula cargada en un campo electromagnético.

En el formalismo lagrangiano teniamos la cantidad conservada, asociada en algunos casos a la energía, que deonotabamos como $h(q,\dot{q},t)$ con la forma 
\[h(q,\dot{q},t)=p_{\alpha}\dot{q}^{\alpha}-L\]
En el formalismo Hamiltoniano, esta cantidad coincide con el Hamiltoniano en si, es decir,

\[h(q,\dot{q},t)=p_{\alpha}\dot{q}^{\alpha}-L=H \text{ con }\dot{q}^{\alpha}(q,p)\]

\begin{proof}
  Esto se puede demostrar sabiendo que $\pdv{H}{t}=-\pdv{L}{t}$ como 
  \[\dv{H}{t}=\overbrace{\pdv{H}{q^{a}}\pdv{q^{a}}}^{-p_{a}\dot{q}^{a}} + \overbrace{\pdv{H}{p_{a}}\pdv{}{}}^{\dot{q}^{a}p_{a}}+\pdv{H}{t}\Rightarrow \dv{H}{t}=\pdv{H}{t}=0 \]
\end{proof}



\subsubsection{Receta Para construir el Hamiltoniano}

La receta para escribir las ecuaciones de Hamilton para un sistema cuya Lagrangiana es conocida es entonces la siguiente\sidenote{Dentro de cada apartado pondré como explico Prado cada paso para un \textbf{caso de interés} que ya desarrolllamos en su momento es cuando el Lagrangiano tiene la forma \\
\[L=\underbrace{\frac{1}{2}T_{ab}(q,t) \dot{q}^{a}\dot{q}^{b}}_{L_{2}}+\underbrace{A_{a}(q, t)\dot{q}^{a}}_{L_{1}}-\underbrace{V(q,t)}_{L_{0}}\],\\ es decir
$L=L_{2}+L_{1}+L_{0}$}:
\begin{itemize}
  \item Utilizar (5.1) para calcular $p_{\alpha}$ en términos de $(q, \dot{q}, t)$.
  \item Invertir estas ecuaciones para obtener el $\dot{q}^{\alpha}$ en términos de $(q, p, t)$.
  \item Insertar el $\dot{q}^{{\alpha}}(q, p, t)$ en la expresión de $L$ para obtener $\hat{L}$.
  \item Utilizar (5.4) para obtener una expresión explícita para $H(q, p, t)$.
  \item Tomando las derivadas de $H$, escribir (5.5).
\end{itemize}

La receta funciona sólo si el paso $i i$ es posible, es decir, sólo si la Ecs. (5.1) se puede resolver de forma única para el $\dot{q}^{\alpha}$. De acuerdo con el teorema de la función implícita una condición necesaria y suficiente para la invertibilidad localmente es que el hessiano $\partial^{2} L / \partial \dot{q}^{\alpha} \partial \dot{q}^{\beta}$ sea no singular. Nos encontramos con el mismo requisito en la sección 2.2.2. Como regla general supondremos que se cumple, aunque habrá ocasiones en que falle en subconjuntos pequeños.

Las ecuaciones canónicas de Hamilton se han derivado de las ecuaciones de Euler-Lagrange, que a su vez se han derivado de las ecuaciones de movimiento de Newton. Hasta ahora el único método que tenemos para hallar la función hamiltoniana es pasando por la lagrangiana, pero a menudo es tan sencillo desde cero escribir una como la otra. Más adelante hablaremos de ello.



\begin{example}[Dado el lagrangiano: $L=\frac{1}{2} m\left[\dot{r}^{2}+r^{2} \dot{\theta}^{2}+\left(r^{2} \sin ^{2} \theta\right) \dot{\phi}^{2}\right]-V(r) $ \\(a) Obtenga el Hamiltoniano y (b) escriba las ecuaciones canónicas de Hamilton para una partícula cargada en un campo electromagnético. Proyectarlas hasta $\mathbb{Q}$ y demostrar que son las ecuaciones usuales para una partícula en un campo electromagnético.]

  Según la Ecuación (5.1): 
  \begin{DispWithArrows}[displaystyle, format=c] 
    p_{r}=m \dot{q}^{\gamma}+e A_{\gamma} \tag{5.6} 
  \end{DispWithArrows} 
  que es el momento generalizado canónicamente conjugado a $q^{\gamma}$. El momento generalizado no es el momento dinámico $m \dot{q}^{\gamma}$, sino que tiene una contribución que proviene del campo electromagnético.

  Para obtener el hamiltoniano, resuelve esta ecuación para $\dot{q}^{\gamma}$ e insértalo en la Ecuación (5.4). El resultado es (usamos $\delta^{\alpha \beta}$ para forzar que las sumas sean sobre pares superíndice-subíndice): \begin{DispWithArrows}[displaystyle, format=c] 
    H(q, p, t)=\frac{1}{2 m} \delta^{\alpha \gamma}\left\{p_{\alpha}-e A_{\alpha}(q, t)\right\}\left\{p_{\gamma}-e A_{\gamma}(q, t)\right\}+e \varphi(q, t) \tag{5.7} 
  \end{DispWithArrows}
  
  (b) Las ecuaciones canónicas obtenidas de este hamiltoniano son: 
  
  \begin{DispWithArrows}[format=c, displaystyle]
    \dot{q}^{\beta}=\frac{1}{m} \delta^{\alpha \beta}\left\{p_{\alpha}-e A_{\alpha}(q, t)\right\}, \tag{5.8a} \\
     \dot{p}{\beta}=\frac{e}{m} \delta^{\alpha \gamma} \frac{\partial A{\alpha}}{\partial q^{\beta}}\left\{p_{\gamma}-e A_{\gamma}(q, t)\right\}-\frac{\partial \varphi}{\partial q^{\beta}}. \tag{5.8b}
  \end{DispWithArrows}
  
  Las ecuaciones de movimiento en $\mathbb{Q}$ se obtienen eliminando $p_{\beta}$ de las Ecuaciones (5.8a, b). Esto se hace tomando las derivadas temporales de $\dot{q}^{\alpha}$ en (5.8a) y reemplazando los $\dot{p}{\alpha}$ que aparecen por sus expresiones en (5.8b). La Ecuación (5.8a) da como resultado: \begin{DispWithArrows}[displaystyle, format=c] \ddot{q}^{\beta}=\frac{1}{m} \delta^{\alpha \beta}\left(\dot{p}{\alpha}-e \partial_{y} A_{\alpha} \dot{q}^{\gamma}-e \partial_{y} A_{\alpha}\right) \end{DispWithArrows}
  
  donde $\partial_{\gamma}=\partial / \partial q^{\gamma}$ y $\partial_{t}=\partial / \partial t$. Ahora, se usa la Ecuación (5.8b) para eliminar $\dot{p}{\alpha}$, pero el $p_{y}$ en el lado derecho debe ser reemplazado por su expresión en términos de $(q, \dot{q})$, es decir, por la Ecuación (5.6). Entonces, la ecuación para $\ddot{q}^{\beta}$ se convierte en (aquí ya no nos preocupamos de si los índices son superíndices o subíndices): 
  \begin{DispWithArrows}[displaystyle, format=c] 
    m \ddot{q}^{\beta}=e\left\{\left(\partial_{\beta} A_{\gamma}-\partial_{\gamma} A_{\beta}\right) \dot{q}^{\gamma}-\partial_{t} A_{\beta}-\partial_{\beta} \varphi\right\} \tag{5.9} 
  \end{DispWithArrows}
  
  Se deduce de la Sección 2.2.4 que $\partial_{\beta} A_{\gamma}-\partial_{y} A_{\beta}=\epsilon_{\beta \gamma \alpha} B_{\alpha}$ y $-\partial_{\psi} A_{\alpha}-\partial_{\alpha} \varphi=E_{\alpha}$, donde $\mathbf{B}=\left(B_{1}, B_{2}, B_{3}\right)$ es el campo magnético y $\mathbf{E}=\left(E_{1}, E_{2}, E_{3}\right)$ es el campo eléctrico (no debe confundirse con la energía). Así que: 
  \begin{DispWithArrows}[displaystyle, format=c] 
    m \ddot{q}^{\beta}=e\left\{\epsilon_{\gamma \alpha \beta} \dot{q}^{\gamma} B_{\alpha}+E_{\beta}\right\} \end{DispWithArrows}
  
  lo cual es la forma en componentes de la fuerza de Lorentz $\mathbf{F} \equiv m \mathbf{a}=e(\mathbf{v} \wedge \mathbf{B}+\mathbf{E})$, que es el resultado requerido.
  
\end{example}


\begin{example}[Dado el lagrangiano: \[L(q, \dot{q}, t)=\frac{1}{2} m \dot{q}^{\beta} \dot{q}^{\beta}-e \varphi(q, t)+e \dot{q}^{\beta} A_{\beta}(q, t)\] \\ (a) Obtén el Hamiltoniano y las ecuaciones canónicas para una partícula en un campo de fuerza central. (b) Toma dos de las condiciones iniciales como \(p_{\phi}(0)=0\) y \(\phi(0)=0\) (esto es esencialmente la elección de un sistema de coordenadas esféricas particular). Discute la simplificación resultante de las ecuaciones canónicas. ]

  Denota los momentos conjugados por \(p_{r}, p_{\theta}\) y \(p_{\phi}\). Las ecuaciones definitorias para ellos son
\begin{DispWithArrows}[format=ll, displaystyle]
  p_{r} & \equiv \frac{\partial L}{\partial \dot{r}}=m \dot{r}  \tag{5.11}\\
  p_{\theta} & \equiv \frac{\partial L}{\partial \dot{\theta}}=m r^{2} \dot{\theta} \\
  p_{\phi} & \equiv \frac{\partial L}{\partial \dot{\phi}}=m\left(r^{2} \sin ^{2} \theta\right) \dot{\phi}
\end{DispWithArrows}

Invertir estas ecuaciones da
\begin{DispWithArrows}[format=c, displaystyle]
  \dot{r}=\frac{p_{r}}{m}  \tag{5.12}\\
  \dot{\theta}=\frac{p_{\theta}}{m r^{2}} \\
  \dot{\phi}=\frac{p_{\phi}}{m r^{2} \sin ^{2} \theta}
\end{DispWithArrows}

El Hamiltoniano es
\begin{DispWithArrows}[format=c, displaystyle]
  H=p_{\alpha} \dot{q}^{\alpha}-L=\frac{p_{r}^{2}}{2 m}+\frac{p_{\theta}^{2}}{2 m r^{2}}+\frac{p_{\phi}^{2}}{2 m r^{2} \sin ^{2} \theta}+V(r) \equiv T+V \tag{5.13}
\end{DispWithArrows}
donde la energía cinética $T$ es la suma de los tres primeros términos. Las ecuaciones canónicas de Hamilton en estas variables son
\begin{DispWithArrows}[format=ll, displaystyle]
  \dot{r} \equiv \frac{\partial H}{\partial p_{r}}=\frac{p_{r}}{m}, & \dot{p}_{r} \equiv-\frac{\partial H}{\partial r}=\frac{1}{m r^{3}}\left[p_{\theta}^{2}+\frac{p_{\phi}^{2}}{\sin ^{2} \theta}\right]-V^{\prime}(r) \\
  \dot{\theta} \equiv \frac{\partial H}{\partial p_{\theta}}=\frac{p_{\theta}}{m r^{2}}, & \dot{p}_{\theta} \equiv-\frac{\partial H}{\partial \theta}=\frac{p_{\phi}^{2} \cos \theta}{m r^{2} \sin ^{3} \theta}  \tag{5.14}\\
  \dot{\phi} \equiv \frac{\partial H}{\partial p_{\phi}}=\frac{p_{\phi}}{m r^{2} \sin ^{2} \theta}, & \dot{p}_{\phi} \equiv-\frac{\partial H}{\partial \phi}=0
\end{DispWithArrows}

Comentario: Nota que las tres ecuaciones a la izquierda siguen inmediatamente de las definiciones de los momentos generalizados. Esto siempre ocurre con las ecuaciones canónicas de Hamilton: al estudiar su derivación queda claro que la mitad de ellas son simplemente inversiones de esas definiciones.

(b) Si $p_{\phi}(0)=0$, la última de las Eqs. (5.14) implica que $p_{\phi}(t)=0$ para todo $t$. Entonces las ecuaciones canónicas se simplifican.

De $\phi(0)=0$ y estas ecuaciones se sigue que $\phi(t)=0$ para todo $t$: el movimiento se mantiene en el plano $\phi=0$. El resultado es un sistema de dos grados de libertad que consiste en las dos primeras líneas de (5.15). La cuarta ecuación implica que $p_{\theta}$ es una constante del movimiento. Esta es el momento angular en el plano $\phi=0$, llamado $l$.
  
\end{example}

\subsection{
Coordenadas unificadas en $T* \mathbb{Q}^{*}$ y corchetes de Poisson}

Uno de los principales beneficios del formalismo hamiltoniano es que las variables $q$ y $p$ se tratan a la misma manera. Para hacer esto de manera consistente, los unimos en un conjunto de $2 n$ variables que serán llamadas $\xi^{\prime}$, con el índice $j$ corriendo desde 1 hasta $2 n$ (índices latinos van desde 1 hasta $2 n$ y índices griegos continúan corriendo desde 1 hasta $n$). El primer $n$ de las $\xi^{j}$ son las $q^{\beta}$, y el segundo $n$ son las $p_{\beta}$. De manera más explícita,

$$
\begin{array}{ll}
\xi^{j}=q^{j}, & j \in\{1, \ldots, n\} \\
\xi^{j}=p_{j-n}, & j \in\{n+1, \ldots, 2 n\}
\end{array}
$$
\todo{De estas dos ecuaciones, se deduce que $\xi^{j}$ es un conjunto de $2 n$ variables, donde $q^{\beta}$ y $p_{\beta}$ son las primeras $n$ y las últimas $n$ variables, respectivamente.}

Si ahora queremos escribir las ecuaciones canónicas de Hamilton en la forma unificada $\dot{\xi}^{\prime}=f^{\prime}(\xi)$, deben encontrarse los $f^{j}$ funciones a partir de la derecha de (5.5): los primeros $n$ del $f^{\prime}$ son las $\partial H / \partial p_{\jmath} \equiv \partial H / \partial \xi^{\prime+n}$, y los últimos $n$ son $-\partial H / \partial q^{J} \equiv-\partial H / \partial \xi^{j}$. Así que las ecuaciones canónicas son

Esta puede escribirse en una de las dos formas equivalentes

$$
\begin{array}{ll}
\dot{\xi}^{\prime}=\omega^{j k} \partial_{k} H \equiv \Delta_{H}\tag{5.42a}\\
\omega_{l,} \dot{\xi}^{\prime}=\partial_{l} H \tag{5.42b}
\end{array}
$$

Aquí $\partial_{k} \equiv \partial / \partial \xi^{k}$.

La ecuación (5.42a) introduce la matriz simétrica $2 n \times 2 n$ $\Omega$, con elementos de matriz $\omega^{\prime h}$ dados por
$$
\Omega=\left|\begin{array}{cc}
0_{n} & -\mathbb{I}_{n}  \tag{5.43}\\
\mathbb{I}_{n} & 0_{n}
\end{array}\right|
$$
donde $\mathbb{I}_{n}$ y $0_{n}$ son las matrices unitarias y de ceros $n \times n$, respectivamente. Dos propiedades importantes de $\Omega$ son que $\Omega^{-1}=-\Omega$ (usamos índices inferiores para los elementos de la matriz de $\Omega^{-1}$, escribiendo $\omega_{j l}$ ) y que es antisimétrica. En otras palabras,
$$
\begin{array}{l}
\Omega^{2}=-\mathbb{I}, \quad \Omega^{\mathrm{T}}=-\Omega \tag{5.44}\\
\omega^{\prime k} \omega_{k l}=\delta_{l}^{\prime}, \quad \omega_{k l} \omega_{l j} \delta^{l l}=-\delta_{k l}, \quad \omega^{k_{l}} \omega^{l /} \delta_{l l}=-\delta^{k j}
\end{array}
$$



\subsection{Corchetes de Poisson}

El término involucrado en $\omega^{\prime k}$ en el lado derecho de (5.46) es importante; se le llama el cuadrado Poisson de $f$ con $H$. En general, si $f, g \in \mathcal{F}\left(\mathbf{T}^{*} \mathbb{Q}\right)$, el cuadrado Poisson de $f$ con $g$\todo{$f, g$ son variables dinámicas} se define como

$$
\begin{aligned}
\{f, g\} & \equiv\left(\partial_{j} f\right) \omega^{\prime k}\left(\partial_{k} g\right) \equiv \frac{\partial f}{\partial \xi^{\prime}} \omega^{\prime k} \frac{\partial g}{\partial \xi^{k}} \\
& \equiv \frac{\partial f}{\partial q^{\alpha}} \frac{\partial g}{\partial p_{\alpha}}-\frac{\partial f}{\partial p_{\alpha}} \frac{\partial g}{\partial q^{\alpha}} \tag{5.47}
\end{aligned}
$$

si cambiamos la $f$ y la $g$ por $\xi^{j}$ y $\xi^{i}$, respectivamente.
\begin{DispWithArrows}[format=c, displaystyle]
\pb{\xi^{j}}{\xi^{i}}=\partial_{\gamma }\xi^{j} \omega^{\gamma  \delta } \partial_{\delta } \xi^{ i} = \delta_{\gamma}^{i}\omega\omega^{\gamma  \delta }\delta_{\delta}^{j} 
\end{DispWithArrows}

Si cambiamos f y g por $\xi^{i}$ y $\xi^{j}$, respectivamente, entonces
\begin{DispWithArrows}[format=ll, displaystyle]
\pb{\xi^{i}}{\xi^{j}}=& \partial_{\gamma }\xi^{i} \omega^{\gamma  \delta } \partial_{\delta } \xi^{ j} \\
=& \delta_{\gamma}^{j}\omega^{\gamma  \delta }\delta_{\delta}^{i}=\omega^{ij} 
\end{DispWithArrows}

Los corchetes de Poisson (CP) satisfacen una regla de producto de tipo Leibniz:

\begin{DispWithArrows}[displaystyle, format=c]
\{f, g h\}=g\{f, h\}+\{f, g\} h \tag{5.48}
\end{DispWithArrows}


\begin{remark}
  Bilinearidad
  \begin{DispWithArrows}[format=c, displaystyle]
  \pb{af+bg}{h}=a\{f, h\}+b\{g, h\} \notag
  \end{DispWithArrows}, 
  antisimetría
  \begin{DispWithArrows}[format=c, displaystyle]
  \pb{f}{g}=-\pb{g}{f} \notag
  \end{DispWithArrows} y la identidad Jacobi 
  \begin{DispWithArrows}[format=c, displaystyle]
  \pb{f}{\pb{g}{h}}+\pb{g}{\pb{h}{f}}+\pb{h}{\pb{f}{g}}=0 \notag
  \end{DispWithArrows}
  son las propiedades definidoras de una estructura algebraica importante llamada algebras Lie. El espacio funcional $\mathcal{F}\left(\mathbf{T}^{*} \mathrm{Q}\right)$ es por tanto una algebras Lie bajo el CP, y la dinámica hamiltoniana se estudia con éxito desde el punto de vista de las algebras Lie. El punto de vista algebraico Lie juega un papel importante también en la transición a mecánica cuántica.
\end{remark}

En términos del CP Eq. (5.46) se convierte
$$
\mathbf{L}_{\Delta} f \equiv \frac{d f}{d t}=\{f, H\}+\partial_{t} f
$$

La ecuación (5.49) se satisface por todas las variables dinámicas, incluyendo las coordenadas mismas. Aplicado a las coordenadas locales, se convierte en

\begin{DispWithArrows}[displaystyle, format=c]
\Delta_{H}^{j}=\dot{\xi}^{j}=\left\{\xi^{j}, H\right\} \left\{\begin{array}{c}
  \dot{q}^{a}=\pb{q^{a}}{H} \\
  \dot{o}_{a}=\pb{p_{a}}{H} \\
\end{array}\right.
    \tag{5.50}
\end{DispWithArrows}

que es solo otra forma de escribir las ecuaciones hamiltonianas canónicas. Otra aplicación de Eq. (5.49) es a la función hamiltoniana. Si hacemos la derivada de una función $f$ cualquiera obtenemos 
\begin{DispWithArrows}[format=c, displaystyle]
  \dv{f}{t}=\dot{f}=\partial_{i}f\dot{\xi}^{i}=\partial_{i}f\omega^{ij}\partial_{j}H+\partial_{t} f=\{f, H\}+\partial_{t} f\equiv \mathbf{L}_{\Delta_{H}}
\end{DispWithArrows}


La antisimetría del CP implica que $\{H, H\}=0$, y luego

\begin{DispWithArrows}[displaystyle, format=c]
\dot{H}=\{H, H\}+\partial_{t} H=\partial_{t} H \tag{5.51}
\end{DispWithArrows}

es decir, la derivada temporal de la función hamiltoniana es igual a su derivada parcial temporal. Esto refleja la conservación de energía para lagrangianos invarientes en el tiempo, ya que se puede demostrar desde la definición original de la función hamiltoniana que $\partial_{t} H=-\partial_{t} L$.

Los paréntesis de Poisson (PB) de funciones tomadas con las coordenadas jugarán un papel especial en gran parte de lo que sigue, así que los presentamos aquí. Si \( f \in \mathcal{F}\left(\mathbf{T}^{*} \mathbb{Q}\right) \), entonces se tienen las siguientes relaciones:

\[
\Delta_{f}^{\alpha }= \underbrace{\partial_{\gamma }\xi^{j}}_{\delta _{\gamma }^{j}}\omega^{jk}\partial_{k}f\left\{\xi^{j}, f\right\} = \omega^{j k} \partial_{k} f, \text{ o } 
\]

\[
\left\{p_{\alpha}, f\right\} = \frac{\partial f}{\partial q^{\alpha}}, \quad \left\{q^{\alpha}, f\right\} = -\frac{\partial f}{\partial p_{\alpha},} 
\]

\[
\left\{\xi^{j}, \xi^{k}\right\} = \omega^{j k}, \text{ o }
\]

\[
\left\{p_{\beta}, q^{\alpha}\right\} = \delta_{\beta}^{\alpha}, \quad \left\{q^{\alpha}, q^{\beta}\right\} = \left\{p_{\alpha}, p_{\beta}\right\} = 0.
\]



\begin{example}[(a) Encuentra las correciones Poisson de las componentes $j_\alpha$ del momento angular $\mathbf{j}$ con variables dinámicas (usamos $\mathbf{j}$ en lugar de $\mathbf{L}$ para evitar la confusión con el Lagrangiano y la derivada de Lie). \\  (b) Encuentra las PBs de las componentes $j_\alpha$ con la posición y momentum en espacio euclidiano 3. \\
  (c) La única forma de formar escalar es mediante productos de dot. Encuentra las correciones Poisson de las $j_\alpha$ con los momientos y posiciones. \\
  (d) La única forma de formar vectores es multiplicando $\mathbf{x}$ y $\mathbf{p}$ por escalar y tomando productos cruzados. Encuentra las PBs de las $j_\alpha$ con vectores. \\
  Especializa en los CP  de las $j_\alpha$ entre sí.]

  Configuración del espacio de la cuadrícula $\mathbb{Q}$ es el espacio vectorial $\mathbb{E}^{3}$ con coordenadas $\left(x^{1}, x^{2}, x^{3}\right)$. El anticommutador entre componentes de la momenta angular no es cero, sino que es igual a la componente $x^{3}$ del producto escalar. Por lo tanto, para cualquier vector escalar $\mathbf{w} = f \mathbf{x} + g \mathbf{p} + h \mathbf{j}$,

  El espacio de configuración $\mathbb{Q}$ es el espacio vectorial $\mathbb{E}^{3}$ con coordenadas $\left(x^{1}, x^{2}, x^{3}\right) \equiv \mathbf{x}$. El momento angular $\mathbf{j}$ de una sola partícula se define como $\mathbf{j}=\mathbf{x} \wedge \mathbf{p}$. Aquí $\mathbf{p}$ es el momento, que se encuentra en otro $\mathbb{E}^{3}$. El momento angular $\mathbf{j}$, en un tercer $\mathbb{E}^{3}$, tiene componentes
  \begin{DispWithArrows}[displaystyle, format=c]
  j_{\alpha}=\epsilon_{\alpha \beta \gamma} x^{\beta} p_{\gamma} \tag{5.53}
  \end{DispWithArrows}
  (Como hacemos a menudo al usar coordenadas cartesianas, violamos las convenciones sobre índices superiores e inferiores).
  (a) De la segunda línea de la Ecuación (5.52) tenemos
  \begin{aligned}
  \left\{j_{\alpha}, f\right\} & =\epsilon_{\alpha \beta Y}\left\{x^{\beta} p_{Y}, f\right\}=\epsilon_{\alpha \beta_{Y}} x^{\beta}\left\{p_{\gamma}, f\right\}+\epsilon_{\alpha \beta \gamma}\left\{x^{\beta}, f\right\} p_{\gamma} \\
  & =\epsilon_{\alpha \beta Y}\left[-x^{\beta} \partial f / \partial x^{\gamma}+p_{\gamma} \partial f / \partial p_{\beta}\right] .
  \end{aligned}
  
  (b) Usa el resultado anterior con $f \equiv x^{\lambda}$ y $f=p_{\lambda}$ para obtener
  \begin{aligned}
  & \left\{j_{\alpha}, x^{\lambda}\right\}=-\epsilon_{\alpha \beta \gamma} x^{\beta} \delta_{\gamma}^{\lambda}=\epsilon_{\alpha \lambda \beta} x^{\beta} \\
  & \left\{j_{\alpha}, p_{\lambda}\right\}=\epsilon_{\alpha \beta \beta \gamma} p_{\gamma} \delta_{\lambda}^{\beta}=\epsilon_{\alpha \lambda \gamma} p_{\gamma}
  \end{aligned}
  
  (c) Usa la regla de Leibniz, Ecuación (5.48), para obtener
  \begin{aligned}
  & \left\{j_{\alpha}, x^{\lambda} x^{\lambda}\right\}=2 x^{\lambda} \epsilon_{\alpha \lambda \beta} x^{\beta}=0 \\
  & \left\{j_{\alpha}, p_{\lambda} p_{\lambda}\right\}=2 p_{\lambda} \epsilon_{\alpha \lambda \gamma} p_{Y}=0 \\
  & \left\{j_{\alpha}, x^{\lambda} p_{\lambda}\right\}=\epsilon_{\alpha \lambda \gamma} x^{\lambda} p_{\gamma}+\epsilon_{\alpha \lambda \beta} x^{\beta} p_{\lambda}=0
  \end{aligned}
  
  (d) Los únicos vectores son productos de funciones escalares con $\mathbf{x}, \mathbf{p}$ y $\mathbf{j}=\mathbf{x} \wedge \mathbf{p}$, es decir, son de la forma
  $$
  \mathbf{w}=f \mathbf{x}+g \mathbf{p}+h \mathbf{j}
  $$
  donde $f, g$ y $h$ son escalares. Por la regla de Leibniz y la Parte (a)
  \begin{aligned}
  \left\{j_{\alpha}, f x^{\lambda}\right\} & =f \epsilon_{\alpha \lambda} x^{\beta} \\
  \left\{j_{\alpha}, g p_{\lambda}\right\} & =g \epsilon_{\alpha \lambda \gamma} p_{\gamma}
  \end{aligned}
  
  Solo necesitan calcularse los corchetes de Poisson de los componentes de $\mathbf{j}$ entre sí:
  \begin{aligned}
  \left\{j_{\alpha}, j_{\beta}\right\} & =\epsilon_{\alpha \mu \nu}\left\{x^{\mu} p_{\nu}, j_{\beta}\right\}=\epsilon_{\alpha \mu \nu} x^{\mu}\left\{p_{\nu}, j_{\beta}\right\}+\epsilon_{\alpha \mu \nu}\left\{x^{\mu}, j_{\beta}\right\} p_{v} \\
  & =\epsilon_{\alpha \mu \nu} \epsilon_{\nu \beta \gamma} x^{\mu} p_{\gamma}+\epsilon_{\alpha \mu \nu} \epsilon_{\mu \beta \gamma} x^{\gamma} p_{\nu} \\
  & =\left(\epsilon_{\alpha \mu \nu} \epsilon_{\nu \beta \gamma}+\epsilon_{\alpha \nu \gamma} \epsilon_{\nu \beta \mu}\right) x^{\mu} p_{\gamma} \\
  & =\left[\left(\delta_{\alpha \beta} \delta_{\mu \nu}-\delta_{\alpha \gamma} \delta_{\mu \beta}\right)+\left(\delta_{\gamma \beta} \delta_{\mu \alpha}-\delta_{\gamma \mu} \delta_{\alpha \beta}\right)\right] x^{\mu} p_{\gamma} \\
  & =\left(\delta_{\gamma \beta} \delta_{\alpha \mu}-\delta_{\alpha \gamma} \delta_{\mu \beta}\right) x^{\mu} p_{\gamma}=\epsilon_{\alpha \beta \nu} \epsilon_{\nu \mu \gamma} x^{\mu} p_{\gamma}
  \end{aligned}
  
  o
  \begin{DispWithArrows}[displaystyle, format=c]
  \left\{j_{\alpha}, j_{\beta}\right\}=\epsilon_{\alpha \beta \nu} j_{\nu} \tag{5.54}
  \end{DispWithArrows}
  
  Dado que $h$ es un escalar, la regla de Leibniz implica que $\left\{j_{\alpha}, h j_{\beta}\right\}=h \epsilon_{\alpha \beta v} j_{v}$. Así, para cualquier vector $\mathbf{w}$,
  $$
  \left\{j_{\alpha}, w_{\beta}\right\}=\epsilon_{\alpha \beta v} w_{v}
  $$
  
  Resumiendo, tenemos
  \begin{aligned}
  & \left\{j_{1}, w_{1}\right\}=\left\{j_{2}, w_{2}\right\}=\left\{j_{3}, w_{3}\right\}=0 \\
  & \left\{j_{1}, w_{2}\right\}=w_{3},\left\{j_{2}, w_{3}\right\}=w_{1}, \quad \text { y permutaciones cíclicas }
  \end{aligned}
  
  En particular,
  \begin{aligned}
  & \left\{j_{1}, j_{2}\right\}=j_{3} \quad \text { y permutaciones cíclicas } \\
  & \left\{j_{1}, j_{1}\right\}=\left\{j_{2}, j_{2}\right\}=\left\{j_{3}, j_{3}\right\}=0
  \end{aligned}
  
  Los componentes cartesianos del momento angular no conmutan entre sí (el término se toma de álgebra matricial). Veremos más adelante que esto tiene consecuencias importantes.
  
\end{example}

\subsubsection{Oscilador armónico istrópico}

Consideremos el oscilador armónico isotrópico en \( n \) grados de libertad (para simplificar, tomemos tanto la masa \( m = 1 \) como la constante del resorte \( k = 1 \)). El lagrangiano es 

\[
L = \frac{1}{2} \delta_{\alpha \beta} \left( \dot{q}^{\alpha} \dot{q}^{\beta} - q^{\alpha} q^{\beta} \right)
\]
\todo{Al pasar a la nueva notación hemos perdido como se transformar los momentos (forma covariante), se verá como se hace}

lo cual lleva a

\[
H = \frac{1}{2} (\delta^{\alpha \beta }p_{\alpha }p_{\beta }+\delta_{\alpha \beta }q^{\alpha \beta })= \frac{1}{2} \delta_{y k} \xi^{y} \xi^{k}
\]

Luego, a partir de (5.50) con bilinealidad y Leibniz, se sigue que

\[
\dot{\xi}^t = \frac{1}{2} \delta_{j k} \left\{ \xi^t, \xi^j \xi^k \right\} = \delta_{1 k} \omega^l \xi^k \tag{5.55}
\]

Una forma de resolver estas ecuaciones es derivarlas con respecto al tiempo y usar la propiedad de que \( \Omega^{2} = -\mathbb{I} \), obteniendo \( \ddot{\xi}^{k} = -\xi^{k} \). Estas \( 2n \) ecuaciones de segundo orden dan lugar a \( 4n \) constantes del movimiento que están conectadas por las ecuaciones de primer orden (5.55), de modo que solo \( 2n \) de ellas son independientes. Una manera diferente, y quizás más interesante, de resolver estas ecuaciones es escribirlas en la forma

\[
\dot{\vec{\xi}} = \Lambda \vec{\xi} \tag{5.56}
\]

donde \( \xi \) es el vector de \( 2n \) dimensiones con componentes \( \xi^{k} \) y \( \Lambda =\Omega^{-1}\mathrm{i}\) es la matriz con elementos \( \lambda_{k}^{y} = \delta_{j k} \omega^{t} \). La solución de la ecuación (5.56) es

\[
\vec{}{\xi}(t) = e^{\Lambda t} \vec{\xi_{0}} \tag{5.57}
\]

donde \( \xi_{0} \) es un vector constante cuyas componentes son las condiciones iniciales, y

\[
e^{\Lambda t} \equiv \sum_{0}^{\infty} \frac{(\Lambda t)^n}{n!} \tag{5.58}
\]

Calcular \( e^{\Lambda t} \) se simplifica por el hecho de que \( \Lambda^{2} = -\mathbb{I} \); de hecho,

\[
\left( \Lambda^{2} \right)_{l}^{l} \equiv \lambda_{k}^{l} \lambda_{l}^{k} = \omega^{l} \delta_{l k} \omega^{k r} \delta_{r l} = -\delta^{r} \delta_{r l} = -\delta_{l}^{l}
\]

Esto se puede usar para escribir todas las potencias de \( \Lambda \):

\[
\Lambda^{3} = -\Lambda, \quad \Lambda^{4} = \mathbb{I}, \quad \Lambda^{5} = \Lambda, \quad \Lambda^{6} = -\mathbb{I}, \ldots
\]

Luego, al expandir la suma en (5.58) se obtiene

\[
e^{\Lambda t} = \mathbb{I} \left( 1 - \frac{t^2}{2!} + \frac{t^4}{4!} + \cdots \right) + \Lambda \left( t - \frac{t^3}{3!} + \frac{t^5}{5!} + \cdots \right)
= \mathbb{I} \cos t + \Lambda \sin t
\]

(comparar con la expansión similar para \( e^{\prime t} \)). La solución de las ecuaciones de movimiento es entonces

\[
\xi(t) = \xi_{0} \cos t + \Lambda \xi_{0} \sin t \tag{5.59a}
\]

o

\[
\xi^{k}(t) = \xi_{0}^{k} \cos t + \lambda^{k}, \xi_{0}^{\prime} \sin t \tag{5.59~b}
\]

Las ecuaciones (5.59a, 5.59b) son solo una forma diferente de escribir (5.57). Debido a que solo se usaron ecuaciones de primer orden, solo las \( 2n \) constantes del movimiento \( \xi_{0}^{k} \) aparecen en estas soluciones. Finalmente, para completar el ejemplo, escribimos la forma \((q, p)\) de la solución:

\[
\begin{aligned}
q^{\alpha}(t) &= q_{0}^{\alpha} \cos t + p_{0 \alpha} \sin t \\
p_{\alpha}(t) &= -q_{0}^{\alpha} \sin t + p_{0 \alpha} \cos t
\end{aligned}
\]

\subsection{Corchetes de Poisson y dinámica Hamiltoniana}

Los corchetes de Poisson juegan un papel especial cuando el movimiento se deriva de las ecuaciones canónicas de Hamilton. No todo movimiento concebible en \(\mathbf{T}^{*} \mathbb{Q}\) es un sistema dinámico hamiltoniano, definido de la manera canónica por (5.42a) o (5.50). Cualquier ecuación de la forma

\[
\dot{\xi}^{j} = X^{\prime}(\xi) \tag{5.60}
\]

con \( X^{\prime} \in \mathcal{F}\left(\mathbf{T}^{*} \mathbb{Q}\right) \), define un sistema dinámico en \(\mathbf{T}^{*} \mathbb{Q}\) (los \( X^{f} \) son las componentes de un campo vectorial dinámico). Si no existe \( H \in \mathcal{F}\left(\mathbf{T}^{*} \mathbb{Q}\right) \) tal que \( X^{\prime} = \omega^{j k} \partial_{k} H \), puede ser imposible poner las ecuaciones de movimiento en la forma canónica de (5.42a).

\begin{example}
  Consideremos, por ejemplo, el sistema en dos grados de libertad dado por

\[
\dot{q} = q p, \quad \dot{p} = -q p \tag{5.61}
\]

Si este fuera un sistema hamiltoniano, \(\dot{q} = q p\) sería igual a \(\partial H / \partial p\), y \(\dot{p} = -q p\) sería igual a \(-\partial H / \partial q\). Pero entonces, \(\partial^{2} H / \partial q \partial p\) no sería igual a \(\partial^{2} H / \partial p \partial q\); por lo tanto, no existe una función \(H\) que sea el hamiltoniano de este sistema. No obstante, este es un sistema dinámico legítimo cuyas curvas integrales se encuentran fácilmente:

\[
q(t) = q_{0} \frac{C e^{C t}}{p_{0} + q_{0} e^{C t}}, \quad p(t) = p_{0} \frac{C}{p_{0} + q_{0} e^{C t}}
\]

donde \(C = q_{0} + p_{0} = q + p\) es una constante del movimiento.
\end{example}

Un papel especial del corchete de Poisson es que proporciona una prueba de si un sistema dinámico es hamiltoniano o no. 

\begin{proposition}
  Ahora mostramos que un sistema dinámico es hamiltoniano (o que los \(X^{\prime}\) son los componentes de un campo vectorial hamiltoniano) si y solo si la derivada temporal actúa sobre los paréntesis de Poisson como si fueran productos (es decir, mediante la regla de Leibniz). Es decir, el sistema es hamiltoniano si y solo si

\[
\frac{d}{d t}\{f, g\} = \{\dot{f}, g\} + \{f, \dot{g}\} \tag{5.62}
\]

\end{proposition}
\begin{proof}
  Para la demostración, asumamos que el sistema dinámico es hamiltoniano y que \( H(\xi, t) \) es la función hamiltoniana. Entonces,

\[
\frac{d}{d t}\{f, g\} = \{\{f, g\}, H\} + \partial_{t}\{f, g\}
\]

Según la identidad de Jacobi (y la antisimetría), el primer término se puede escribir como

\[
\{\{f, g\}, H\} = \{\{f, H\}, g\} + \{f, \{g, H\}\}
\]

El segundo término es

\[
\begin{aligned}
\partial_{t}\{f, g\} & = \partial_{t}\left[(\partial_j f) \, \omega_{j k} \, \partial_k g\right] \\
& = (\partial_j \, \partial_{t} f) \, \omega_{j k} \, \partial_k g + (\partial_j f) \, \omega_{j k} \, \partial_k \, \partial_{t} g \\
& = \{\partial_{t} f, g\} + \{f, \partial_{t} g\}.
\end{aligned}
\]

Combinando ambos resultados, se obtiene

\[
\begin{aligned}
\frac{d}{d t}\{f, g\} &= \left\{\{f, H\} + \partial_{t} f, g\right\} + \left\{f, \{g, H\} + \partial_{t} g\right\} \\
&= \{\dot{f}, g\} + \{f, \dot{g}\}.
\end{aligned}
\]

Esto prueba que, si existe una función hamiltoniana, la regla de Leibniz es válida para el paréntesis de Poisson.
\end{proof}

\begin{proof}
  Para demostrar la afirmación contraria, usaremos la siguiente lógica: si la regla de Leibniz de la Ecuación (5.62) se cumple para todos \( f, g \in \mathcal{F}\left(\mathbf{T}^{*} \mathbb{Q}\right) \), entonces se cumple en particular para el par \( \xi^{l}, \xi^{\prime} \). La demostración mostrará que si (5.62) se cumple para este par, entonces existe una función Hamiltoniana \( H \) tal que \( \omega_{j k} \dot{\xi}^{k} = \partial_{j} H \); es decir, la existencia de la función Hamiltoniana se deduce de la regla de Leibniz.

Dado que \( \left\{\xi^{l}, \xi^{\prime}\right\} = \omega^{l i} \), su derivada temporal es cero. Así, tenemos:

\[
\begin{aligned}
\frac{d}{d t}\{\xi^{l}, \xi^{\prime}\} &= 0 = \{\dot{\xi}^{l}, \xi^{\prime}\} + \{\xi^{l}, \dot{\xi}^{\prime}\} = \{X^{l}, \xi^{\prime}\} + \{\xi^{l}, X^{\prime}\} \\
&= (\partial_{j} X^{l}) \omega^{j k} \partial_{k} \xi^{\prime} + (\partial_{j} \xi^{l}) \omega^{j k} \partial_{k} X^{\prime} \\
&= \partial_{j}(X^{l} \omega^{j k}) + \partial_{k}(\omega^{j k} X^{l}),
\end{aligned}
\]

donde hemos usado la Ecuación (5.60). Ahora, multiplicamos por \( \omega_{l p} \omega_{i r} \) y sumamos sobre los índices repetidos, obteniendo

\[
\partial_{p} Z_{r} - \partial_{r} Z_{p} = 0,
\]

donde \( Z_{l} = \omega_{j k} X^{k} \). Esta es la condición de integrabilidad local (ver Problema 2.4) para la existencia de una función \( H \) que satisfaga el conjunto de ecuaciones diferenciales parciales

\[
\partial_{j} H = Z_{l} \equiv \omega_{l k} X^{k} \equiv \omega_{j k} \dot{\xi}^{k}.
\]

Esto completa la demostración. Hemos probado el resultado solo de forma local, lo cual es lo máximo que se puede lograr. Si el paréntesis de Poisson se comporta como un producto con respecto a la diferenciación temporal, entonces existe una función Hamiltoniana local \( H \); es decir, el sistema dinámico es localmente hamiltoniano. Puede no existir una única función \( H \) que sea válida en toda \( \mathbf{T}^{*} \mathbb{Q} \).
\end{proof}

\begin{corollary}
  
\end{corollary}

\section{Geometría simpléctica}

En esta sección se analiza la geometría de 
$\mathbf{T}^{*} \mathbb{Q}$ y cómo esta geometría contribuye a sus propiedades como una variedad portadora para la dinámica.

\subsection{Espacio cotangente}

Primero demostraremos la diferencia entre \( T \mathbb{Q} \) y \( \mathbf{T}^{*} \mathbb{Q} \). El haz tangente \( \mathbf{T} \mathbb{Q} \) consiste en la variedad de configuración \( \mathbb{Q} \) y el conjunto de los espacios tangentes \( \mathbf{T}_{q} \mathbb{Q} \), cada uno asociado a un punto \( q \in \mathbb{Q} \). Los puntos de \( \mathbf{T} \mathbb{Q} \) tienen la forma \( (q, \dot{q}) \), donde \( q \in \mathbb{Q} \) y \( \dot{q} \) es un vector en \( \mathbf{T}_{q} \mathbb{Q} \). Sin embargo, los puntos \( (q, p) \) de \( \mathbf{T}^{*} \mathbb{Q} \) no tienen esta forma, ya que \( p \) no es un vector en \( \mathbf{T}_{q} \mathbb{Q} \). Mostramos esto al considerar la forma diferencial \( \theta_{L} \equiv \left(\partial L / \partial \dot{q}^{\alpha}\right) d q^{\alpha} = p_{\alpha} d q^{\alpha} \) de la Ecuación (3.86) y al compararla con el campo vectorial \( \dot{q}^{\alpha}\left(\partial / \partial q^{\alpha}\right) \).

Los \( \dot{q}^{\alpha} \) son los componentes locales del campo vectorial \( \dot{q}^{\alpha}\left(\partial / \partial q^{\alpha}\right) \) en \( \mathbb{Q} \): para funciones dadas \( \dot{q}^{\alpha} \in \mathcal{F}(\mathbb{Q}) \), especifican el vector con componentes \( \dot{q}^{\alpha}(q) \) en el espacio tangente \( \mathbf{T}_{q} \mathbb{Q} \) en cada \( q \in \mathbb{Q} \). Sin embargo, los \( p_{\alpha} \) son los componentes locales de la forma diferencial \( \theta_{L} = p_{\alpha} d q^{\alpha} \). Aunque \( \theta_{L} \) se introdujo como una forma diferencial en \( \mathbb{T} \mathbb{Q} \), también puede verse como una forma diferencial en \( \mathbb{Q} \), ya que no contiene una parte \( d \dot{q} \) (cuando se ve de esta manera, sus componentes dependen, sin embargo, de los \( \dot{q}^{\alpha} \) como parámetros). Una forma diferencial no es un campo vectorial, y por lo tanto los \( p_{\alpha} = \partial L / \partial \dot{q}^{\alpha} \), componentes de una forma diferencial, no son componentes de un campo vectorial. En la Sección 3.4.1 mencionamos que las formas diferenciales son duales a los campos vectoriales, por lo que \( \theta_{L} \) pertenece a un espacio dual al \( \mathbf{T}_{q} \mathbb{Q} \). Este nuevo espacio se denota como \( \mathbf{T}_{q}^{*} \mathbb{Q} \) y se llama el espacio cotangente en \( q \in \mathbb{Q} \). Sus elementos, las formas diferenciales, mapean los vectores (es decir, se combinan con ellos en una especie de producto interno) hacia funciones.

\subsubsection{dos-formas}
Recordemos que en el tema 1, una 1-forma \( \alpha \) en \( T \mathbb{Q} \) se definió como un mapeo lineal de los campos vectoriales \( X \) a funciones, es decir, \( \alpha: \mathcal{X} \rightarrow \mathcal{F}: X \mapsto \langle \alpha, X \rangle \). (Esta definición es válida para cualquier variedad diferencial, tanto para \( \mathbf{T}^{*} \mathbb{Q} \) como para \( \mathbf{T Q} \)). Las 2-formas se definen como mapeos bilineales y antisimétricos de pares de campos vectoriales a funciones. Es decir, si \( \omega \) es una 2-forma en \( \mathbf{T}^{*} \mathbb{Q} \) y \( X \) y \( Y \) son campos vectoriales en \( \mathbf{T}^{*} \mathbb{Q} \), entonces

$$
\omega(X, Y)=-\omega(Y, X) \in \mathcal{F}\left(\mathbf{T}^{*} \mathbb{Q}\right) \tag{5.66}
$$

matemáticamente, \( \omega: \mathcal{X} \times \mathcal{X} \rightarrow \mathcal{F} \). Como el mapeo es bilineal, puede representarse localmente mediante una matriz cuyos elementos son (recordando que \( \partial_{j} \equiv \partial / \partial \xi^{\prime} \) es un campo vectorial)

$$
\omega_{j k} = \omega\left(\partial_{j}, \partial_{k}\right) = -\omega_{k j} = -\omega\left(\partial_{k}, \partial_{j}\right) \tag{5.67}
$$

de modo que, por linealidad, si (localmente) \( X = X^{j} \partial_{j} \) y \( Y = Y^{k} \partial_{j} \), entonces

$$
\omega(X, Y) = \omega_{j k} X^{j} Y^{k} \tag{5.68}
$$

¿Qué ocurre si una 2-forma se aplica solo a un campo vectorial \( Y^{k} \partial_{k} \), es decir, qué tipo de objeto es \( \omega_{j k} Y^{k} \)? No es una función, ya que su subíndice \( j \) implica que tiene \( n \) componentes. Sin embargo, se puede construir una función a partir de ella y de un segundo campo vectorial \( X^{j} \partial_{j} \), multiplicando los componentes y sumando sobre \( j \), obteniendo así \( \left(\omega_{j k} Y^{k}\right) X^{j} \), que es el lado derecho de la Ecuación (5.68). Esto significa que \( \omega_{j k} Y^{k} \) es el componente \( j \)-ésimo de una 1-forma, ya que puede utilizarse para mapear cualquier campo vectorial \( X^{j} \partial_{j} \) a una función. En una notación evidente, esta 1-forma puede llamarse \( \omega(\bullet, Y) = -\omega(Y, \bullet) \) (el \( \bullet \) indica dónde colocar el otro campo vectorial): cuando la 1-forma \( \omega(\bullet, Y) \) se aplica a un campo vectorial \( X = X^{j} \partial_{j} \), el resultado es \( \omega(X, Y) \).

Es conveniente introducir una terminología y notación uniformes para la acción de 1-formas y 2-formas sobre campos vectoriales. Se dice que los campos vectoriales se contraen con o se insertan en las formas, denotado por \( i_{X} \):

$$
i_{X} \alpha \equiv \alpha(X) \equiv \langle \alpha, X \rangle \quad \text { y } \quad i_{X} \omega \equiv \omega(\bullet, X) \tag{5.69}
$$

donde \( X \) es un campo vectorial, \( \alpha \) es una 1-forma, y \( \omega \) es una 2-forma. Entonces \( i_{X} \alpha \) y \( i_{Y} i_{X} \omega \equiv \omega(X, Y) = -i_{X} i_{Y} \omega \) son funciones, y \( i_{X} \omega \) es una 1-forma.
\subsubsection{k-formas}

De forma enteramente análoga a como hemos definidos las dos-formas, se definen las $k$-formas\sidenote{\href{https://www.youtube.com/watch?v=xRf9-hdxB0w}{En este video explica un un poco mejor que son las k-formas}} como tensores covariantes totalmente antisimétricos que actúan sobre $k$ vectores para dar un número real (o una función en el caso de campos). Mediante una asignación suave de una $k$-forma a cada punto de variedad diferenciables $\mathscr{M}$ obtenemos un campo de $k$-formas.

Los productos tensoriales antisimetrizados y ordenados $d \zeta^{\alpha_{1}} \dot{\Lambda} \cdots \dot{\lambda} d \zeta^{\alpha_{k}}$ forman una base de $k$-formas y cualquier $k$-forma puede escribir como combinación lineal de los mismos:
$$
\omega=\sigma_{\alpha_{1} \cdots \alpha_{k}} \mathrm{~d} \zeta^{\alpha_{1}} \dot{\wedge} \cdots \dot{\wedge} \mathrm{~d} \zeta^{\alpha_{k}}=\frac{1}{k!} \sigma_{\alpha_{1} \cdots \alpha_{k}} \mathrm{~d} \zeta^{\alpha_{1}} \wedge \cdots \wedge \mathrm{~d} \zeta^{\alpha_{k}}
$$

Definiremos la contracción $i_{X} \omega$ de una $k$-forma $\omega$ con un vector $X$ como
$$
i_{X} \omega=\omega(X, \diamond, \ldots, \diamond)=\frac{1}{(k-1)!} \sigma_{\alpha_{1} \cdots \alpha_{k}} X^{\alpha_{1}} \mathrm{~d} \zeta^{\alpha_{2}} \wedge \cdots \wedge \mathrm{~d} \zeta^{\alpha_{k}}
$$

Si $\sigma$ es una $k$-forma y $\rho$ una $l$-forma, definimos el producto exterior de ambas $\sigma \wedge \rho$ como su producto tensorial completamente antisimetrizado, cuyas componentes son:
$$
(\sigma \wedge p)_{\alpha_{1} \cdots \alpha_{k} \beta_{1} \cdots \beta_{l}}=\frac{(k+l)!}{k!!!} \sigma_{\left[\alpha_{1} \cdots \alpha_{k}\right.} \rho_{\left.\beta_{1} \cdots \beta_{2}\right]}
$$
Se puede demostrar que si $\sigma$ es una $k$-forma y $\rho$ una $l$-forma, entonces
$$
i_{X}(\omega \wedge \rho)=i_{X} \omega \wedge \rho+(-1)^{k} \omega \wedge i_{X} \rho
$$

\subsubsection{Derivada exterior}

Definimos la derivada exterior d como una operación que actúa sobre una $k$-forma $\omega$ para dar la $(k+1)$-forma
$$
\begin{aligned}
\mathrm{d} \omega & =\frac{1}{k!} \mathrm{d} \sigma_{\alpha_{1} \cdots \alpha_{k}} \wedge \mathrm{~d} \zeta^{\alpha_{1}} \wedge \cdots \wedge \mathrm{~d} \zeta^{\alpha_{k}}= \\
& =\frac{1}{k!} \partial_{\beta} \sigma_{\alpha_{1} \cdots \alpha_{k}} \mathrm{~d} \zeta^{\beta} \wedge \mathrm{d} \zeta^{\alpha_{1}} \wedge \cdots \wedge \mathrm{~d} \zeta^{\alpha_{k}}= \\
& =(k+1) \partial_{[\beta} \sigma_{\alpha_{1} \cdots \alpha_{k}} \mathrm{~d} \zeta^{\beta} \dot{\wedge} \zeta^{\alpha_{1}} \dot{\wedge} \dot{\wedge} \mathrm{~d} \zeta^{\alpha_{k}}
\end{aligned}
$$
cuyas componentes son $(k+1) \partial_{[\beta} \sigma_{\left.\alpha_{1} \cdots \alpha_{k}\right]}$. La derivada exterior satisface la "regla de Leibniz antisimetrizada"
$$
\mathrm{d}(\omega \wedge \rho)=\mathrm{d} \omega \wedge \rho+(-1)^{k} \omega \wedge \mathrm{~d} \rho
$$
donde $\omega$ es una $k$-forma y $\rho$ una $l$-forma.
Por ejemplo, la derivada exterior $\mathrm{d} \omega$ de una uno-forma $\omega=\sigma_{\alpha} \mathrm{d} \zeta^{\alpha}$ es 
$$
\mathrm{d} \omega=\mathrm{d}\left(\sigma_{\beta} \mathrm{d} \zeta^{\beta}\right)=\partial_{\alpha} \sigma_{\beta} \mathrm{d} \zeta^{\alpha} \wedge \mathrm{d} \zeta^{\beta}=2 \partial_{[\alpha} \sigma_{\beta]} \mathrm{d} \zeta^{\alpha} \dot{\lambda} \mathrm{d} \zeta^{\beta}
$$

Una forma es exacta si la derivada exterior de otra. Una forma es cerrada si su derivada exterior se anula. La condición necesaria y suficiente para que una $k$ forma sea exacta (localmente) es que sea una forma cerrada. En particular, para uno-formas $\omega$ y dos-formas $\rho$,
$$
\mathrm{d} \omega=0 \Leftrightarrow \exists f|\omega=\mathrm{d} f, \quad \mathrm{~d} \rho=0 \Leftrightarrow \exists \omega| \rho=\mathrm{d} \omega \quad \text { (localmente). }
$$

En componentes, $\partial_{[\alpha} \sigma_{\beta]}=0 \Leftrightarrow \sigma_{\alpha}=\partial_{\alpha} f, \quad \partial_{[\alpha} \rho_{\beta \gamma]}=0 \Leftrightarrow \rho_{\alpha \beta}=2 \partial_{[\alpha} \sigma_{\beta]}$.
\subsubsection{Ley de transformación II}

Como hemos visto, una $k$-forma es un tensor antisimétrico covariante. Por tanto, bajo cambios de coordenadas, una $k$-forma se transforma de forma inversa a como lo hacen los vectores en cada uno de sus índices:

\begin{DispWithArrows}[format=c, displaystyle]
  \omega_{\alpha}^{\prime}=\frac{\partial \zeta^{\beta}}{\partial \zeta^{\prime \alpha}} \omega_{\beta}, \tag{1-forma}\\  \omega_{\alpha \beta}^{\prime}=\frac{\partial \zeta^{\gamma}}{\partial \zeta^{\prime} \alpha} \frac{\partial \zeta^{\delta}}{\partial \zeta^{\prime \beta}} \omega_{\gamma \delta}, \tag{2-forma}\\  \omega_{\alpha_{1} \cdots \alpha_{k}}^{\prime}=\frac{\partial \zeta^{\gamma_{1}}}{\partial \zeta^{\prime \alpha_{1}}} \cdots \frac{\partial \zeta^{\gamma_{k}}}{\partial \zeta^{\prime \alpha_{k}}} \omega_{\gamma_{1} \cdots \gamma_{k}} \tag{k-forma}
\end{DispWithArrows}

\subsection{Geometria simppléctica}