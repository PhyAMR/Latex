\setchapterpreamble[u]{\margintoc}
\chapter{Mecánica Hamiltoniana}
\labch{Part}


\section{Motivación de sistemas Hamiltonianos}

Uno podría imaginar, de forma más general, la construcción de una variedad diferente $\mathbb{F}$ añadiendo un tipo diferente de fibra a cada punto de $\mathbb{Q}$ (el resultado sería un haz de fibras sobre $\mathbb{Q}$ ). Si se pudiera hacer que $\mathbb{F}$ llevara de alguna manera la dinámica (para separar las trayectorias), y si las trayectorias sobre $\mathbb{F}$ pudieran proyectarse hasta las trayectorias físicas sobre $\mathbb{Q}$, entonces $\mathbb{F}$ serviría para nuestros propósitos tan bien como $\mathbf{TQ}$. En el formalismo hamiltoniano esto es precisamente lo que se hace: $\mathbb{F}$ es el haz cuyas fibras consisten en los posibles valores del momento $p$, una variedad cuyos puntos son de la forma ( $q, p$ ). Se denomina haz cotangente o colector (o espacio) de fase $\mathbf{T}^{*}. \mathbf{Q}$; sus fibras no son los espacios tangentes a los puntos de $\mathbb{Q}$, sino sus espacios duales (definidos en el apartado 3.4.1), llamados espacios cotangentes. Estas afirmaciones se aclararán a medida que avancemos.

\subsection{Ecuaciones de Hamilton}

A continuación vamos a derviar las ecuaciones de Hamilton a partir de lo que conocemos, es decir, el lagrangiano. Para ello vamos a hacer uso de la transformada de Legendre, que ya vimos en Termodinámica de segundo (Ver apéndice \ref{Legendre}). Lo que vamos a ver es análogo a lo del apéndice. 

El momento generalizado $p_{\alpha}$ conjugado con $q^{\alpha}$ ha sido definido (Sección 2.2.1) como
\begin{equation*}
  p_{\alpha}=\frac{\partial L}{\partial \dot{q}^{\alpha}} \tag{5.1}
\end{equation*}

donde $L: \mathbb{Q} \times \mathbb{R} \rightarrow \mathbb{R}$ es, como siempre, la función lagrangiana. (El dominio de $L$ está formado por $\mathbf{T}$ y el tiempo $t \in \mathbb{R}$). Por tanto, las ecuaciones de E-L y las ecuaciones diferenciales para $q^{\alpha}$ pueden escribirse de la forma

\begin{DispWithArrows}[format=ll, displaystyle]
  \frac{d p_{\alpha}}{d t} & =\frac{\partial L}{\partial q^{\alpha}}  \tag{5.2a}\\
  \frac{d q^{\alpha}}{d t} & =\dot{q}^{\alpha} \tag{5.2b}
\end{DispWithArrows}

\todo{Ahora no hay ecuaciones de segundo grado, solo $2n$ ecuaciones de primer grado}
Excepto por el hecho de que $L$ y $\dot{q}^{\alpha}$ son todas funciones de ( $q, \dot{q}, t$ ) ($\dot{q}^{\alpha}$ trivialmente) en lugar de $(q, p, t)$, estas ecuaciones tienen la forma simple deseada. Para obtener precisamente la forma deseada sólo se requiere que los lados derechos se escriban en términos de $(q, p, t)$.

En principio, esto es fácil de hacer para la Ec. (5.2b): escribir el $\dot{q}^{\alpha}$ en el lado derecho como funciones de $(q, p, t)$ invirtiendo (5.1). Supongamos que esto se puede hacer (vamos a encontrar las condiciones necesarias para ello más adelante), y llamar a las funciones resultantes $\dot{q}^{\alpha}(q, p, t)$. El lado derecho de la Ec. (5.2a) puede transformarse de forma similar: en todos los $\partial L / \partial q^{\beta}$ sustituir los $\dot{q}^{\alpha}$ dondequiera que aparezcan por $\dot{q}^{\alpha}(q, p, t)$. Sería más fácil hacer primero la sustitución en $L$, convirtiendo $L$ en una función de ( $q, p, t$ ), y luego tomar las derivadas. Esto requiere cuidado, sin embargo, porque la derivada parcial de $L(q, \dot{q}, t)$ con respecto a $q^{alpha}$ no es la misma que la de \todo{$\hat{L}$ es igual que $L$, pero en este caso $\dot{q}$ no es un parámetro libre, si no que es función de $(q,p,t)$} $\hat{L}(q, p, t) \equiv L(q, \dot{q}(q, p, t), t)$, ya que $L$ es una función de $(q, \dot{q}, t)$, y $\hat{L}$ es una función de $(q, p, t)$. Al tomar derivadas de $L$, las $\dot{q} s$ se mantienen fijas, mientras que en $\hat{L}$ las $p$ s se mantienen fijas.
fijas. Comparamos estos dos conjuntos de derivadas:

\[\frac{\partial \hat{L}}{\partial q^{\alpha}}=\frac{\partial L}{\partial q^{\alpha}}+\frac{\partial L}{\partial \dot{q}^{\beta}} \frac{\partial \dot{q}^{\beta}}{\partial q^{\alpha}}=\frac{\partial L}{\partial q^{\alpha}}+p_{\beta} \frac{\partial \dot{q}^{\beta}}{\partial q^{\alpha}}\]

donde hemos utilizado (5.1). El último término del lado derecho es la derivada de la función $\dot{q}^{\beta}(q, p, t)$. Ahora ponga todas las funciones de $(q, p, t)$ en un lado de la ecuación para obtener

\[\frac{\partial L}{\partial q^{\alpha}}=\frac{\partial}{\partial q^{\alpha}}\left[\hat{L}(q, p, t)-p_{\beta} \dot{q}^{\beta}(q, p, t)\right]\]

También será necesaria la derivada de $\hat{L}$ con respecto a $p_{\alpha}$. Esto es

\[\frac{\partial \hat{L}}{\partial p_{\alpha}}=\frac{\partial L}{\partial \dot{q}^{\beta}} \frac{\partial \dot{q}^{\beta}}{\partial p_{\alpha}}=p_{\beta} \frac{\partial \dot{q}^{\beta}}{\partial p_{\alpha}}\]

o (de nuevo poner todas las funciones de $(q, p, t)$ en un lado)

\begin{equation*}
  \frac{\partial}{\partial p_{\alpha}}\left[\hat{L}(q, p, t)-p_{\beta} \dot{q}^{\beta}(q, p, t)\right]=-\dot{q}^{\alpha} \tag{5.3}
  \end{equation*}

  Esta ecuación parece la solución que se obtendría para $\dot{q}^{text {ox }}$ invirtiendo (5.1), pero en realidad no lo es, ya que el lado izquierdo depende de $\dot{q}^{\beta}$. En su lugar es una ecuación diferencial para $\dot{q}^{\alpha}$ como una función de $(q, p, t)$.

  La función 
  
  \begin{equation*}
    H(q, p, t) \equiv p_{\beta} \dot{q}^{\beta}(q, p, t)-\hat{L}(q, p, t) \tag{5.4}
    \end{equation*}

aparece en las ecuaciones (5.3) y (5.4). Esta función tan importante se denomina función Hamiltoniana, o simplemente Hamiltoniana.


Con la ayuda del Hamiltoniano, las Ecs. (5.2a, b) pueden ahora escribirse enteramente en términos de $(q, p, t)$ :


\[\left.\begin{array}{l}
  \dot{q}^{\beta}=\frac{\partial H}{\partial p_{\beta}}  \tag{5.5}\\
  \dot{p}_{\beta}=-\frac{\partial H}{\partial q^{\beta}}
  \end{array}\right\}\]

Se denominan \textit{ecuaciones canónicas de Hamilton} \sidenote{El término «canónico» significa estándar o convencional (según los cánones). Se utiliza técnicamente para los sistemas dinámicos, en particular en el formalismo hamiltoniano.}. Son las ecuaciones del movimiento en el formalismo hamiltoniano. Su solución da expresiones locales para las trayectorias
$(q(t), p(t))$ en $\mathbf{T}^{*} \mathbb{Q}$, la multiplicidad $(q, p)$. Esta variedad se denomina haz cotangente (que se discutirá en la Sección 5.2). Las proyecciones de estas trayectorias hacia $\mathbb{Q}$ (esencialmente ignorando la parte $p(t)$) son las trayectorias del colector de configuración $q(t)$. Dejamos para la Sección 5.2 una discusión más detallada de $\mathbf{T}^{*} \mathbb{Q}$ y una explicación de la diferencia entre éste y $\mathbf{T Q}$.


\subsubsection{Energía}

Es de destacar que el Hamiltoniano es la variable dinámica que hemos llamado $E$, escrito ahora en términos de $(q, p)$ en lugar de $(q, \dot{q})$. Recordemos que para muchos sistemas dinámicos $E$ es la energía, y por tanto cuando $L=T-V$ el Hamiltoniano es $H=T+V$ expresado en términos de $(q, p)$. En los casos comunes en que $V$ es independiente de la $\dot{q}^{\beta}$ sólo es necesario escribir la energía cinética $T$ en términos de $(q, p)$. Sin embargo, no siempre es tan sencillo, como se verá en el ejemplo práctico 5.1 del Hamiltoniano para una partícula cargada en un campo electromagnético.

En el formalismo lagrangiano teniamos la cantidad conservada, asociada en algunos casos a la energía, que deonotabamos como $h(q,\dot{q},t)$ con la forma 
\[h(q,\dot{q},t)=p_{\alpha}\dot{q}^{\alpha}-L\]
En el formalismo Hamiltoniano, esta cantidad coincide con el Hamiltoniano en si, es decir,

\[h(q,\dot{q},t)=p_{\alpha}\dot{q}^{\alpha}-L=H \text{ con }\dot{q}^{\alpha}(q,p)\]

\begin{proof}
  Esto se puede demostrar sabiendo que $\pdv{H}{t}=-\pdv{L}{t}$ como 
  \[\dv{H}{t}=\overbrace{\pdv{H}{q^{a}}\pdv{q^{a}}}^{-p_{a}\dot{q}^{a}} + \overbrace{\pdv{H}{p_{a}}\pdv{}{}}^{\dot{q}^{a}p_{a}}+\pdv{H}{t}\Rightarrow \dv{H}{t}=\pdv{H}{t}=0 \]
\end{proof}



\subsubsection{Receta Para construir el Hamiltoniano}

La receta para escribir las ecuaciones de Hamilton para un sistema cuya Lagrangiana es conocida es entonces la siguiente\sidenote{Dentro de cada apartado pondré como explico Prado cada paso para un \textbf{caso de interés} que ya desarrolllamos en su momento es cuando el Lagrangiano tiene la forma \\
\[L=\underbrace{\frac{1}{2}T_{ab}(q,t) \dot{q}^{a}\dot{q}^{b}}_{L_{2}}+\underbrace{A_{a}(q, t)\dot{q}^{a}}_{L_{1}}-\underbrace{V(q,t)}_{L_{0}}\],\\ es decir
$L=L_{2}+L_{1}+L_{0}$}:
\begin{itemize}
  \item Utilizar (5.1) para calcular $p_{\alpha}$ en términos de $(q, \dot{q}, t)$.
  \item Invertir estas ecuaciones para obtener el $\dot{q}^{\alpha}$ en términos de $(q, p, t)$.
  \item Insertar el $\dot{q}^{{\alpha}}(q, p, t)$ en la expresión de $L$ para obtener $\hat{L}$.
  \item Utilizar (5.4) para obtener una expresión explícita para $H(q, p, t)$.
  \item Tomando las derivadas de $H$, escribir (5.5).
\end{itemize}

La receta funciona sólo si el paso $i i$ es posible, es decir, sólo si la Ecs. (5.1) se puede resolver de forma única para el $\dot{q}^{\alpha}$. De acuerdo con el teorema de la función implícita una condición necesaria y suficiente para la invertibilidad localmente es que el hessiano $\partial^{2} L / \partial \dot{q}^{\alpha} \partial \dot{q}^{\beta}$ sea no singular. Nos encontramos con el mismo requisito en la sección 2.2.2. Como regla general supondremos que se cumple, aunque habrá ocasiones en que falle en subconjuntos pequeños.

Las ecuaciones canónicas de Hamilton se han derivado de las ecuaciones de Euler-Lagrange, que a su vez se han derivado de las ecuaciones de movimiento de Newton. Hasta ahora el único método que tenemos para hallar la función hamiltoniana es pasando por la lagrangiana, pero a menudo es tan sencillo desde cero escribir una como la otra. Más adelante hablaremos de ello.



\begin{example}[Dado el lagrangiano: $L=\frac{1}{2} m\left[\dot{r}^{2}+r^{2} \dot{\theta}^{2}+\left(r^{2} \sin ^{2} \theta\right) \dot{\phi}^{2}\right]-V(r) $ \\(a) Obtenga el Hamiltoniano y (b) escriba las ecuaciones canónicas de Hamilton para una partícula cargada en un campo electromagnético. Proyectarlas hasta $\mathbb{Q}$ y demostrar que son las ecuaciones usuales para una partícula en un campo electromagnético.]

  Según la Ecuación (5.1): 
  \begin{DispWithArrows}[displaystyle, format=c] 
    p_{r}=m \dot{q}^{\gamma}+e A_{\gamma} \tag{5.6} 
  \end{DispWithArrows} 
  que es el momento generalizado canónicamente conjugado a $q^{\gamma}$. El momento generalizado no es el momento dinámico $m \dot{q}^{\gamma}$, sino que tiene una contribución que proviene del campo electromagnético.

  Para obtener el hamiltoniano, resuelve esta ecuación para $\dot{q}^{\gamma}$ e insértalo en la Ecuación (5.4). El resultado es (usamos $\delta^{\alpha \beta}$ para forzar que las sumas sean sobre pares superíndice-subíndice): \begin{DispWithArrows}[displaystyle, format=c] 
    H(q, p, t)=\frac{1}{2 m} \delta^{\alpha \gamma}\left\{p_{\alpha}-e A_{\alpha}(q, t)\right\}\left\{p_{\gamma}-e A_{\gamma}(q, t)\right\}+e \varphi(q, t) \tag{5.7} 
  \end{DispWithArrows}
  
  (b) Las ecuaciones canónicas obtenidas de este hamiltoniano son: 
  
  \begin{DispWithArrows}[format=c, displaystyle]
    \dot{q}^{\beta}=\frac{1}{m} \delta^{\alpha \beta}\left\{p_{\alpha}-e A_{\alpha}(q, t)\right\}, \tag{5.8a} \\
     \dot{p}{\beta}=\frac{e}{m} \delta^{\alpha \gamma} \frac{\partial A{\alpha}}{\partial q^{\beta}}\left\{p_{\gamma}-e A_{\gamma}(q, t)\right\}-\frac{\partial \varphi}{\partial q^{\beta}}. \tag{5.8b}
  \end{DispWithArrows}
  
  Las ecuaciones de movimiento en $\mathbb{Q}$ se obtienen eliminando $p_{\beta}$ de las Ecuaciones (5.8a, b). Esto se hace tomando las derivadas temporales de $\dot{q}^{\alpha}$ en (5.8a) y reemplazando los $\dot{p}{\alpha}$ que aparecen por sus expresiones en (5.8b). La Ecuación (5.8a) da como resultado: \begin{DispWithArrows}[displaystyle, format=c] \ddot{q}^{\beta}=\frac{1}{m} \delta^{\alpha \beta}\left(\dot{p}{\alpha}-e \partial_{y} A_{\alpha} \dot{q}^{\gamma}-e \partial_{y} A_{\alpha}\right) \end{DispWithArrows}
  
  donde $\partial_{\gamma}=\partial / \partial q^{\gamma}$ y $\partial_{t}=\partial / \partial t$. Ahora, se usa la Ecuación (5.8b) para eliminar $\dot{p}{\alpha}$, pero el $p_{y}$ en el lado derecho debe ser reemplazado por su expresión en términos de $(q, \dot{q})$, es decir, por la Ecuación (5.6). Entonces, la ecuación para $\ddot{q}^{\beta}$ se convierte en (aquí ya no nos preocupamos de si los índices son superíndices o subíndices): 
  \begin{DispWithArrows}[displaystyle, format=c] 
    m \ddot{q}^{\beta}=e\left\{\left(\partial_{\beta} A_{\gamma}-\partial_{\gamma} A_{\beta}\right) \dot{q}^{\gamma}-\partial_{t} A_{\beta}-\partial_{\beta} \varphi\right\} \tag{5.9} 
  \end{DispWithArrows}
  
  Se deduce de la Sección 2.2.4 que $\partial_{\beta} A_{\gamma}-\partial_{y} A_{\beta}=\epsilon_{\beta \gamma \alpha} B_{\alpha}$ y $-\partial_{\psi} A_{\alpha}-\partial_{\alpha} \varphi=E_{\alpha}$, donde $\mathbf{B}=\left(B_{1}, B_{2}, B_{3}\right)$ es el campo magnético y $\mathbf{E}=\left(E_{1}, E_{2}, E_{3}\right)$ es el campo eléctrico (no debe confundirse con la energía). Así que: 
  \begin{DispWithArrows}[displaystyle, format=c] 
    m \ddot{q}^{\beta}=e\left\{\epsilon_{\gamma \alpha \beta} \dot{q}^{\gamma} B_{\alpha}+E_{\beta}\right\} \end{DispWithArrows}
  
  lo cual es la forma en componentes de la fuerza de Lorentz $\mathbf{F} \equiv m \mathbf{a}=e(\mathbf{v} \wedge \mathbf{B}+\mathbf{E})$, que es el resultado requerido.
  
\end{example}


\begin{example}[Dado el lagrangiano: \[L(q, \dot{q}, t)=\frac{1}{2} m \dot{q}^{\beta} \dot{q}^{\beta}-e \varphi(q, t)+e \dot{q}^{\beta} A_{\beta}(q, t)\] \\ (a) Obtén el Hamiltoniano y las ecuaciones canónicas para una partícula en un campo de fuerza central. (b) Toma dos de las condiciones iniciales como \(p_{\phi}(0)=0\) y \(\phi(0)=0\) (esto es esencialmente la elección de un sistema de coordenadas esféricas particular). Discute la simplificación resultante de las ecuaciones canónicas. ]

  Denota los momentos conjugados por \(p_{r}, p_{\theta}\) y \(p_{\phi}\). Las ecuaciones definitorias para ellos son
\begin{DispWithArrows}[format=ll, displaystyle]
  p_{r} & \equiv \frac{\partial L}{\partial \dot{r}}=m \dot{r}  \tag{5.11}\\
  p_{\theta} & \equiv \frac{\partial L}{\partial \dot{\theta}}=m r^{2} \dot{\theta} \\
  p_{\phi} & \equiv \frac{\partial L}{\partial \dot{\phi}}=m\left(r^{2} \sin ^{2} \theta\right) \dot{\phi}
\end{DispWithArrows}

Invertir estas ecuaciones da
\begin{DispWithArrows}[format=c, displaystyle]
  \dot{r}=\frac{p_{r}}{m}  \tag{5.12}\\
  \dot{\theta}=\frac{p_{\theta}}{m r^{2}} \\
  \dot{\phi}=\frac{p_{\phi}}{m r^{2} \sin ^{2} \theta}
\end{DispWithArrows}

El Hamiltoniano es
\begin{DispWithArrows}[format=c, displaystyle]
  H=p_{\alpha} \dot{q}^{\alpha}-L=\frac{p_{r}^{2}}{2 m}+\frac{p_{\theta}^{2}}{2 m r^{2}}+\frac{p_{\phi}^{2}}{2 m r^{2} \sin ^{2} \theta}+V(r) \equiv T+V \tag{5.13}
\end{DispWithArrows}
donde la energía cinética $T$ es la suma de los tres primeros términos. Las ecuaciones canónicas de Hamilton en estas variables son
\begin{DispWithArrows}[format=ll, displaystyle]
  \dot{r} \equiv \frac{\partial H}{\partial p_{r}}=\frac{p_{r}}{m}, & \dot{p}_{r} \equiv-\frac{\partial H}{\partial r}=\frac{1}{m r^{3}}\left[p_{\theta}^{2}+\frac{p_{\phi}^{2}}{\sin ^{2} \theta}\right]-V^{\prime}(r) \\
  \dot{\theta} \equiv \frac{\partial H}{\partial p_{\theta}}=\frac{p_{\theta}}{m r^{2}}, & \dot{p}_{\theta} \equiv-\frac{\partial H}{\partial \theta}=\frac{p_{\phi}^{2} \cos \theta}{m r^{2} \sin ^{3} \theta}  \tag{5.14}\\
  \dot{\phi} \equiv \frac{\partial H}{\partial p_{\phi}}=\frac{p_{\phi}}{m r^{2} \sin ^{2} \theta}, & \dot{p}_{\phi} \equiv-\frac{\partial H}{\partial \phi}=0
\end{DispWithArrows}

Comentario: Nota que las tres ecuaciones a la izquierda siguen inmediatamente de las definiciones de los momentos generalizados. Esto siempre ocurre con las ecuaciones canónicas de Hamilton: al estudiar su derivación queda claro que la mitad de ellas son simplemente inversiones de esas definiciones.

(b) Si $p_{\phi}(0)=0$, la última de las Eqs. (5.14) implica que $p_{\phi}(t)=0$ para todo $t$. Entonces las ecuaciones canónicas se simplifican.

De $\phi(0)=0$ y estas ecuaciones se sigue que $\phi(t)=0$ para todo $t$: el movimiento se mantiene en el plano $\phi=0$. El resultado es un sistema de dos grados de libertad que consiste en las dos primeras líneas de (5.15). La cuarta ecuación implica que $p_{\theta}$ es una constante del movimiento. Esta es el momento angular en el plano $\phi=0$, llamado $l$.
  
\end{example}